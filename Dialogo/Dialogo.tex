En la presente sección, usamos la teoría de funtores adjuntos para probar que $\Lambda$ y $\Gamma$ determinan una adjunción, y también para, a partir de esto, derivar la equivalencia de las categorías $\Sh(X)$ y $\Etale(X)$, para cada espacio topológico $X$, que constituye nuestro objetivo. Los resultados que usemos acerca de la teoría de funtores adjuntos se presentarán sin prueba, pues su estudio no constituye directamente un propósito del presente escrito (para referencia puede visitarse \cite[Chapter~4]{CWM}, por ejemplo).

\phantom{O}

Buscamos aplicar el siguiente lema a las categorías $PreSh(X)$ y $\Bund(X)$, pasando a las respectivas subcategorías $\Sh(X)$ y $\Etale(X)$. Lo enunciamos directamente con nuestras categorías y funtores de interés:

\begin{Lema}\label{Lema:LemaAdjuntos}
   Supongamos que los funtores $\Lambda:PreSh(X)\to\Top/X$ y $\Gamma:\Top/X\to PreSh(X)$ son adjuntos, con unidad $\eta$ y counidad $\epsilon$ que satisfacen las siguientes condiciones:
   \begin{itemize}
      \item Para cada $B\in\Top/X$, $\eta_{\Gamma B}:\Gamma B \to \Gamma \Lambda \Gamma B$ es un isomorfismo.
      \item Para cada $P\in\PreSh(X)$, $\epsilon_{\Lambda P}: \Lambda \Gamma \Lambda P \to \Lambda P$ es un isomorfismo.
   \end{itemize}
   Sea $\mathcal{P}_{0}$ la subcategoría plena de $\PreSh(X)$ que tiene por objetos todos aquellos $P\in\PreSh(X)$ que son isomorfos a algún $\Gamma B$. Sea $\mathcal{B}_{0}$ la subcategoría plena de $\mathcal{B}$ cuyos objetos son isomorfos a algún $\Lambda_P$. Entonces la adjunción $(\Lambda, \Gamma)$ se restringe a una equivalencia de categorías $(\Lambda_{0}, \Gamma_{0})$, como en el siguiente diagrama:
   % https://tikzcd.yichuanshen.de/#N4Igdg9gJgpgziAXAbVABwnAlgFyxMJZABgBpiBdUkANwEMAbAVxiRAB12BbOnACwDGjYAAUAvgH1iIMaXSZc+QijIBGKrUYs2nHvyENRYmXJAZseAkVXkN9Zq0QduvQcIBCk6bPkWl10nVqe20ndxkNGCgAc3giUAAzACcILiQyEBwIJAAmajg+LAScJABaG00HHXYAGTouACMoOikTRJS0xDzM7MQMgqKSxHLgrUdnAHF6nlafEGTU9OospAqQ8c58HBajEGoGOgaYBhEFS2UQJKxovhK5hc7ulcQAZlGqp02IbYlgTzb5h1Vstem8QANimU1mNqnVGs0AQ8kGDnhUIUMRpVQpNpnQImIgA
\begin{center}
\begin{tikzcd}
\mathcal{P}_0 \arrow[r, "\Lambda_0", shift left] \arrow[d, "\iota_{P}"'] & \mathcal{B}_0 \arrow[l, "\Gamma_0", shift left] \arrow[d, "\iota_{B}"] \\
\mathcal{P} \arrow[r, "\Lambda", shift left]                             & B \arrow[l, "\Gamma", shift left]                                     
\end{tikzcd}
\end{center}

   donde las flechas verticales son las respectivas inclusiones.
\end{Lema}

Sabemos que $\Sh(X)$ y $\Etale(X)$ son subcategorías plenas de $\PreSh(X)$ y $\Top/X$; además, en la Sección \ref{section:BundToPreSh} probamos que cada manojo $\langle Y,p\rangle$ sobre $X$ determina un haz (de secciones transversales) $\Gamma Y$ (es decir $\Gamma Y\in \Sh(X)$); igualmente en la Sección \ref{section:PreShToBund} probamos que cada prehaz $P$ sobre $X$ determina un espacio étalé $\Lambda P$ sobre $X$ (es decir, $\Lambda_P\in\Etale(X))$. Por tanto, las categorías $\Sh(X)$ y $\Etale(X)$ son apliclables al anterior lema en función de $\mathcal{P}_0$ y $\mathcal{B}_0$, respectivamente.
\subsection{La transformación natural $\eta$}
   Es la acción de ir a $\Top/X$ por $\Lambda$ y volver a $PreSh(X)$ mediante $\Gamma$ lo que genera la transformación natural $\eta$. Igualmente, $\eta$ surge con el llamado proceso de hacificación. Dado $P$ un prehaz sobre $X$, tenemos que $\Lambda_P$ es un manojo sobre $X$ y $\Gamma_{\Lambda_P}$ es un haz sobre $X$: el haz de secciones transversales del manojo $\Lambda_P$. Así, tenemos un funtor $\Gamma_\Lambda: \PreSh(X)\to\Sh(X)$ de prehaces en haces sobre $X$. Éste funtor es llamado \textit{funtor de hacificación}, y si $P$ es un prehaz sobre $X$, decimos que $\Gamma_{\Lambda_P}$ es la hacificación de $P$ y corresponde a la ``mejor aproximación" de $P$ a un haz. Esta ``mejor aproximación" también debería implicar que si $P$ es ya un haz, entonces el haz de secciones $\Gamma_{\Lambda_P}$ difiera lo menos posible de $P$. De eso trata el siguiente teorema, que además nos dice que todo haz es un haz de secciones transversales:
\begin{Tma}[Teorema de hacificación]
   Si $P$ es un haz sobre $X$ entonces en la categoría $Sh(X)$ se tiene un isomorfismo $P\cong \Gamma_{\Lambda_P}$.
\end{Tma}
\begin{proof}
   Para probar este teorema se construye, para cada $P\in \PreSh(X)$ una transformación natural de prehaces $\eta_{P}:P\to\Gamma_{\Lambda_P}$, cuyas componentes sean biyecciones; esto basta, pues una transformación natural con componentes en $\Set$ biyectivas es un isomorfismo natural, y por tanto se obtiene $P\cong \Gamma_{\Lambda_P}$. La definición de $\eta_P$ se hace a partir de sus componentes: para cada $U\in \mathcal{O}(X)$, $\eta_{P} U:PU\to (\Gamma_{\Lambda_P})(U)$ le asigna a cada $s\in PU$ la sección transversal $\dot{s}$ de $U$ en $\Lambda_P$.
   % https://tikzcd.yichuanshen.de/#N4Igdg9gJgpgziAXAbVABwnAlgFyxMJZABgBpiBdUkANwEMAbAVxiRAAUBVEAX1PUy58hFAEZyVWoxZsAFAB15AGToBbAEZQ6AfWCKA4mtU72PAJTc+A7HgJEyoyfWatEIBFZAYbwouMfUzjJuilAQOMBwPLySMFAA5vBEoABmAE4QqkhkIDgQSOJSLmyKMDgmAASW-CDpmUgATNR52dQMdOowDOyCtiIgWGDYsCCB0q4gioO8NXVZiADMzfmIhUETU4RtHV09PnZug8OsPBQ8QA
\begin{center}
\begin{tikzcd}
PU \arrow[r, "\eta_P U"]       & (\Lambda_{\Gamma_P})U                \\
s \arrow[u, "\in" description, blue] & \dot{s} \arrow[u, "\in" description,blue]
\end{tikzcd}
\end{center}

   Esta asignación para $\eta_{P}U$ es inyectiva y sobreyectiva, y hace de $\eta_P$ un isomorfismo natural de $P$ en $\Gamma_{\Lambda_P}$.
\end{proof}
La asignación de $\eta_P$ para cada $P\in \PreSh (X)$ dada en la anterior prueba es lo que define, a partir de sus componentes, la transformación natural $\eta :1_{\PreSh(X)}\to \Gamma \Lambda$:
% https://tikzcd.yichuanshen.de/#N4Igdg9gJgpgziAXAbVABwnAlgFyxMJZABgBpiBdUkANwEMAbAVxiRAAUQBfU9TXfIRQBGclVqMWbADrSA4nQC2iugH1gsgDJKARlDXsu3XiAzY8BIqOHj6zVohCyFyugAItu-W848+5wSIyG2o7KUdhdVl2ACcYAGUACwAKAA0ASi5fcRgoAHN4IlAAMxiIRSQyEBwIJFEJexlpGBw1YENjErKKxABmahrK0MkHEABeEGoGOh0YBnZ+CyEQLDBsWE6QUvKkACYB2sR6sNGJqZm5hYDLR1X11i4KLiA
\begin{center}
\begin{tikzcd}
P \arrow[r, "\eta_{P}"]                   & \Gamma_{\Lambda_P}                          \\
1_{\PreSh(X)}P \arrow[u, "=" description] & \Gamma \Lambda P \arrow[u, "=" description]
\end{tikzcd}
\end{center}

Como corolario del Teorema de hacificación obtenemos que $\Lambda$ y $\Gamma$ satisfacen la propiedad \textbf{(L1)} del Lema \ref{Lema:LemaAdjuntos}: dado $B\in \Top/X$ cualquiera, sabemos que $\Gamma B$ es un haz sobre $X$, luego 
$$
   \eta_{\Gamma B}:\Gamma B\dot{\to}\Gamma \Lambda \Gamma B
$$
es un isomorfismo. 

\subsection{La transformación natural $\epsilon$}
   Es la acción de ir a $\PreSh(X)$ por $\Gamma$ y volver a $\Top/X$ por $\Lambda$ lo que genera la transformación natural $\epsilon$. Deseamos que $\epsilon$ sea una transformación natural entre los funtores $\Lambda \Gamma$ y $1_{\Top/X}$ de $\Top/X$ en $\Top/X$. Dado un manojo $p:Y\to X$ sobre $X$, debemos definir un manojo $\epsilon Y:\Lambda \Gamma Y \to Y$. Notemos que
$$
   \Lambda \Gamma Y = \Lambda_{\Gamma Y} = \bigcup \mathcal{B}_{\Gamma Y} = \bigcup_{\substack{U\in\mathcal{O}(X)\\s\in(\Gamma Y)U}}\dot{s}(U)
$$
de modo que un elemento de $\Lambda \Gamma Y$ es de la forma $\dot{s}(x)$ para algunos $U\in\mathcal{O}(X)$, $x\in U$ y $s\in(\Lambda Y)(U)$. Observando esto es natural, para cada manojo $p:Y\to X$, definir $\epsilon Y:\Lambda \Gamma Y\to 1_{\Top/X}$ mediante $\epsilon Y (\dot{s}(x))=s(x)\in Y$. La representación de $\dot{s}(x)$ en $\Lambda \Gamma Y$ puede no ser única, esto es, pueden existir $V\in \mathcal{O}(X)$ y $t\in(\Gamma Y)V$ tales que $\dot{s}(x)=\dot{t}(x)$, es decir, $s$ y $t$ tienen el mismo germen en $x$, pero esto implica que en alguna vecindad de $x$, $s=t$, y en particular $s(x)=t(x)$ con lo cual $\epsilon_{Y}$ está bien definida. Esta asignación de $\epsilon_{Y}$ para cada $Y\in\Top/X$ hacen de $\epsilon$ una transformación natural $\Lambda \Gamma \dot{\to} 1_{\Top/X}$.
%         \bigcup_{\substack{U\in\mathcal{O}(X) \\ s\in PU}}\dot{s}(U)=\Lambda_P.

