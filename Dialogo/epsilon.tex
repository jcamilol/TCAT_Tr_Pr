Es la acción de ir a $\PreSh(X)$ por $\Gamma$ y volver a $\Top/X$ por $\Lambda$ lo que genera la transformación natural $\epsilon$. Deseamos que $\epsilon$ sea una transformación natural entre los funtores $\Lambda \Gamma$ y $1_{\Top/X}$ de $\Top/X$ en $\Top/X$. Dado un manojo $p:Y\to X$ sobre $X$, debemos definir un manojo $\epsilon Y:\Lambda \Gamma Y \to Y$. Notemos que
$$
   \Lambda \Gamma Y = \Lambda_{\Gamma Y} = \bigcup \mathcal{B}_{\Gamma Y} = \bigcup_{\substack{U\in\mathcal{O}(X)\\s\in(\Gamma Y)U}}\dot{s}(U)
$$
de modo que un elemento de $\Lambda \Gamma Y$ es de la forma $\dot{s}(x)$ para algunos $U\in\mathcal{O}(X)$, $x\in U$ y $s\in(\Lambda Y)(U)$. Observando esto es natural, para cada manojo $p:Y\to X$, definir $\epsilon Y:\Lambda \Gamma Y\to 1_{\Top/X}$ mediante $\epsilon Y (\dot{s}(x))=s(x)\in Y$. La representación de $\dot{s}(x)$ en $\Lambda \Gamma Y$ puede no ser única, esto es, pueden existir $V\in \mathcal{O}(X)$ y $t\in(\Gamma Y)V$ tales que $\dot{s}(x)=\dot{t}(x)$, es decir, $s$ y $t$ tienen el mismo germen en $x$, pero esto implica que en alguna vecindad de $x$, $s=t$, y en particular $s(x)=t(x)$ con lo cual $\epsilon_{Y}$ está bien definida. Esta asignación de $\epsilon_{Y}$ para cada $Y\in\Top/X$ hacen de $\epsilon$ una transformación natural $\Lambda \Gamma \dot{\to} 1_{\Top/X}$.
%         \bigcup_{\substack{U\in\mathcal{O}(X) \\ s\in PU}}\dot{s}(U)=\Lambda_P.
