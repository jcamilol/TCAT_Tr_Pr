Es la acción de ir a $\Top/X$ por $\Lambda$ y volver a $PreSh(X)$ mediante $\Gamma$ lo que genera la transformación natural $\eta$. Igualmente, $\eta$ surge con el llamado proceso de hacificación. Dado $P$ un prehaz sobre $X$, tenemos que $\Lambda_P$ es un manojo sobre $X$ y $\Gamma_{\Lambda_P}$ es un haz sobre $X$: el haz de secciones transversales del manojo $\Lambda_P$. Así, tenemos un funtor $\Gamma_\Lambda: \PreSh(X)\to\Sh(X)$ de prehaces en haces sobre $X$. Éste funtor es llamado \textit{funtor de hacificación}, y si $P$ es un prehaz sobre $X$, decimos que $\Gamma_{\Lambda_P}$ es la hacificación de $P$ y corresponde a la ``mejor aproximación" de $P$ a un haz. Esta ``mejor aproximación" también debería implicar que si $P$ es ya un haz, entonces el haz de secciones $\Gamma_{\Lambda_P}$ difiera lo menos posibles de $P$. De eso trata el siguiente teorema, que además nos dice que todo haz es un haz de secciones transversales:
\begin{Tma}[Teorema de hacificación]
   Si $P$ es un haz sobre $X$ entonces en la categoría $Sh(X)$ se tiene un isomorfismo $P\cong \Gamma_{\Lambda_P}$.
\end{Tma}
\begin{proof}
   Para probar este teorema se construye, para cada $P\in \PreSh(X)$ una transformación natural de prehaces $\eta_{P}:P\to\Gamma_{\Lambda_P}$, cuyas componentes sean biyecciones; esto basta, pues una transformación natural con componentes en $\Set$ biyectivas es un isomorfismo natural, y por tanto se obtiene $P\cong \Gamma_{\Lambda_P}$. La definición de $\eta_P$ se hace a partir de sus componentes: para cada $U\in \mathcal{O}(X)$, $\eta_{P} U:PU\to (\Gamma_{\Lambda_P})(U)$ le asigna a cada $s\in PU$ la sección transversal $\dot{s}$ de $U$ en $\Lambda_P$.
   % https://tikzcd.yichuanshen.de/#N4Igdg9gJgpgziAXAbVABwnAlgFyxMJZABgBpiBdUkANwEMAbAVxiRAAUBVEAX1PUy58hFAEZyVWoxZsAFAB15AGToBbAEZQ6AfWCKA4mtU72PAJTc+A7HgJEyoyfWatEIBFZAYbwouMfUzjJuilAQOMBwPLySMFAA5vBEoABmAE4QqkhkIDgQSOJSLmyKMDgmAASW-CDpmUgATNR52dQMdOowDOyCtiIgWGDYsCCB0q4gioO8NXVZiADMzfmIhUETU4RtHV09PnZug8OsPBQ8QA
\begin{center}
\begin{tikzcd}
PU \arrow[r, "\eta_P U"]       & (\Lambda_{\Gamma_P})U                \\
s \arrow[u, "\in" description, blue] & \dot{s} \arrow[u, "\in" description,blue]
\end{tikzcd}
\end{center}

   Esta asignación para $\eta_{P}U$ es inyectiva y sobreyectiva, y hace de $\eta P$ un isomorfismo natural de $P$ en $\Gamma_{\Lambda_P}$.
\end{proof}
La asignación de $\eta_P$ para cada $P\in \PreSh (X)$ dada en la anterior prueba es lo que define, a partir de sus componentes, la transformación natural $\eta :1_{\PreSh(X)}\to \Gamma \Lambda$:
% https://tikzcd.yichuanshen.de/#N4Igdg9gJgpgziAXAbVABwnAlgFyxMJZABgBpiBdUkANwEMAbAVxiRAAUQBfU9TXfIRQBGclVqMWbADrSA4nQC2iugH1gsgDJKARlDXsu3XiAzY8BIqOHj6zVohCyFyugAItu-W848+5wSIyG2o7KUdhdVl2ACcYAGUACwAKAA0ASi5fcRgoAHN4IlAAMxiIRSQyEBwIJFEJexlpGBw1YENjErKKxABmahrK0MkHEABeEGoGOh0YBnZ+CyEQLDBsWE6QUvKkACYB2sR6sNGJqZm5hYDLR1X11i4KLiA
\begin{center}
\begin{tikzcd}
P \arrow[r, "\eta_{P}"]                   & \Gamma_{\Lambda_P}                          \\
1_{\PreSh(X)}P \arrow[u, "=" description] & \Gamma \Lambda P \arrow[u, "=" description]
\end{tikzcd}
\end{center}

