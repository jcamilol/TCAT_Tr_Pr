\documentclass[letterpaper]{article} %único necesario para: crear documento, hacer título, espacios, interlineado

\usepackage{vmargin} %necesario para ajustar las márgenes con \setmargins
%\usepackage{euler} %usa la fuente "euler" para las ecuaciones
\usepackage{amsfonts} %para Z estilizada
\usepackage{pifont} %para más estilos de viñetas
\usepackage{amsmath} %necesario equation*
\usepackage{amsthm} %necesario para proof enviroment
\usepackage{dsfont} %alguna letra caligráfica
\usepackage{tipa} %Epsilon bonito
\usepackage{graphicx} %para manejar imágenes
\usepackage{wrapfig} %para imágenes en modo "wrap"
\usepackage{lipsum} %para usar el texto de relleno
\usepackage{bold-extra} %permite usar textsc

\newtheorem{theorem}{Teorema}
\numberwithin{theorem}{section}
\newtheorem{definition}{Definición}
\numberwithin{definition}{section}
\newtheorem{lemma}{Lema}
\newtheorem{exercise}{Ejercicio}[section]
\renewcommand*{\proofname}{\textbf{Prueba}} %Muestra "prueba" en lugar de "proof"
\renewcommand\refname{Referencias}

%letterpaper: 21.59cm x 27.94cm
\setmargins{2.8 cm} %margen izquierdo
{1.5 cm} %margen Superior
{15.5cm} %anchura del texto
{23cm} %altura del texto
{0pt} %altura de los encabezados
{1.5cm} %espacio entre el texto y los encabezados
{5 cm} %altura del pie de página
{1 cm} %espacio entre el texto y el pie de página

\addtolength{\skip\footins}{2pc}

%\pagenumbering{gobble} %Quita la numeración de las páginas
\renewcommand{\baselinestretch}{1.5} %Aumenta el interlineado a n veces el automático
\relpenalty=9999 %Evita que se rompan las ecuaciones en el cambio de renglón
\binoppenalty=9999
\renewenvironment{abstract}
{\small
   \list{}{%
      \setlength{\leftmargin}{1.4cm}% <---------- CHANGE HERE
      \setlength{\rightmargin}{\leftmargin}%
   }%
   \item\relax}
{\endlist}

\renewcommand{\abstractname}{\textsc{Resumen}}

\date{}

\begin{document}

   \normalsize{
    \noindent\textit{Teoría de categorías (2024-I)}

    \vspace{-0.1cm}
    \noindent Trabajo de profundización
    
    \vspace{-0.2cm}
}
\noindent\rule{10cm}{0.1pt}\\
\vspace{0.5cm}
\begin{center}
   \textsc{
      \Large{\textbf{Dos aproximaciones equivalentes a la noción de haz}}
   }\\ \vspace{0.7cm}
   \large{\textsc{Juan Camilo Lozano Suárez \footnote{Estudiante de pregrado en matemáticas, Universidad Nacional de Colombia.\\Email: jclozanos@unal.edu.co}}}\\ \vspace{0.5cm}
\end{center}
\hspace{3.5cm} \hrulefill \hspace{3.5cm}

   \vspace{1cm}

   \begin{abstract}
      \textsc{Resumen.} Introducimos la noción de haz de dos maneras en principio independientes; primero como un funtor contravariante con buenas propiedades de pegado y luego como espacio fibrado o étalé. Posteriormente probaremos que las categorías que cada una produce son equivalentes.

        
      \vspace{0.3cm}
      \textsc{Palabras Clave.} Integral de Lebesgue; teoremas de Levi; funciones escalonadas; funciones superiores; funciones Lebesgue-integrables.
   \end{abstract}

   \vspace{1cm}

   \section{Sección 1}

\end{document}
