\documentclass[letterpaper]{article} %único necesario para: crear documento, hacer título, espacios, interlineado

\usepackage{vmargin} %necesario para ajustar las márgenes con \setmargins
%\usepackage{euler} %usa la fuente "euler" para las ecuaciones
\usepackage{amsfonts} %para Z estilizada
\usepackage{pifont} %para más estilos de viñetas
\usepackage{amsmath} %necesario equation*
\usepackage{amsthm} %necesario para proof enviroment
\usepackage{dsfont} %alguna letra caligráfica
\usepackage{tipa} %Epsilon bonito
\usepackage{graphicx} %para manejar imágenes
\usepackage{wrapfig} %para imágenes en modo "wrap"
\usepackage{lipsum} %para usar el texto de relleno
\usepackage{bold-extra} %permite usar textsc

\usepackage{tikz-cd} %para los diagramas conmutativos

\newtheorem{Tma}{Teorema}
\newtheorem{Prop}{Proposición}
\numberwithin{Tma}{section}
\numberwithin{Prop}{section}
\newtheorem{Def}{Definición}
\numberwithin{Def}{section}
\newtheorem{Lema}{Lema}
\numberwithin{Lema}{section}
\newtheorem{Ejc}{Ejercicio}
\numberwithin{Ejc}{section}
\newtheorem{Ejm}{Ejemplo}
\numberwithin{Ejm}{section}

\renewcommand*{\proofname}{\textbf{Prueba}} %Muestra "prueba" en lugar de "proof"
\renewcommand\refname{Referencias}

%letterpaper: 21.59cm x 27.94cm
\setmargins{2.8 cm} %margen izquierdo
{1.5 cm} %margen Superior
{15.5cm} %anchura del texto
{23cm} %altura del texto
{0pt} %altura de los encabezados
{1.5cm} %espacio entre el texto y los encabezados
{5 cm} %altura del pie de página
{1 cm} %espacio entre el texto y el pie de página

\addtolength{\skip\footins}{2pc}

%\pagenumbering{gobble} %Quita la numeración de las páginas
\renewcommand{\baselinestretch}{1.5} %Aumenta el interlineado a n veces el automático
\relpenalty=9999 %Evita que se rompan las ecuaciones en el cambio de renglón
\binoppenalty=9999
\renewenvironment{abstract}
{\small
   \list{}{%
      \setlength{\leftmargin}{1.4cm}% <---------- CHANGE HERE
      \setlength{\rightmargin}{\leftmargin}%
   }%
   \item\relax}
{\endlist}

\renewcommand{\abstractname}{\textsc{Resumen}}

\date{}

\begin{document}

   \normalsize{
    \noindent\textit{Teoría de categorías (2024-I)}

    \vspace{-0.1cm}
    \noindent Trabajo de profundización
    
    \vspace{-0.2cm}
}
\noindent\rule{10cm}{0.1pt}\\
\vspace{0.5cm}
\begin{center}
   \textsc{
      \Large{\textbf{Dos aproximaciones equivalentes a la noción de haz}}
   }\\ \vspace{0.7cm}
   \large{\textsc{Juan Camilo Lozano Suárez \footnote{Estudiante de pregrado en matemáticas, Universidad Nacional de Colombia.\\Email: jclozanos@unal.edu.co}}}\\ \vspace{0.5cm}
\end{center}
\hspace{3.5cm} \hrulefill \hspace{3.5cm}

   \vspace{1cm}

   \begin{abstract}
      \textsc{Resumen.} Introducimos la noción de haz de dos maneras en principio independientes; primero como un funtor contravariante con buenas propiedades de pegado y luego como espacio fibrado o étalé. Posteriormente probaremos que las categorías que cada una produce son equivalentes.

        
      \vspace{0.3cm}
      \textsc{Palabras Clave.} Haz; espacio étalé; homeomorfismo local; manojo; hacificación; equivalencia de categorías; local vs global.
   \end{abstract}

   \vspace{1cm}

   \section{Haz como funtor}
      \subsection{Un ejemplo como motivación}
   Una constante en el quehacer matemático es el tránsito entre aspectos locales y aspectos globales. Consideremos un ejemplo enmarcado en el área de la topología. Sean $X$ un espacio topológico y $U$ un subconjunto abierto de $X$, al cual dotamos con un cubrimiento $\left\lbrace U_i\right\rbrace_{i\in I}$ de subconjuntos abiertos de $U$. Una función continua $f:U\to \mathbb{R}$ se presenta como una herramienta para entender globalmente el conjunto $U$, y fácilmente nos permite pasar al conocimiento local de $U$ en el siguiente sentido:
\begin{itemize}
   \item[\textbf{(P1)}] Si $V\stackrel{ab}\subseteq U$ entonces $f|^U_V:V\to\mathbb{R}$ (la restricción de $f$ de $U$ a $V$) es también una función continua. 
\end{itemize}
De forma recíproca, gracias al lema de pegado (Teorema \ref{Tma:lemaPegado}), un apropiado conocimiento local de $U$ nos permite pasar a un conocimiento global, en la siguiente forma:
\begin{itemize}
   \item[\textbf{(P2)}] Si $f:U\to \mathbb{R}$ es una función tal que $f|^{U}_{U_i}:U_i\to\mathbb{R}$ es continua para todo $i\in I$, entonces $f$ es continua.
\end{itemize}
Las propiedades \textbf{(P1)} y \textbf{(P2)} pueden ser capturadas en lenguaje categórico. Para esto, consideremos la categoría $\mathcal{O}(X)$ que tiene como objetos los subconjuntos abiertos de $X$, y en la cual, dados $U,V\in \mathcal{O}(X)$, hay una flecha de $V$ en $U$ si y solo si $V\subseteq U$; dicha flecha en $\mathcal{O}(X)$ (que será la única de $V$ en $U$) la representamos igualmente mediante ``$V\subseteq U$". Ahora, para cada $U\in\mathcal{O}(X)$ definimos el conjunto $CU$ de todas las funciones reales continuas sobre U:
$$
CU:=\left\lbrace f:U\to\mathbb{R}\mid f \text{ es continua}\right\rbrace,
$$
y para cualquier flecha $V\subseteq U$ en $\mathcal{O}(X)$, definimos la función de conjuntos
\begin{center}
% https://tikzcd.yichuanshen.de/#N4Igdg9gJgpgziAXAbVABwnAlgFyxMJZABgBpiBdUkANwEMAbAVxiRAGEAKANQB1e4TAEZwYOGAEcABAFUAlIhABfUuky58hFAEZyVWoxZt2M5apAZseAkQBMe6vWatF7bmbVXNRXdv1OjRQAzDwt1ay1kez9HQxcQIIAfAH13JX0YKABzeCJQIIAnCABbJF0QHAgkWxV8otLEAGZqSqQAFha6LAY2Yro0OFb0pSA
\begin{tikzcd}[row sep=-4pt, column sep=7pt]
C(V\subseteq U): & CU \arrow[r]         & CV   \\
                 & f \arrow[r, maps to] & f|^U_V
\end{tikzcd}
\end{center}

que a cada función continua de $U$ en $\mathbb{R}$ le asigna su respectiva función restricción al subconjunto $V$, que a su vez es una función continua de $V$ en $\mathbb{R}$. Tendremos entonces la siguiente propiedad:
\begin{Prop}\label{Prop:P1}
   La regla $C$ que a cada $U\in\mathcal{O}(X)$ le asigna el conjunto $CU$ y a cada flecha $V\subseteq U$ en $\mathcal{O}(X)$ le asigna la función restricción de V en U, $C(V\subseteq U): CU\to CV$, es un funtor contravariante de $\mathcal{O}(X)$ en \normalfont{\textbf{Set}}.
\end{Prop}
\begin{proof}
   \begin{itemize}
      \item Trivialmente se tiene que $C$ respeta identidades, pues para cualquier $U\in\mathcal{O}(X)$ tenemos
         \begin{center}
% https://tikzcd.yichuanshen.de/#N4Igdg9gJgpgziAXAbVABwnAlgFyxMJZABgBpiBdUkANwEMAbAVxiRAGEAKARgH0BVAJSIQAX1LpMufIRTdyVWoxZt2-MRJAZseAkQBMC6vWasRajZJ0yi87opMqRAM0tapu2ckP3jysyDOAD4CALyuooowUADm8ESgzgBOEAC2SPIgOBBI+uKJKemIAMzU2UgALGV0WAxsqXRocOWRokA
\begin{tikzcd}[row sep=-4pt, column sep=7pt]
C(1_U): & CU \arrow[r]         & CU     \\
        & f \arrow[r, maps to] & f|^U_U=f
\end{tikzcd}
\end{center}

         es decir, $C(1_U)=1_{C(U)}$.
      \item Supongamos que en $\mathcal{O}(X)$ tenemos $W\subseteq V\subseteq U$. Entonces $W\subseteq U$ y en \normalfont{\textbf{Set}} tenemos la función restricción de $U$ en $W$, $C(W\subseteq U):CU\to CW$. Tenemos ademas en \normalfont{\textbf{Set}} la composición $C(W\subseteq V)\circ C(V\subseteq U):CU\to CW$. Para cada $f\in CU$ se tiene
         $$
         \begin{aligned}
            (C(W\subseteq V)\circ C(V\subseteq U))(f)&=C(W\subseteq V)(C(V\subseteq U)(f))\\
                                                     &=C(W\subseteq V)(f|^U_V)\\
                                                     &=(f|^U_V)|^V_W\\
                                                     &=f|^U_W\\
                                                     &=C(W\subseteq U)(f),
         \end{aligned}
         $$
      con lo cual $C(W\subseteq V)\circ C(V\subseteq U) = C(W\subseteq U)$ y $C$ respeta composiciones.
      \end{itemize}
\end{proof}
Con lo anterior, podemos decir que $C$ es un \textbf{prehaz} (de conjuntos):
\begin{Def}[Prehaz]
   Un prehaz (de conjuntos) sobre un espacio topológico $X$ es un funtor contravariante de $\mathcal{O}(X)$ en \normalfont{\textbf{Set}}.
\end{Def}
La Proposición \ref{Prop:P1} permite capturar de manera categórica la propiedad \textbf{(P1)}. Para lograr hacer lo mismo con la propiedad \textbf{(P2)} introducimos el concepto de \textit{igualadores}.


   \section{Anexos}
      \begin{theorem}[Lema de pegado]\label{tma:lemaPegado}
   Sean $X$ y $Y$ espacios topológicos. Sean $U\stackrel{ab}\subseteq X$, $\{U_i\}_{i\in I}$ un cubrimiento abierto de $U$ y $\{f_i\}_{i\in I}$ una familia de funciones, de modo que para cada $i\in I$, $f_i:U_i\to Y$ es una función continua. Además suponemos la siguiente ``condición de pegado": para cualesquiera $i,j\in I$ se tiene $f_i(x)=f_j(x)$ para todo $x\in U_i \cap U_j$. Entonces, $f:=\bigcup_{i\in I}f_i$ es una función continua de $U$ en $Y$.
\end{theorem}
\begin{proof}
   \begin{itemize}
      \item Veamos que $f$ es en efecto una función de $U$ en $Y$. Sea $x\in U=\bigcup_{i\in I} U_i$. Existe $j\in I$ tal que $x\in U_j$, luego $\langle x,f_j(x)\rangle\in f_j \subseteq \bigcup_{i\in I} f_i =f$. Como $f_j(x)\in Y$, obtenemos que $f$ relaciona a $x$ con un elemento de $Y$. Supongamos que para $y,y'\in Y$ se tiene $\langle x,y\rangle, \langle x,y'\rangle\in f=\bigcup_{i\in I}f_i$. Existen $j,k\in I$ tales que $\langle x,y\rangle\in f_j$ y $\langle x,y'\rangle\in f_k$, es decir $x\in I_j$ y $y=f_j(x)$, y, $x\in U_k$ y $y'\in f_k(x)$; entonces $x\in U_j\cap U_k$ y por la condición de pegado se tiene $y=f_j(x)=f_k(x)=y'$, con lo cual $\langle x,y\rangle=\langle x,y'\rangle$. Lo anterior nos muestra que $f$ relaciona cada elemento de $U$ con un único elemento de $Y$, es decir, $f$ es una función de $U$ en Y.
      \item Probemos que $f:U\to Y$ es continua mostrando que devuelve abiertos de $Y$ en abiertos de $U$ por la imagen recíproca . Sea $V\stackrel{ab}\subseteq Y$. Notemos que $f^{-1}(V)=\bigcup_{i\in I} f_i^{-1}(V)$:
         \begin{itemize}
            \item[$\subseteq$:] Sea $x\in f^{-1}(V)\subseteq U$, es decir, $f(x)\in V$. Existe $j\in I$ tal que $x\in U_j$, luego $f_j(x)=f(x)\in V$, y $x\in f_j^{-1}(V)\subseteq\bigcup_{i\in I}f_i^{-1}(V)$. 
            \item[$\supseteq$:] Sea $x\in\bigcup_{i\in I} f_i^{-1}(V)$, es decir $x\in f_j^{-1}(V)$ para algún $j\in I$. Entonces $f(x)=f_j(x)\in V$ y $x\in f^{-1}(V)$.
         \end{itemize}
         Ahora bien, para cada $i\in I$ tenemos $f_i^{-1}(U)\stackrel{ab}\subseteq U_i$, luego $f_i^{-1}(V)=W_i\cap U_i$ con $W_i\stackrel{ab}\subseteq U$. Como $U_i \stackrel{ab}\subseteq U$ entonces $f_i^{-1}(V) \stackrel{ab}\subseteq U$, de modo que
         $$
         f^{-1}(V)=\bigcup_{i\in I}f_i^{-1}(V) \stackrel{ab}\subseteq U. 
         $$
         Con esto, concluimos que $f=\bigcup_{i\in I}f_i:U\to  Y$ es continua.
   \end{itemize}
\end{proof}

\begin{Prop}
   Si $p:Y\to X$ es un homeomorfismo local, entonces $p$ es una función continua y abierta. Además, la colección de todos los conjuntos abiertos de $Y$ que satisfacen \textit{(i)} y \textit{(ii)} de la Definición \Ref{Def:HomeomorfismoLocal} forman una base para la topología de $Y$.
\end{Prop}
\begin{proof}
   \begin{itemize}
      \item Probamos la continuidad de $p$ puntualmente. Sean $y\in Y$ y $U\subseteqab X$ con $p(y)\in U$. Tenemos que $y\in V_y\subseteqab Y$ y $p(y)\in p(V_y)\subseteqab X$. Tomando $W=p(V_y)\cap U$ se tiene $p(y)\in W\subseteqab X$. Además $W\subseteq p(V_y)$ y $W\subseteq U$. Como $p|^{Y}_{V_y}:V_y\to p(V_y)$ es un homeomorfismo, entonces $p^{-1}(W)=(p|^{Y}_{V_y})^{-1}(W)\subseteqab V_y \subseteqab Y$, así que $y\in p^{-1}(W)\subseteqab Y$; igualmente, $p(p^{-1}(W))\subseteq W \subseteq U$, lo cual prueba que $p$ es continua en $y$. Como $y$ es arbitraria en $Y$, obtenemos que $p:Y\to X$ es continua.
      \item Sea $V\subseteqab Y$; probemos que $p(V)\subseteqab X$. Para cada $y\in V$ definimos $W_y=V_y\cap V\subseteqab V_y$. Como $p|^{V}_{V_y}:V_y\to p(V_y)$ es un homeomorfismo, en particular es una función abierta, luego $p(W_y)=(p|^{V}_{V_y})(W_y)\subseteqab p(V_y)$; como $p(V_y)\subseteqab X$ entonces $p(W_y)\subseteqab X$ para cada $y\in V$, luego $\bigcup_{y\in V} p(W_y)\subseteqab X$. Ya que
         $$
         \begin{aligned}
            \bigcup_{y\in V}p(W_y)&=\bigcup_{y\in V}p(V\cap V_y)\\
                                  &=p\left( \bigcup_{y\in V}(V\cap V_y)\right)\\
                                  &=p\left( V\cap\bigcup_{y\in V}V_y\right)\\
                                  &=p(V),
         \end{aligned}
         $$
         pues $V\subseteq \bigcup_{y\in V}V_y$, entonces $p(V)\subseteqab X$. Obtenemos así que $p:Y\to X$ es una función abierta.
   \end{itemize}
\end{proof}

\begin{Prop}
   Sean $U\subseteqab X$ y $s$ una sección transversal, de un espacio étalé $\langle p, Y\rangle$, sobre $U$. Entonces:
   \begin{itemize}
      \item[(i)] $p|^{Y}_{s(U)}=s^{-1}$.
      \item[(ii)] $s:U\to s(U)$ es un homeomorfismo y por tanto $s$ queda completamente determinada por $s(U)$.
      \item[(iii)] $s(U)$ es un subconjunto abierto de $Y$.
   \end{itemize}
\end{Prop}
\begin{proof}
   \begin{itemize}
      \item[(i)] Dado $x\in U$ tenemos
         $$
         \begin{aligned}
            (p|^Y_{s(U)}\circ s)(x)&=p|^Y_{s(U)}(s(x))\\
                                   &=p(s(x))\\
                                   &=in_{U,X}(x)\\
                                   &=x\\
                                   &=1_U(x),
         \end{aligned}
         $$
         luego $p|^Y_{s(U)}\circ s=1_U(x)$. Dado $y\in s(U)$, existe $z\in U$ tal que $y=s(z)$, y:
         $$
         \begin{aligned}
            (s\circ p|^Y_{s(U)})(y)&=s(p|^Y_{s(U)}(y))\\
                                  &=s(p(y))\\
                                  &=s(p(s(z)))\\
                                  &=s(z)\\
                                  &=y,
         \end{aligned}
         $$
         de modo que $s\circ p|^{Y}_{s(U)}=1_{s(U)}$. Esto prueba \textit{(i)}.
      \item[(ii)] Sabemos que $s$ es una función continua, y su inversa $p|^Y_{s(U)}$ es continua por ser la restricción de una función continua; por tanto, $s:U\to s(U)$ es un homeomorfismo.
   \item[(iii)] Ahora veamos que $s(U)\subseteqab Y$. Sea $y\in s(U)$. Dado $x\in s^{-1}(V_y)$, tenemos $s(x)\in V_y$ y $x=p(s(x))\in p(V_y)$; por tanto $s^{-1}(V_y)\subseteq p(V_y)$. Como $V_y\subseteqab Y$ y $s:U\to Y$ es continua, entonces $s^{-1}(V_y)\subseteqab U$; como $U\subseteqab X$, entonces $s^{-1}(V_y)\subseteqab X$ y $s^{-1}(V_y)\subseteqab p(V_y)$. Como $p|^Y_{V_y}$ es en particular continua, se tiene $(p|^Y_{V_y})^{-1}(s^{-1}(V_y))\subseteqab V_y$; como $V_y\subseteqab Y$ entonces $(p|^Y_{V_y})^{-1}(s^{-1}(V_y))\subseteqab Y$. Además, como $y\in s(U)$, se sigue que $s(p|^Y_{V_y}(y))=s(p|^Y_{s(U)}(y))=s(s^{-1}(y))=y\in V_y$ y por tanto $y\in (p|^Y_{V_y})^{-1}(s^{-1}(V_y))$. Así, nos falta probar que $(p|^Y_{V_y})^{-1}(s^{-1}(V_y))\subseteq s(U)$; para esto basta ver que $(p|^Y_{V_y})^{-1}(s^{-1}(V_y))=V_y\cap s(U)$:
      \begin{itemize}
         \item[$\subseteq$:] Por definición sabemos que $(p|^Y_{V_y})^{-1}(s^{-1}(V_y))\subseteq V_y$. Sea $t\in (p|^Y_{V_y})^{-1}(s^{-1}(V_y))$. Tenemos que $s(p|^Y_{V_y}(t))\in V_y$ y $p|^Y_{V_y}(t)\in s^{-1}(V_y)\subseteq U$. Además,
            $$
               p|^Y_{V_y}(s(p|^Y_{V_y}(t)))= p(s(p|^Y_{V_y}(t)))=p|^Y_{V_y}(t).
            $$
            Como $p|^Y_{V_y}$ es en particular inyectiva, obtenemos $t=s(p|^Y_{V_y}(t))$; ya que $p|^Y_{V_y}(t)\in U$, se sigue $t\in s(U)$. Con lo anterior se tiene $(p|^Y_{V_y})^{-1}(s^{-1}(V_y))\subseteq s(U)$ y por lo tanto $(p|^Y_{V_y})^{-1}(s^{-1}(V_y))\subseteq V_y\cap s(U)$.
         \item[$\supseteq$:] Sea $t\in V_y\cap s(U)$. Tenemos $p|^Y_{V_y}(t)=p(t)=p|^Y_{s(U)}(t)$, luego
            $$
               s(p|^Y_{V_y}(t))=s(p|^Y_{s(U)}(t))=s(s^{-1}(t))=t\in V_y,
            $$
            de modo que $t\in (p|^Y_{V_y})^{-1}(s^{-1}(V_y))$. Así, $V_y\cap s(U)\subseteq (p|^Y_{V_y})^{-1}(s^{-1}(V_y))$. 
      \end{itemize}
   \end{itemize}
\end{proof}



\end{document}
