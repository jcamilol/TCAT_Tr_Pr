\documentclass[letterpaper]{article} %único necesario para: crear documento, hacer título, espacios, interlineado

\usepackage{vmargin} %necesario para ajustar las márgenes con \setmargins
%\usepackage{euler} %usa la fuente "euler" para las ecuaciones
\usepackage{amsfonts} %para Z estilizada
\usepackage{pifont} %para más estilos de viñetas
\usepackage{amsmath} %necesario equation*
\usepackage{amsthm} %necesario para proof enviroment
\usepackage{dsfont} %alguna letra caligráfica
\usepackage{tipa} %Epsilon bonito
\usepackage{graphicx} %para manejar imágenes
\usepackage{wrapfig} %para imágenes en modo "wrap"
\usepackage{lipsum} %para usar el texto de relleno
\usepackage{bold-extra} %permite usar textsc
\usepackage{tikz-cd} %para los diagramas conmutativos

\counterwithin{equation}{subsection}
\newtheorem{Tma}[equation]{Teorema}
\newtheorem{Prop}[equation]{Proposición}
\newtheorem{Def}[equation]{Definición}
\newtheorem{Lema}[equation]{Lema}
\newtheorem{Cor}[equation]{Corolario}
\newtheorem{Ejc}[equation]{Ejercicio}
\newtheorem{Ejm}[equation]{Ejemplo}
\newtheorem{Not}[equation]{Notación}

\newcommand{\Sh}{\normalfont{\text{Sh}}}
\newcommand{\PreSh}{\normalfont{\text{PreSh}}}
\newcommand{\PSu}{\normalfont{\text{P-Su}}}
\newcommand{\QSu}{\normalfont{\text{Q-Su}}}
\newcommand{\RSu}{\normalfont{\text{R-Su}}}
\newcommand{\Pgerm}{\normalfont{\text{P-germ}}}
\newcommand{\Qgerm}{\normalfont{\text{Q-germ}}}
\newcommand{\Rgerm}{\normalfont{\text{R-germ}}}
\newcommand{\germ}{\normalfont{\text{germ}}}
\newcommand{\Su}{\normalfont{\text{Su}}}
\newcommand{\Set}{\normalfont{\textbf{Set}}}
\newcommand{\Top}{\normalfont{\textbf{Top}}}
\newcommand{\Bund}{\normalfont{\textbf{Bund}}}
\newcommand{\Etale}{\normalfont{\textbf{Etale}}}
\newcommand{\Open}{\mathcal{O}}
\newcommand{\subseteqab}{\stackrel{ab}{\subseteq}}

\renewcommand*{\proofname}{\textbf{Prueba}} %Muestra "prueba" en lugar de "proof"
\renewcommand\refname{Referencias} %Muestra "Referencias" en lugar de "References"

%letterpaper: 21.59cm x 27.94cm
\setmargins{2.8 cm} %margen izquierdo
{1.5 cm} %margen Superior
{15.5cm} %anchura del texto
{23cm} %altura del texto
{0pt} %altura de los encabezados
{1.5cm} %espacio entre el texto y los encabezados
{5 cm} %altura del pie de página
{1 cm} %espacio entre el texto y el pie de página

\addtolength{\skip\footins}{2pc}

%\pagenumbering{gobble} %Quita la numeración de las páginas
\renewcommand{\baselinestretch}{1.5} %Aumenta el interlineado a n veces el automático
\relpenalty=9999 %Evita que se rompan las ecuaciones en el cambio de renglón
\binoppenalty=9999
\renewenvironment{abstract}
{\small
   \list{}{%
      \setlength{\leftmargin}{1.4cm}% <---------- CHANGE HERE
      \setlength{\rightmargin}{\leftmargin}%
   }%
   \item\relax}
{\endlist}

\renewcommand{\abstractname}{\textsc{Resumen}} %Muestra "Resumen" en lugar de "Abstract"

\date{}

\begin{document}

   \normalsize{
    \noindent\textit{Teoría de categorías (2024-I)}

    \vspace{-0.1cm}
    \noindent Trabajo de profundización
    
    \vspace{-0.2cm}
}
\noindent\rule{10cm}{0.1pt}\\
\vspace{0.5cm}
\begin{center}
   \textsc{
      \Large{\textbf{Dos aproximaciones equivalentes a la noción de haz}}
   }\\ \vspace{0.7cm}
   \large{\textsc{Juan Camilo Lozano Suárez \footnote{Estudiante de pregrado en matemáticas, Universidad Nacional de Colombia.\\Email: jclozanos@unal.edu.co}}}\\ \vspace{0.5cm}
\end{center}
\hspace{3.5cm} \hrulefill \hspace{3.5cm}

   \vspace{1cm}

   \begin{abstract}
      \textsc{Resumen.} Introducimos la noción de haz de dos maneras en principio independientes; primero como un funtor contravariante con buenas propiedades de pegado y luego como espacio fibrado o étalé. Posteriormente probaremos que las categorías que cada una produce son equivalentes.

        
      \vspace{0.3cm}
      \textsc{Palabras Clave.} Haz; espacio étalé; prehaz; homeomorfismo local; manojo; hacificación; equivalencia de categorías; local vs global.
   \end{abstract}

   \vspace{1cm}

   \tableofcontents

   \section{Haz como funtor}
      \subsection{Un ejemplo como motivación}
   Una constante en el quehacer matemático es el tránsito entre aspectos locales y aspectos globales. Consideremos un ejemplo enmarcado en el área de la topología. Sean $X$ un espacio topológico y $U$ un subconjunto abierto de $X$, al cual dotamos con un cubrimiento $\left\lbrace U_i\right\rbrace_{i\in I}$ de subconjuntos abiertos de $U$. Una función continua $f:U\to \mathbb{R}$ se presenta como una herramienta para entender globalmente el conjunto $U$, y fácilmente nos permite pasar al conocimiento local de $U$ en el siguiente sentido:
\begin{itemize}
   \item[\textbf{(P1)}] Si $V\stackrel{ab}\subseteq U$ entonces $f|^U_V:V\to\mathbb{R}$ (la restricción de $f$ de $U$ a $V$) es también una función continua. 
\end{itemize}
De forma recíproca, gracias al lema de pegado (Teorema \ref{Tma:lemaPegado}), un apropiado conocimiento local de $U$ nos permite pasar a un conocimiento global, en la siguiente forma:
\begin{itemize}
   \item[\textbf{(P2)}] Si $f:U\to \mathbb{R}$ es una función tal que $f|^{U}_{U_i}:U_i\to\mathbb{R}$ es continua para todo $i\in I$, entonces $f$ es continua.
\end{itemize}
Las propiedades \textbf{(P1)} y \textbf{(P2)} pueden ser capturadas en lenguaje categórico. Para esto, consideremos la categoría $\mathcal{O}(X)$ que tiene como objetos los subconjuntos abiertos de $X$, y en la cual, dados $U,V\in \mathcal{O}(X)$, hay una flecha de $V$ en $U$ si y solo si $V\subseteq U$; dicha flecha en $\mathcal{O}(X)$ (que será la única de $V$ en $U$) la representamos igualmente mediante ``$V\subseteq U$". Ahora, para cada $U\in\mathcal{O}(X)$ definimos el conjunto $CU$ de todas las funciones reales continuas sobre U:
$$
CU:=\left\lbrace f:U\to\mathbb{R}\mid f \text{ es continua}\right\rbrace,
$$
y para cualquier flecha $V\subseteq U$ en $\mathcal{O}(X)$, definimos la función de conjuntos
\begin{center}
% https://tikzcd.yichuanshen.de/#N4Igdg9gJgpgziAXAbVABwnAlgFyxMJZABgBpiBdUkANwEMAbAVxiRAGEAKANQB1e4TAEZwYOGAEcABAFUAlIhABfUuky58hFAEZyVWoxZt2M5apAZseAkQBMe6vWatF7bmbVXNRXdv1OjRQAzDwt1ay1kez9HQxcQIIAfAH13JX0YKABzeCJQIIAnCABbJF0QHAgkWxV8otLEAGZqSqQAFha6LAY2Yro0OFb0pSA
\begin{tikzcd}[row sep=-4pt, column sep=7pt]
C(V\subseteq U): & CU \arrow[r]         & CV   \\
                 & f \arrow[r, maps to] & f|^U_V
\end{tikzcd}
\end{center}

que a cada función continua de $U$ en $\mathbb{R}$ le asigna su respectiva función restricción al subconjunto $V$, que a su vez es una función continua de $V$ en $\mathbb{R}$. Tendremos entonces la siguiente propiedad:
\begin{Prop}\label{Prop:P1}
   La regla $C$ que a cada $U\in\mathcal{O}(X)$ le asigna el conjunto $CU$ y a cada flecha $V\subseteq U$ en $\mathcal{O}(X)$ le asigna la función restricción de V en U, $C(V\subseteq U): CU\to CV$, es un funtor contravariante de $\mathcal{O}(X)$ en \normalfont{\textbf{Set}}.
\end{Prop}
\begin{proof}
   \begin{itemize}
      \item Trivialmente se tiene que $C$ respeta identidades, pues para cualquier $U\in\mathcal{O}(X)$ tenemos
         \begin{center}
% https://tikzcd.yichuanshen.de/#N4Igdg9gJgpgziAXAbVABwnAlgFyxMJZABgBpiBdUkANwEMAbAVxiRAGEAKARgH0BVAJSIQAX1LpMufIRTdyVWoxZt2-MRJAZseAkQBMC6vWasRajZJ0yi87opMqRAM0tapu2ckP3jysyDOAD4CALyuooowUADm8ESgzgBOEAC2SPIgOBBI+uKJKemIAMzU2UgALGV0WAxsqXRocOWRokA
\begin{tikzcd}[row sep=-4pt, column sep=7pt]
C(1_U): & CU \arrow[r]         & CU     \\
        & f \arrow[r, maps to] & f|^U_U=f
\end{tikzcd}
\end{center}

         es decir, $C(1_U)=1_{C(U)}$.
      \item Supongamos que en $\mathcal{O}(X)$ tenemos $W\subseteq V\subseteq U$. Entonces $W\subseteq U$ y en \normalfont{\textbf{Set}} tenemos la función restricción de $U$ en $W$, $C(W\subseteq U):CU\to CW$. Tenemos ademas en \normalfont{\textbf{Set}} la composición $C(W\subseteq V)\circ C(V\subseteq U):CU\to CW$. Para cada $f\in CU$ se tiene
         $$
         \begin{aligned}
            (C(W\subseteq V)\circ C(V\subseteq U))(f)&=C(W\subseteq V)(C(V\subseteq U)(f))\\
                                                     &=C(W\subseteq V)(f|^U_V)\\
                                                     &=(f|^U_V)|^V_W\\
                                                     &=f|^U_W\\
                                                     &=C(W\subseteq U)(f),
         \end{aligned}
         $$
      con lo cual $C(W\subseteq V)\circ C(V\subseteq U) = C(W\subseteq U)$ y $C$ respeta composiciones.
      \end{itemize}
\end{proof}
Con lo anterior, podemos decir que $C$ es un \textbf{prehaz} (de conjuntos):
\begin{Def}[Prehaz]
   Un prehaz (de conjuntos) sobre un espacio topológico $X$ es un funtor contravariante de $\mathcal{O}(X)$ en \normalfont{\textbf{Set}}.
\end{Def}
La Proposición \ref{Prop:P1} permite capturar de manera categórica la propiedad \textbf{(P1)}. Para lograr hacer lo mismo con la propiedad \textbf{(P2)} introducimos el concepto de \textit{igualadores}.


   \section{Espacios étalé o espacios fibrados}
      En esta sección introducimos los espacios étalé. La palabra étalé, proveniente del francés, viene a significar ``ramificado", ``extendido", "esparcido", "repartido", etc. Se entiende entonces que los espacios étalé también se conozcan como espacios fibrados. Algunos autores entienden por haces a los espacios fibrados. Un objetivo de este escrito será comprobar que esta acepción es completamente válida.
\subsection{Manojos}
   \begin{Def}[Manojo]
    Un manojo sobre un espacio topológico $X$ es una pareja $\langle Y, p\rangle$ con $Y$ un espacio topológico y $p:Y\to X$ una función continua.
\end{Def}
Según el contexto, es frecuente denotar al manojo $\langle Y,p\rangle$ simplemente por $p$. Las inversas puntuales $p^{-1}\left\lbrace x\right\rbrace$ ($x\in X$) son llamadas las fibras de $p$ sobre $X$. La colección de todos los manojos sobre un espacio topológico tiene estructura de categoría:
\begin{Def}[Categoría $\textbf{Top}/X$]
   La categoría $\Top /X$ (léase ``categoría Top sobre X") o $\Bund (X)$ tiene por objetos todos los manojos sobre $X$. Dados $\langle Y,p\rangle,\langle Y',q\rangle\in \Top/X$, $f$ es una flecha $\langle Y,p\rangle\to\langle Y',q\rangle$ en $\Top/X$ si $f:Y\to Y'$ es una función continua y $q\circ f=p$.
   \begin{center}
    % https://tikzcd.yichuanshen.de/#N4Igdg9gJgpgziAXAbVABwnAlgFyxMJZABgBpiBdUkANwEMAbAVxiRAE0QBfU9TXfIRQBGclVqMWbdgHJuvEBmx4CRUcPH1mrRCAAa3cTCgBzeEVAAzAE4QAtkjIgcEJKIna2l+VdsPETi5IAEzUWlK6aCDUDHQARjAMAAr8KkIg1lgmABY4PiA29m7UQYihHhEgAI6GXEA
\begin{tikzcd}
Y \arrow[r, "f"] \arrow[rd, "p"'] & Y' \arrow[d, "q"] \\
                                  & X                
\end{tikzcd}
\end{center}

\end{Def}
La categoría $\Top /X$ es un ejemplo de ``categoría sobre" (ver por ejemplo \cite[p.~45]{CWM} donde se denota por $(\Top\downarrow X)$). La notación $\Bund (X)$ es debido a \textit{bundle}, la traducción al inglés de la palabra \textit{manojo}. 
\begin{Def}[Homeomorfismo local]
   Dados $X$ y $Y$ espacios topológicos, decimos que una función $p:Y\to X$ es un homeomorfismo local si para todo $y\in Y$ existe $U_y\in\Open (Y)$ con $y\in U_y$ tal que $p(U_y)\in\Open (X)$ y $p|^{Y}_{U_y}:U_y\to p(U_y)$ es un homeomorfismo. 
\end{Def}
La siguiente proposición, cuya prueba se presenta en la Sección \Ref{section:Apendice} (Apéndice) nos permite considerar a un homeomorfismo local sobre $X$ como un caso especial de manojo sobre $X$.
\begin{Prop}
   Si $f:Y\to X$ es un homeomorfismo local, entonces $f$ es una función continua y abierta; además, la familia $\left\lbrace U_y\right\rbrace_{y\in Y}$ forma una base para la topología de Y.
\end{Prop}

\subsection{Espacios étalé}
   Para introducir los espacios étalé recordamos un concepto de topología:
\begin{Def}[Homeomorfismo local]\label{Def:HomeomorfismoLocal}
   Dados $X$ y $Y$ espacios topológicos, decimos que una función $p:Y\to X$ es un homeomorfismo local sobre $X$ si para todo $y\in Y$ existe $V_y\subseteqab Y$ con $y\in V_y$ tal que:
   \begin{itemize}
      \item[(i)] $p(V_y)$ es un subconjunto abierto de $X$.
      \item[(ii)] $p|^{Y}_{V_y}:V_y\to p(V_y)$ es un homeomorfismo. 
   \end{itemize}
\end{Def}
Una propiedad importante de los homeomorfismos locales (cuya prueba se presenta en la Sección \Ref{section:Apendice} (Apéndice)) es la siguiente:
\begin{Prop}
   Si $p:Y\to X$ es un homeomorfismo local, entonces $p$ es una función continua y abierta. Además, la colección de todos los conjuntos abiertos de $Y$ que satisfacen \textit{(i)} y \textit{(ii)} de la Definición \Ref{Def:HomeomorfismoLocal} forman una base para la topología de $Y$. 
\end{Prop}
Con esto, podemos introducir los espacios fibrados o étalé como un caso particular de manojo:
\begin{Def}[Espacio étalé]
   Un espacio fibrado o étalé sobre un espacio topológico $X$ es un manojo $\langle p, Y\rangle$, donde $p: Y\to X$ es además un homeomorfismo local.
\end{Def}
Con lo anterior, podemos considerar la subcategoría plena de $\Top/X$ que tiene por objetos todos los espacios étalé sobre $X$, y que denotamos por $\Etale(X)$.

Si $p:Y\to X$ es un homeomorfismo local, las inversas puntuales $p^{-1}\left\lbrace x\right\rbrace$ ($x\in X$) son llamadas las fibras de $p$ sobre $X$. De este modo, el espacio $Y$ se presenta como la unión disyunta de las fibras de $p$, justificando así el uso de las palabras \textit{fibrado} y \textit{étalé} en las definiciones dadas. Así mismo, hablamos de $Y$ como el espacio alto o desplegado, y nos referimos a $X$ como el espacio bajo o base. El espacio $Y$ se muestra, primero, como un despliegue vertical del espacio $X$, representado en las fibras de cada elemento en el espacio base, y segundo, como un despliegue horizontal de $X$, representado mediante el pegamiento de fibras que establecen los conjuntos abiertos de $Y$. Así mismo, la función $p$ se presenta como una proyección del espacio alto en el espacio bajo. Las secciones de un espacio étalé se comportan especialmente bien; muestra de ello lo da la siguiente proposición (para su prueba ver la Sección \Ref{section:Apendice}):
\begin{Prop}
   Sean $U\subseteqab X$ y $s$ una sección transversal, de un espacio étalé $\langle p, Y\rangle$, sobre $U$. Entonces:
   \begin{itemize}
      \item $p|^{Y}_{s(U)}=s^{-1}$.
      \item $s(U)$ es un subconjunto abierto de $Y$.
      \item $s:U\to s(U)$ es un homeomorfismo.
   \end{itemize}
\end{Prop}
Lo anterior nos permite por tanto caracterizar las secciones de $\langle p, Y\rangle$ con los abiertos básicos de $Y$. Ésto a su vez nos muestra que el pegamiento horizontal de fibras que se da en $Y$ mediante conjuntos abiertos es bien portado en términos de continuidad. Un problema crucial en la teoría es el de la posibilidad de pegar secciones locales para construir secciones mayores y eventualmente globales \cite{MHPM}, lo cual captura las problemáticas de lo local versus lo global que motivaron nuestra definición de haz. Estos paralelismos manifiestos entre haces y espacios étalé permiten que la equivalencia entre las categorías $\Sh (X)$ y $\Etale (X)$, cuya prueba constituye nuestro objetivo en lo que sigue, no nos parezca en absoluto ajena.


   \section{De manojos a prehaces}
      (A lo largo de esta sección y la siguiente, $X$ denota un espacio topológico fijo).

En esta sección construimos y estudiamos un funtor $\Gamma$ de la categoría $\Top /X$ de los manojos sobre $X$, en la categoría $\PreSh (X)$ de prehaces sobre $X$.

\begin{Def}[Acción de $\Gamma_p$ sobre objetos]
   Sea $p:Y\to X$ un manojo sobre $X$. Para cada $U\in\mathcal{O}(X)$ definimos $\Gamma_p U$ como el conjunto de todas las secciones transversales de $p$ sobre $U$.
   \begin{center}
% https://tikzcd.yichuanshen.de/#N4Igdg9gJgpgziAXAbVABwnAlgFyxMJZABgBoBGAXVJADcBDAGwFcYkQBVEAX1PU1z5CKcqWLU6TVuwCaPPiAzY8BIqKo0GLNohAANef2VCiZcZqk6QAHWsBxegFtH9APpoABF24SYUAObwRKAAZgBOEI5IZCA4EEgATDQ49FiM7AAWEBAA1iA0AEYwYFDRNHAZWCE4SOS8oRFRiKKx8YhJktrsaIYg4ZFlrbU0jPRFjAAKAirCIFhg2LD5ndK6AHQbvf1NMXHDIEUlSADMxPV9jYN7zYXFpYgAtKfn21dtLYf3ACwAnD7cQA
\begin{tikzcd}
\Gamma_p U                                                                                                                         & Y \arrow[d, "p"] \\
U \arrow[r, hook, shift right] \arrow[ru, "..." description] \arrow[ru, bend left] \arrow[ru, bend right] \arrow[ru, bend left=49] & X               
\end{tikzcd}
\end{center}

\end{Def}
Notemos que si se tiene la flecha $V\subseteq U$ en $\mathcal{O}(X)$, para cada $s\in\Gamma_p U$ surge una sección transversal de $p$ sobre $V$, a saber, $s\circ \iota_{V,U}=s|^{U}_{V}$, es decir $s|^{U}_{V}\in\Gamma_p V$:
\begin{center}
% https://tikzcd.yichuanshen.de/#N4Igdg9gJgpgziAXAbVABwnAlgFyxMJZABgBoAmAXVJADcBDAGwFcYkQA1EAX1PU1z5CKAIykR1Ok1bsAqjz4gM2PASLlSxSQxZtEIAJoL+KoeorbpekAA0ekmFADm8IqABmAJwgBbJGRAcCCQAZhoceixGdgALCAgAaxAaOBisdxwkMSlddgAdPPwIgH1gDlIbbmSQRnoAIxhGAAUBVWEQTywnGMzeD28-RACgrPDI6P04xOrU9MzEbJ0ZfQKi+lLy2SqaWobm1rN9Tu7exS9ff3DgxA0QBrAoUICl6zgAHwA9YC2Nqr6Qc6DbIjG4pNIZUY5ZYgBD-QGQkFhQLjWLxJI0F75QoQErfCrbGr1RotUxqfSMGAQuEDJC3REYqzsND2bhAA
\begin{tikzcd}
                                                                                                                                        &                                                                  & Y \arrow[dd, "p"] \\
                                                                                                                                        & U \arrow[ru, "s", shift right] \arrow[rd, "{\iota_{U,X}}", hook] &                   \\
V \arrow[rr, "{\iota_{V,X}}"', hook, shift right] \arrow[ru, "{\iota_{V,U}}"', hook, shift right] \arrow[rruu, "s|^{U}_{V}", bend left] &                                                                  & X                
\end{tikzcd}
\end{center}

\begin{Def}[Acción de $\Gamma_p$ sobre flechas]
   Sea $p:Y\to X$ un manojo sobre $X$. Para cada flecha $V\subseteq U$ en $\mathcal{O}(X)$, definimos la flecha (de $\Set$) $\Gamma_p(V\subseteq U):\Gamma_p U\to \Gamma_p V:s\mapsto s|^{U}_{V}$. 
\end{Def}
Las anteriores asignaciones de flechas y objetos hacen de $\Gamma_p$ un funtor contravariante $\Gamma_p:\mathcal{O}(X)^{\text{op}}\to \Set$, es decir, un prehaz sobre $X$; más aún, $\Gamma_p$ resulta ser un haz sobre $X$.
\begin{Prop}
   Para cada manojo $p:Y\to X$, $\Gamma_p$ es un haz sobre $X$.
\end{Prop}
\begin{proof}
   \begin{itemize}
      \item La prueba de que $\Gamma_p$ es un prehaz sobre $X$ es similar a la que se da para la Proposición ($\Ref{Prop:P1}$).
      \item Sean $U\in \mathcal{O}(X)$ y $\left\lbrace U_i\right\rbrace_{i\in I}$ un cubrimiento abierto de $U$. El siguiente diagrama en $\Set$ es un igualador:
         \begin{center}
\begin{tikzcd}
\Gamma_p U \arrow[r, "e"] & \prod_{i\in I}\Gamma_p U_i \arrow[r, "\pi_2"', shift right] \arrow[r, "\pi_1", shift left] & {\prod_{\left(i,j\right)\in I\times I}\Gamma_p (U_i\cap U_j)}
\end{tikzcd}
\end{center}

         La prueba de esto es similar a la que se da para la Proposición (\Ref{Prop:P2}), sumada al siguiente hecho: dada una familia $\left\lbrace g_i\right\rbrace_{i\in I}\in \prod_{i\in I}\Gamma_p U$, se tiene $p\circ \bigcup_{i\in I} g_i=\iota_{U,X}$: dada $x\in U=\bigcup_{i\in I}U_i$, existe $j\in I$ tal que $x\in U_j$; como $g_j:U_j\to Y$ es una sección transversal de $p$ sobre $U_j$, se tiene $p\circ g_j = \iota_{U_j,X}$, y por tanto
         $$
         \begin{aligned}
            \left(p\circ \bigcup_{i\in I}g_i\right)(x)&=p\left( \bigcup_{i\in I}g_i(x)\right)\\
                                           &=p(g_j(x))\\
                                           &=\iota_{U_j,X}(x)\\
                                           &=x\\
                                           &=\iota_{U,X}(x);
         \end{aligned}
         $$
         con esto $p\circ \bigcup_{i\in I}g_i=\iota_{U,X}$. Lo anterior se hace para garantizar $\bigcup_{i\in I}g_i\in \Gamma_p U$.
   \end{itemize}
   Por tanto $\Gamma_p$ es un haz sobre $X$.
\end{proof}
A los haces del tipo $\Gamma_p$, con $p:Y\to X$ un manojo sobre $X$, los llamamos \textit{haces de secciones transversales} sobre $X$. En ocaciones denotamos $\Gamma Y$ en lugar de $\Gamma_p$. Obtenemos por cada $p\in\Top /X$ un objeto $\Gamma_p \in \Sh (X)$. Ahora deseamos obtener, por cada flecha $f:\langle Y,p \rangle \to \langle Y',p'\rangle$ en $\Top /X$, una flecha $\Gamma_f:\Gamma_p\to\Gamma_p'$ de $\Sh(X)$, es decir, una transformación natural entre los funtores $\Gamma_p,\Gamma_p':\mathcal{O}(X)^\text{op}\to\Set$; para esto necesitamos asignar, para cada $U\in\mathcal{O}(X)$, una función $\Gamma_f U$ entre los conjuntos $\Gamma_p U$ y $\Gamma_p' U$, que además nos garantice que si $V\subseteq U$ es una flecha en $\mathcal{O}(X)$, entonces el siguiente diagrama de $\Set$ conmute:
% https://tikzcd.yichuanshen.de/#N4Igdg9gJgpgziAXAbVABwnAlgFyxMJZABgBpiBdUkANwEMAbAVxiRAB12BxOgW17oB9NAAIAaiAC+pdJlz5CKMgEYqtRizace-IaICqUmSAzY8BIgCZya+s1aIO3PgMHA0AckkTpsswqtSVWo7TUdtFyF3LxFDSTUYKABzeCJQADMAJwheJGVqHAgkMnV7LWddYQAKMU44JgAjOBgcGABHWIBKIwzs3MQAZgKixGsQBjoGmAYABTlzRRBMrCSACxwQEI0HJx1XaMkausbm1o79bt8QLJzi4aQx0J2IyvTxTZA4Vax0jcQSqZgKDFK43fr5ECFJBDUphXaRQRvQzUCZTWbzAKOZZrDbxSRAA
\begin{center}
\begin{tikzcd}
\Gamma_p V \arrow[rr, "\Gamma_f V"]                                      &  & \Gamma_{p'}V                                          \\
\Gamma_p U \arrow[u, "\Gamma_p(V\subseteq U)"] \arrow[rr, "\Gamma_f U"'] &  & \Gamma_{p'} U \arrow[u, "\Gamma_{p'}(V\subseteq U)"']
\end{tikzcd}
\end{center}

Notemos que la función $\Gamma_f U$ nos exige asignar a cada sección transversal $s$ de $p$ sobre $U$ (i.e. $s:U\to Y$ es una función continua y $p\circ s=\iota_{U,X}$) una sección transversal $\Gamma_f U (s)$ de $p'$ sobre $U$ (i.e. una función continua $\Gamma_f U (s):U\to Y'$ tal que $p'\circ \Gamma_f U(s)=\iota_{U,X}$). El siguiente diagrama que se forma en $\Top /X$
% https://tikzcd.yichuanshen.de/#N4Igdg9gJgpgziAXAbVABwnAlgFyxMJZABgBoBGAXVJADcBDAGwFcYkQA1EAX1PU1z5CKchWp0mrdgA0efEBmx4CRUcXEMWbRCACac-kqFEATKXU1NUnboDkPcTCgBzeEVAAzAE4QAtkjIQHAgkMwktdgReTx9-RDDgpABmS0ltEA8DDNikUSCQxBTw6wV7GkZ6ACMYRgAFAWVhEC8sZwALHCzvPwCaRMQ8nHosRnY2iAgAaxByqpr6oxUdFvbO1IidAB1N-CGAfWAOUmluLpz4voK8q3S0B24gA
\begin{center}
\begin{tikzcd}
                                                    & Y \arrow[r, "f"] \arrow[d, "p"] & Y' \\
U \arrow[ru, "s"] \arrow[r, "{\iota_{U,X}}"', hook] & X \arrow[ru, "p'"']             &   
\end{tikzcd}
\end{center}

nos muestra que $f\circ s$ es una función continua de $U$ en $Y$, y
$$
\begin{aligned}
   p'\circ (f\circ s)&=(p'\circ f)\circ s\\
                     &=p\circ s\\
                     &=\iota_{U,X},
\end{aligned}
$$
sugiriéndonos tomar $\Gamma_f U (s)=f\circ s$. Con esto tenemos
$$
\begin{aligned}
   (\Gamma_f V\circ\Gamma_p(V\subseteq U))(s) &=\Gamma_f V(\Gamma_p (V\subseteq U)(s))\\
                                              &= \Gamma_f V(s|^{U}_{V})\\
                                              &=f\circ s|^{U}_{V}\\
                                              &=f\circ (s\circ \iota_{V,U})\\
                                              &=(f\circ s)\circ \iota_{V,U}\\
                                              &=(f\circ s)|^{U}_{V}\\
                                              &=\Gamma_{p'}(V\subseteq U)(f\circ s)\\
                                              &=\Gamma_{p'}(V\subseteq U)(\Gamma_f U(s))\\
                                              &=(\Gamma_{p'}(V\subseteq U)\circ\Gamma_f U)(s),\\
\end{aligned}
$$
Y por tanto $\Gamma_f V\circ \Gamma_p(V\subseteq U)=\Gamma_{p'}(V\subseteq U)\circ \Gamma_f U$. Lo anterior prueba que $\Gamma_f$ es una transformación natural, para cada flecha $f$ en $\Top /X$. Estas asignaciones hacen de $\Gamma$ un funtor de $\Top /X$ en $\PreSh(X)$ (de hecho, en $\Sh(X)$, pero las razones de tomar la categoría de prehaces sobre $X$ como codominio de $\Gamma$ se entenderán más adelante):
\begin{Prop}
   Las asignaciones $p\mapsto \Gamma_p$ para cada $p\in \Top /X$ y $f\mapsto \Gamma_f$ para cada flecha $f$ en $\Top /X$, determinan un funtor $\Gamma:\Top /X\to\PreSh(X)$.   
\end{Prop}
\begin{proof}
   \begin{itemize}
      \item Veamos que $\Gamma$ respeta identidades. Sea $\langle Y, p\rangle\in \Top /X$. Debemos ver $\Gamma_{1_{p}}=1_{\Gamma_{p}}$, donde $1_{\Gamma_{p}}$ es la transformación natural identidad del funtor $\Gamma_p$ en sí mismo. Dados $U\in \mathcal{O}(X)$ y $s\in \Gamma_p U$, tenemos que $\Gamma_{1_p} U$ es una función de $\Gamma_p U$ en $\Gamma_p U$ y $\Gamma_{1_p}U(s)=1_p\circ s=s$; por tanto $\Gamma_{1_p}U=1_{\Gamma_p U}=1_{\Gamma_p}U$. Como lo anterior se tiene para $U\in \mathcal{O}(X)$ arbitrario, se sigue que $\Gamma_{1_p}=1_{\Gamma_p}$.
      \item Veamos que $\Gamma$ respeta composiciones. Supongamos que tenemos en $\Top /X$ el siguiente diagrama conmutativo
         % https://tikzcd.yichuanshen.de/#N4Igdg9gJgpgziAXAbVABwnAlgFyxMJZABgBpiBdUkANwEMAbAVxiRDRAF9T1Nd9CKMgCYqtRizZoA5NK492fPASIBGUqrH1mrROzmcxMKAHN4RUADMAThAC2SMiBwQk68TrYmAOt4DGWNZ+AASWINQMdABGMAwACkoCbNZYJgAWOPJWtg6ITi5IwtTaknph3Nn2hdQFiO4luiAmXBScQA
\begin{center}
\begin{tikzcd}
p \arrow[dd, "g\circ f"'] \arrow[rd, "f"] &                    \\
                                          & p' \arrow[ld, "g"] \\
p''                                       &                   
\end{tikzcd}
\end{center}

         y veamos que en $\PreSh(X)$ el siguiente diagrama conmuta:
         % https://tikzcd.yichuanshen.de/#N4Igdg9gJgpgziAXAbVABwnAlgFyxMJZABgBpiBdUkANwEMAbAVxiRAB12BxOgW17oB9NCAC+pdJlz5CKAIyk5VWoxZtOPfkOBoA5KLESQGbHgJEyAJmX1mrRB258Bgnbv1jlMKAHN4RUAAzACcIXiQyEBwIJAUVO3UnLVdAg3Eg0PDEOOikS2pbNQcNZyEfQwywiOpcxHz4osdNF2AfTgBjLGD2gAJUkGoGOgAjGAYABSkzWRBgrB8ACxxPUSA
\begin{center}
\begin{tikzcd}
\Gamma_p \arrow[rd, "\Gamma_{f}"] \arrow[dd, "\Gamma_{g\circ f}"'] &                                    \\
                                                                   & \Gamma_{p'} \arrow[ld, "\Gamma_g"] \\
\Gamma_{p''}                                                       &                                   
\end{tikzcd}
\end{center}

         Sean $U\in\mathcal{O}(X)$ y $s\in\Gamma_p U$. Tenemos
         $$
         \begin{aligned}
            (\Gamma_{g\circ f}U)(s)&=(g\circ f)\circ s\\
                                   &=g\circ(f\circ s)\\
                                   &=(\Gamma_{g}U)(f\circ s)\\
                                   &=(\Gamma_{g}U)(\Gamma_{f}U(s))\\
                                   &=(\Gamma_{g}U\circ\Gamma_{f}U)(s).
         \end{aligned}
         $$
         Por tanto $\Gamma_{g\circ f}=\Gamma_{g}U\circ\Gamma_{f}U$; como esto vale para $U\in\mathcal{O}(X)$ arbitrario, se sigue $\Gamma_{g\circ f}=\Gamma_{g}\circ\Gamma_{f}$.
   \end{itemize}
   Lo anterior prueba que $\Gamma$ es un funtor de $\Top /X$ en $\PreSh(X)$.
\end{proof}

   \section{De prehaces a manojos}
      En esta sección construimos y estudiamos un funtor $\Lambda$ de la categoría $\PreSh(X)$ de prehaces sobre $X$ en la categoría $\Top /X$ de manojos sobre $X$.

\subsection{Suelos, gérmenes y tallos}

Iniciamos definiendo el ``suelo" de un elemento de $X$ respecto a un prehaz sobre $X$:
\begin{Def}[$P$-suelo de $x$]
   Dados $x\in X$ y $P\in \PreSh(X)$, definimos el ``P-suelo de $x$" (denotado como $\PSu (x)$) como el conjunto $\left\lbrace (U,s)\mid x\in U\in \mathcal{O}(X); s\in PU\right\rbrace$
\end{Def}
 Buscamos definir una relación de equivalencia sobre el suelo de cada elemento de $X$:
\begin{Def}
   Sean $P\in \PreSh(X)$ y $x\in X$. Definimos en $\PSu(x)$ la relación $\sim_{P,x}$ de la siguiente manera:
   \begin{center}
      Dados $(U,s),(V,t)\in \PSu(x)$, $(U,s)\sim_{P,x}(V,t)$, si y sólo si, existe $W\in\mathcal{O}(X)$ tal que $x\in W\subseteq U\cap V$ y $s|^{U}_{W}=t|^{V}_{W}$.
   \end{center}
   Si $(U,s)\sim_{P,x}(V,t)$, decimos que $(U,s)$ y $(V,t)$ tienen el mismo P-germen en $x$. 
\end{Def}
La igualdad $s|^{U}_{W}=t|^{V}_{W}$ con $W\subseteq U\cap V$ nos recuerda la idea de ``coincidir localmente" que se mostró en las motivaciones dadas en la primera sección.

Notemos que la condición $W\subseteq U\cap V$ se tiene si y sólo si $W\subseteq U$ y $W\subseteq V$, y que en este caso $s|^{U}_{W}$ y $t|^{V}_{W}$ son elementos de $P(W)$, siempre que $s\in PU$ y $t\in PV$. De esta forma, podemos representar $(U,s)\sim_{P,x}(V,t)$ con el cumplimiento simultaneo de los dos siguientes diagramas:
% https://tikzcd.yichuanshen.de/#N4Igdg9gJgpgziAXAbVABwnAlgFyxMJZABgBpiBdUkANwEMAbAVxiRAFUQBfU9TXfIRRkATFVqMWbAGrdeIDNjwEiARlKrx9Zq0QgA6nL5LBRERq2TdIAB5GF-ZUOQBmcpZ1sACpx7GBKihuYtTaUnpesn4OJoHIACwWoVbehtGKAc6JlMmeegjpjqYoAKxJEnkgcAA+AHrA7FwA+sD6XAC8OHXA0s2tXPYZTkSJIRXhIDjc4jBQAObwRKAAZgBOEAC2SOaTEEjE0Wub29Q4e4iqh+tbiG67JyAAFjB0UEhgTAwMp3RYDGyQMCsagMLBAthwCCgt65CYAHThYJAILoACMYAwvEVAiBVlg5o8pldjohEvdEAA2Yk3MrkqnyI43ACcp3OtLC1gRSOpSAAHKykBTYZzEYQeYgAOwC0nCthcsUULhAA
\begin{center}
\begin{tikzcd}
U &                         &                     & PU \arrow[rd] & s \arrow[l, "\in",blue] &                                        \\
  & W \arrow[lu] \arrow[ld] & x {\arrow[l, "\in"', blue]} &               & PW                 & s|^{U}_{W}=t|^{V}_{W} \arrow[l, "\in",blue] \\
V &                         &                     & PV \arrow[ru] & t \arrow[l, "\in",blue] &                                 
\end{tikzcd}
\end{center}

Donde el diagrama de la izquierda está en $\mathcal{O}(X)$, el de la derecha en $Set$ y las flechas azules denotan pertenencia conjuntista.

En las dos anteriores definiciones, si es claro el prehaz $P$ con el que se está trabajando, solemos denotar $\Su(x)$ y $\sim_{x}$ en lugar de \PSu(x) y $\sim_{P,x}$, respectivamente.
\begin{Prop}
   Para cualesquiera $P\in\PreSh(X)$ y $x\in X$, la relación $\sim_{x}$ es de equivalencia sobre $\Su(x)$.
\end{Prop}
\begin{proof}
   La reflexividad y simetría de $\sim_{x}$ se siguen directamente de la definición. Veamos que $\sim_{x}$ es transitiva. Sean $(T,r),(U,s),(V,t)\in\Su(x)$ tales que $(T,r)\sim_{x}(U,s)$ y $(U,s)\sim_{x}(V,t)$. Existen $W_1,W_2\in\mathcal{O}(X)$ tales que $x\in W_1\subseteq T\cap U$ y $x\in W_2\subseteq U\cap V$. Además $r|^{T}_{W_1}=s|^{U}_{W_1}$ y $s|^{U}_{W_1}=t|^{V}_{W_2}$. Tenemos $W_1\cap W_2\in\mathcal{O}(X)$ y $x\in W_1\cap W_2\subseteq T\cap V$; se forma en $\mathcal{O}(X)$ el siguiente diagrama conmutativo:
   
   % https://tikzcd.yichuanshen.de/#N4Igdg9gJgpgziAXAbVABwnAlgFyxMJZABgBpiBdUkANwEMAbAVxiRABUQBfU9TXfIRQBGUsKq1GLNgHUA+sO68QGbHgJEATKU0T6zVohDzhAHVMBjOmgAE8zUr5rBRUQGY9Uw8bkOeTgQ0UMl1qfWkjAFVHFX51IRJSABZPAzYANRjVQIS3HVSIkAAPbgkYKABzeCJQADMAJwgAWyRREBwIJGJ-EAbm1uoOpCSevpbEbXbOxGFRxvHJocQ3Of7lwemR5TGkPKmkAFZVhY3D6gAjGDAoXe7t+aRF6bIQS+ukAFo3O7qHxAA2U4TMJeNjmLCEagMOiXBgABTiLiM9SwFQAFjhSlwgA
\begin{center}
\begin{tikzcd}
T &                           &                                                                                    &                     \\
  & W_1 \arrow[lu] \arrow[ld] &                                                                                    &                     \\
U &                           & W_1\cap W_2 \arrow[lu] \arrow[ld] \arrow[lldd, bend left] \arrow[lluu, bend right] & x \arrow[l, "\in"',blue] \\
  & W_2 \arrow[lu] \arrow[ld] &                                                                                    &                     \\
V &                           &                                                                                    &                    
\end{tikzcd}
\end{center}

   Como $P$ es un funtor contravariante de $\mathcal{O}(X)$ en $\Set$, obtenemos en $\Set$ el siguiente diagrama conmutativo:
   % https://tikzcd.yichuanshen.de/#N4Igdg9gJgpgziAXAbVABwnAlgFyxMJZABgBpiBdUkANwEMAbAVxiRAAUAVEAX1PUy58hFAEZSoqrUYs27AOoB9Ub34gM2PASIAmUjqn1mrRBwAUS0QB0rAYzpoABEp0BKVQM3Ci4gMyGZEw4XD3VBLRESfQDjOQBVUI0hbRQyABYY2VN2ADVeKRgoAHN4IlAAMwAnCABbJDIQHAgkUT4K6rrENOomlraQKtqWnubEHX7Bzt8RpHG1SaRuxtHfCY6kAFYZxFX59cQt5dnqACMYMCgkAFpfYjWhxAbesdPzy52GuAALLHKcevunSWz3GFB4QA
\begin{center}
\begin{tikzcd}
PT \arrow[rd] \arrow[rrdd, bend left]  &                 &                \\
                                       & PW_1 \arrow[rd] &                \\
PU \arrow[ru] \arrow[rd] \arrow[rr]    &                 & P(W_1\cap W_2) \\
                                       & PW_2 \arrow[ru] &                \\
PV \arrow[ru] \arrow[rruu, bend right] &                 &               
\end{tikzcd}
\end{center}

   Siguiéndolo tenemos que
   $$
   \begin{aligned}
      r|^{T}_{W_1\cap W_2}&=(r|^{T}_{W_1})|^{W_1}_{W_1\cap W_2}\\
                          &=(s|^{U}_{W_1})|^{W_1}_{W_1\cap W_2}\\
                          &=s|^{U}_{W_1\cap W_2}\\
                          &=(s|^{U}_{W_2})|^{W_2}_{W_1\cap W_2}\\
                          &=(t|^{V}_{W_2})|^{W_2}_{W_1\cap W_2}\\
                          &=t|^{V}_{W_1\cap W_2}.\\
   \end{aligned}
   $$
   Por tanto, $(T,r)\sim_{x} (V,t)$ y $\sim_{x}$ es transitiva. Obtenemos que $\sim_{x}$ es una relación de equivalencia en $\Su(x)$.
\end{proof}
Ahora consideramos las clases de equivalencia generadas por la relación de tener el mismo germen en un punto:
\begin{Def}[Germen en un punto]
   Sean $P\in \PreSh(X)$ y $x\in X$. Para cada $(U,s)\in \Su(x)$, la clase de equivalencia de $(U,s)$ respecto a $\sim_{P,x}$ se denota por $\Pgerm_{x}s_{U}$, y la llamamos el germen de $(U,s)$ en $x$.
\end{Def}
Nuevamente, si por el contexto es claro con qué prehaz estamos trabajando, puede omitirse el ``$P-$" en la anterior definición.

El siguiente lema, que será utilizado más adelante, nos muestra que el germen de un elemento se conserva bajo restricciones; propiedad que en efecto concuerda con la intuición desarrollada hasta el momento.
\begin{Lema}\label{Lema:LemaAzul}
   Sean $P\in\PreSh(X)$ y $U,V\in\mathcal{O}(X)$ con $V\subseteq U$ y $s\in PU$. Si $x\in V$ entonces $\germ_{x}s_{U}=\germ_{x}(s|^{U}_{V})_{V}$.
\end{Lema}
\begin{proof}
   Supongamos que $x\in V$; en particular $x\in V\subseteq U\cap V$. Tenemos los diagramas (izquierda en $\mathcal{O}(X)$ y derecha en $\Set$):
   % https://tikzcd.yichuanshen.de/#N4Igdg9gJgpgziAXAbVABwnAlgFyxMJZABgBpiBdUkANwEMAbAVxiRAFUQBfU9TXfIRQBGUsKq1GLNgDVuvEBmx4CRMgCYJ9Zq0Qg5PPssFEAzOS1TdIAAqdDi-iqHIALGMs62NgwqUDVFHNNam1pPR9uCRgoAHN4IlAAMwAnCABbJFEQHAgkMkkvPXYAHRK4JgAjOBgcGABHAAI5agY6SpgGGycTPRSsWIALHHlktMzEbNykdVCrWTKK6tqG5tGQVIykcxy8xHdC8NsAClLyqpq6ppkASnXNiYBWamn9uaKTmUWLleu71vanW6xkCIH6QxGXAoXCAA
\begin{center}
\begin{tikzcd}
U &                                                          &  & PU \arrow[rd, "P(U\subseteq V)"]  &    \\
  & V \arrow[lu, "U\subseteq V"'] \arrow[ld, "V\subseteq V"] &  &                                   & PV \\
V &                                                          &  & PV \arrow[ru, "P(V\subseteq V)"'] &   
\end{tikzcd}
\end{center}

   Como $V\subseteq V$ es la flecha identidad de $V$, y $P$ es en particular un funtor, entonces $P(V\subseteq V)$ es la flecha identidad de $PV$; como $s\in PU$ entonces $P(U\subseteq V)(s)=s|^{U}_{V}\in PV$, luego 
   $$
   s|^{U}_{V}=P(V\subseteq V)(s|^{U}_{V})=(s|^{U}_{V})|^{V}_{V}.
   $$
   Luego $(s,U)\sim_{x}(s|^{U}_{V},V)$, de modo que las respectivas clases de equivalencia son iguales, es decir, $\germ_{x}s_{U}=\germ_{x}(s|^{U}_{V})_{V}$.
\end{proof}
Al pasar al cociente por la relación de equivalencia de ``tener el mismo germen", obtenemos el tallo (stalk en inglés) de un prehaz en un punto dado:
\begin{Def}[Tallo en un punto]
   Sean $P\in\PreSh(X)$ y $x\in X$. Al conjunto cociente
   $$
   P_{x}:=\PSu(x)/\sim_{x}=\left\lbrace \Pgerm_{x}s_{U}\mid (U,s)\in\PSu(x)\right\rbrace
   $$
   lo llamamos el tallo de $P$ en $x$.
\end{Def}
Los tallos de un prehaz no son necesariamente disyuntos; ello lo muestra el siguiente ejemplo:
\begin{Ejm}
   Tomemos $X=\left\lbrace a,b\right\rbrace$ (con $a\neq b$) dotado con la topología trivial $\tau=\left\lbrace \phi,X\right\rbrace$, $C$ el haz de funciones continuas sobre $\mathbb{R}$ y $f:X\to \mathbb{R}$ la función constante en $0$ ($f(a)=f(b)=0$). Las únicas funciones continuas de $X$ en $\mathbb{R}$ son las constantes, es decir $CX=\left\lbrace g:X\to\mathbb{R}\mid g(a)=f(a)\right\rbrace$. Notemos que
   $$
   \begin{aligned}
      \Su(a)&=\left\lbrace (U,g)\mid a\in U\subseteqab X;g\in CU\right\rbrace\\
            &=\left\lbrace (X,g)\mid g:X\to\mathbb{R} \text{ es continua}\right\rbrace\\
            &=\left\lbrace (X,g)\mid g(a)=g(b)\right\rbrace;\\
   \end{aligned}
   $$
   del mismo modo se llega a $\Su(b)=\left\lbrace (X,g)\mid g(a)=g(b)\right\rbrace$, y por tanto $\Su(a)=\Su(b)$. Sea $(X,g)\in\germ_{a}f_{X}$. Entonces $(X,g)\in\Su(a)=\Su(b)$ y $(X,g)\sim_{a}(X,f)$ y por tanto existe $W\subseteqab X$ tal que $a\in W\subseteqab X$ y $g|^{X}_{W}=f|^{X}_{W}$; pero como el único abierto no vacío de $X$ es $X$, tenemos $W=X$ y $f=f|^{X}_{X}=g|^{X}_{X}=g$. De este modo $\germ_{a}f_{X}=\left\lbrace (X,f)\right\rbrace$. Análogamente se llega a $\germ_{b}f_{X}=\left\lbrace (X,f)\right\rbrace$. Con lo anterior, $\germ_{a}f_{X}\in C_{a}\cap C_{b}$, esto es, $C_a$ y $C_b$ no son disyuntos para $a\neq b$. Por lo tanto los tallos del haz $C$ no son disyuntos.
\end{Ejm}
\subsection{El manojo $(\Lambda_P,\mathfrak{p})$}
Nos interesa forzar a los tallos de un prehaz a que tengan intersección vacía; para esto tomamos su unión disyunta (que resulta ser el coproducto de los tallos en la categoría $\Set$):
\begin{Def}
   Sea $P\in\PreSh(X)$. Denotamos por $\Lambda_P$ a la unión disyunta de los tallos de $P$ en los elementos de $X$:
   $$
   \begin{aligned}
      \Lambda_P&:=\coprod_{x\in X}P_{x}\\
               &=\bigcup_{x\in X}(P_{x}\times\left\lbrace x\right\rbrace)\\
               &=\left\lbrace (\Pgerm_{x}s_{U},x)\mid x\in X;(U,s)\in\PSu(x)\right\rbrace.
   \end{aligned}
   $$
\end{Def}
Queremos obtener por cada prehaz sobre $X$ un manojo sobre $X$; el conjunto $\Lambda_P$ es el primer paso para esto; ahora definimos una función de dicho conjunto en $X$:
\begin{Def}
   Sea $P\in\PreSh(X)$. Definimos la función $\mathfrak{p}:\Lambda_P\to X$ mediante $\mathfrak{p}(\Pgerm_{x}s_{U},x)=x$ para cada $(\Pgerm_{x}s_{U},x)\in\Lambda_P$, y la llamamos la función canónica de $\Lambda_P$ en $X$. 
\end{Def}
Para que $\mathfrak{p}$ represente un manojo sobre $X$, debemos dotar a $\Lambda_P$ de una topología; con este fin introducimos una nueva familia de funciones:
\begin{Def}
   Para cualesquiera $U\in\mathcal{O}(X)$ y $s\in PU$, definimos la función $\dot{s}:U\to\Lambda_P$ mediante $\dot{s}(x)=(\Pgerm_{x}s_{U},x)$ para cada $x\in U$.
\end{Def}
Para cada función $\dot{s}$, el siguiente diagrama en $\Set$ es conmutativo:
   % https://tikzcd.yichuanshen.de/#N4Igdg9gJgpgziAXAbVABwnAlgFyxMJZABgBoBGAXVJADcBDAGwFcYkQBVEAX1PU1z5CKchWp0mrdgA0efEBmx4CRUcXEMWbRCAA6ugDL0AtgCMo9APrAACtx7iYUAObwioAGYAnCMaRkQHAgkACYaTSkdfSgIHGA4e15PHz9EAKCkUUD6LEZ2AAsICABrEHDJbT1dfBwrYA5SaXsaRnpTGEYbAWVhEC8sZ3ycOWTfUJoMxCyIyv1jehx873pi4DREym4gA
\begin{center}
\begin{tikzcd}
                                                          & \Lambda_{P} \arrow[d, "\mathfrak{p}"] \\
U \arrow[ru, "\dot{s}"] \arrow[r, "{\iota_{U,X}}"', hook] & X                                    
\end{tikzcd}
\end{center}

Este diagrama nos recuerda las secciones transversales sobre un manojo. Deseamos que la topología que asignemos a $\Lambda_P$ haga de cada función $\dot{s}$ una sección transversal sobre $\mathfrak{p}$.
\begin{Prop}
   Sea $P\in\PreSh(X)$. El conjunto
   $$
      \mathcal{B}_{\Lambda_P}:=\left\lbrace \dot{s}(U)\mid U\in\mathcal{O}(X);s\in PU\right\rbrace
   $$
   es base para una topología sobre $\Lambda_P$.
\end{Prop}
\begin{proof}
   Hacemos uso de la caracterización dada en la Proposición \Ref{Prop:ConjuntoEsBase}.
   \begin{itemize}
      \item Veamos que $\bigcup \mathcal{B}_{\Lambda_P}=\Lambda_P$, es decir,
      $$
         \bigcup_{\substack{U\in\mathcal{O}(X) \\ s\in PU}}\dot{s}(U)=\Lambda_P.
      $$
      La contenencia $\subseteq$ es inmediata, pues $\dot{s}(U)\subseteq\Lambda_P$ para cada $U\in\mathcal{O}(X)$ y $s\in PU$. Ahora, sea $z\in\Lambda_{P}$. Existen $y\in X$ y $(V,t)\in\Su(y)$ tales que $z=(\germ_{y}t_{V},y)$. Tenemos $y\in V\in\mathcal{O}(X)$ y $t\in PV$; como $\dot{t}_{V}(y)=(\germ_{y}t_{V},y)=z$, tenemos $z\in\dot{t}(V)\subseteq\bigcup{\dot{s}(U)}$; esto nos da la contenencia $\supseteq$. Obtenemos que $\Lambda_P$ es unión de elementos de $\mathcal{B}_{\Lambda_P}$.
   \item Ahora, sean $A,B\in\mathcal{B}_{\Lambda_P}$ y probemos que $A\cap B$ es unión de elementos de $\mathcal{B}_{\Lambda_P}$. Existen $T,V\in\mathcal{O}(X), t\in PT$ y $r\in PV$ tales que $A=\dot{t}(T)$ y $B=\dot{r}(V)$. Si $A\cap B=\phi$, entonces $A\cap B$ es la unión vacía de elementos de $\mathcal{B}_{\Lambda_P}$. Supongamos que $A\cap B\neq\phi$. Dado
      $$ 
         z\in A\cap B =\dot{t}(T)\cap\dot{r}(V)=\left\lbrace (\germ_{x}t_{T},x)\mid x\in T\right\rbrace\cap\left\lbrace (\germ_{x}r_{V},x)\mid x\in V\right\rbrace,
      $$
      existe $x\in T\cap V\in \mathcal{O}(X)$ tal que $z=(germ_{x}t_{T},x)=(germ_{x}r_{V},x)$; luego $germ_{x}t_{T}=germ_{x}r_{V}$ y $(T,t)\sim_{x}(V,r)$. Así, existe $W_{z}\in\mathcal{O}(X)$ tal que $x\in W_z\subseteq T\cap V$ y $t|^{T}_{W_z}=r|^{V}_{W_z}\in PW_{z}$. Probemos que
      $$
         A\cap B=\bigcup_{z\in A\cap B}\dot{(t|^{T}_{W_z})}(W_z).
      $$
      \begin{itemize}
         \item[($\subseteq$)] Sea $\omega\in A\cap B$. Existe $x\in W_{\omega}$ tal que $\omega=(\germ_{x}t_{T},x)$. Como, por el Lema \Ref{Lema:LemaAzul} se tiene $\germ_{x}t_{T}=\germ_{x}(t|^{T}_{W_{\omega}})_{W_{\omega}}$, entonces
            $$
            \begin{aligned}
               \omega&=(\germ_{x}t_{T},x)\\
                     &=(\germ_{x}(t|^{T}_{W_{\omega}})_{W_{\omega}},x)\\
                     &=\dot{(t|^{T}_{W_{\omega}})}_{W_{\omega}}(x)
            \end{aligned}
            $$
            con $x\in W_{\omega}$; así, 
            $$
               \omega\in\dot{(t|^{T}_{W_\omega})}(W_\omega)
            $$
            con $\omega\in A\cap B$, luego
            $$
               \omega\in\bigcup_{z\in A\cap B}\dot{(t|^{T}_{W_z})}(W_z),
            $$
            y con esto,
            $$
            A\cap B\subseteq\bigcup_{z\in A\cap B}\dot{(t|^{T}_{W_z})}(W_z).
            $$
         \item[($\supseteq$)] Sea $y\in \bigcup_{z\in A\cap B}\dot{(t|^{T}_{W_z})}(W_z)$. Existe $\omega\in A\cap B$ tal que $y\in\dot{(t|^{T}_{W_\omega})}(W_\omega)$, de modo que existe $x\in W_{\omega}$ tal que $y=\dot{(t|^{T}_{W_\omega})}(x)=(\germ_{x}(t|^{T}_{W_\omega}),x)$ con $W_\omega\subseteq T\cap V$ y $t|^{T}_{W_\omega}=r|^{V}_{W_\omega}$, con lo cual $(T,t)\sim_{x}(V,r)$ y $\germ_{x}t_{T}=\germ_{x}r_{V}$. Como $\germ_{x}t_{T}=germ_{x}(t|^{T}_{W_\omega})_{W_\omega}$, tenemos
            $$
            \begin{aligned}
               y&=(\germ_{x}(t|^{T}_{W_\omega})_{W_\omega},x)\\
                &=(\germ_{x}t_{T},x)\\
                &=(\germ_{x}r_{V},x),
            \end{aligned}
            $$
            es decir $y=\dot{t}(x)=\dot{r}(x)$ con $x\in W_{\omega}\subseteq T\cap V$, de modo que $y\in\dot{t}(T)\cap\dot{r}(V)=A\cap B$. Obtenemos $\bigcup_{z\in A\cap B}\dot{(t|^{T}_{W_z})}(W_z)\subseteq A\cap B$.
      \end{itemize}
      Así, $A\cap B=\bigcup_{z\in A\cap B}\dot{(t|^{T}_{W_z})}(W_z)$, y para cada $z\in A\cap B$, $\dot{(t|^{T}_{W_z})}(W_z)\in\mathcal{B}_{\Lambda_P}$; es decir, la intersección de dos elementos de $\Lambda_P$ es unión de elementos de $\mathcal{B}_{\Lambda_P}$. Concluimos que $\mathcal{B}_{\Lambda_P}$ es base para una topología sobre $\Lambda_P$.
   \end{itemize}
\end{proof}
Teniendo en cuenta la anterior proposición, a partir de ahora consideramos, para cada prehaz $P$ sobre $X$, a $\Lambda_P$ como espacio topológico, con la topología generada por $\mathcal{B}_{\Lambda_{P}}$ (es decir, aquella que tiene por conjuntos abiertos todas las uniones arbitrarias de elementos de $\mathcal{B}_{\Lambda_P}$). Igualmente, cada $U\in\mathcal{O}(X)$ se considera con la topología de subespacio heredada de $X$.

Las siguientes proposiciones nos muestran, respectivamente, que hemos logrado obtener, con $(\Lambda_{P},\mathfrak{p})$, un manojo sobre $X$ para cada $P\in\PreSh(X)$, y que hemos cumplido nuestro propósito de que cada función del tipo $\dot{s}$ sea una sección transversal sobre $\mathfrak{p}$.
\begin{Prop}
   Sea $P\in\PreSh(X)$. La función $\mathfrak{p}:\Lambda_P\to X$ es continua. 
\end{Prop}
\begin{proof}
   Probemos que $\mathfrak{p}$ devuelve abiertos de $X$ en abiertos de $\Lambda_P$ por la imagen inversa. Sean $U\subseteqab X$ y $z\in\mathfrak{p}^{-1}(U)\subseteq \Lambda$. Por la definición de $\Lambda_P$, existen $x\in X$ y $(V,t)\in\Su(x)$ (i.e. $x\in V\subseteqab X$ y $t\in PV$), tales que $z=(\germ_{x}t_{V},x)$; así, $x=\mathfrak{p}(z)\in U$ y $x\in U\cap V$.
   \begin{itemize}
      \item Probemos $z\in\dot{(t|^{V}_{U\cap V})}(U\cap V)\subseteq\mathfrak{p}^{-1}(U)$. Como $x\in U\cap V$ y
         $$
         \begin{aligned}
            \dot{(t|^{V}_{U\cap V})}(x)&=(\germ_{x}(t|^{V}_{U\cap V})_{U\cap V},x)\\
                                       &=(\germ_{x}t_{V},x)\\
                                       &=z,
         \end{aligned}
         $$
         luego $z\in\dot{(t|^{V}_{U\cap V})}(U\cap V)$. Ahora, sea $w\in\dot{(t|^{V}_{U\cap V})}(U\cap V)$. Existe $y\in U\cap V$ tal que $w=\dot{(t|^{V}_{U\cap V})}(y)=(\germ_{y}(t|^{V}_{U\cap V})_{U\cap V},y)=(\germ_{y}t_{V},y)$, luego $\mathfrak{p}(w)=\mathfrak{p}(\germ_{y}t_{V},y)=y$, con $y\in U$, de modo que $w\in\mathfrak{p}^{-1}(U)$. Así, $\dot{(t|^{V}_{U\cap V})}(U\cap V)\subseteq \mathfrak{p}^{-1}(U)$.
   \end{itemize}
   Notemos que $\dot{(t|^{V}_{U\cap V})}(U\cap V)\in\mathcal{B}_{\Lambda_P}$, luego $\dot{(t|^{V}_{U\cap V})}(U\cap V)\subseteqab \Lambda_P$. Como $z\in\dot{(t|^{V}_{U\cap V})}(U\cap V)\subseteq \mathfrak{p}^{-1}(U)$, hemos probado que $\mathfrak{p}^{-1}(U)\subseteqab \Lambda_P$. Concluimos que $\mathfrak{p}:\Lambda_P\to X$ es una función continua. 
\end{proof}
\begin{Prop}
   Sea $P\in \PreSh(X)$. Para cada $U\in\mathcal{O}(X)$ y cada $s\in PU$ se tiene que la función $\dot{s}:U\to \Lambda_P$ es continua. Además $\mathfrak{p}\circ\dot{s}=\iota_{U,X}$.
\end{Prop}
\begin{proof}
   Sean $U\in\mathcal{O}(X)$ y $s\in PU$ cualesquiera. Veamos que $\dot{s}:U\to\Lambda_P$ devuelve abiertos de $\Lambda_P$ en abiertos de $U$ por la imagen recíproca. Sea $\omega\subseteqab \Lambda_P$. Existen $\left\lbrace V_i\right\rbrace_{i\in I}$ familia de abiertos de $X$ y $\left\lbrace t_i\right\rbrace_{i\in I}$ con $t_i\in PV_i$ para cada $i\in I$, tales que $\omega=\bigcup_{i\in I}\dot{t_i}(V_i)$. Ahora, sea $x\in\dot{s}^{-1}(\omega)$, y probemos que $x$ tiene una vecindad abierta (en $X$) contenida en $\dot{s}^{-1}(\omega)$. Tenemos $\dot{s}(x)\in\omega=\bigcup_{i\in I}\dot{(t_i)}(V_i)$, de modo que existe $j\in J$ tal que $\dot{s}(x)\in\dot{(t_{j})}(V_j)$, luego, existe $y\in V_j$ tal que $\dot{s}(x)=\dot{(t_j)}(y)$, es decir, $(\germ_{x}s_U,x)=(\germ_{y}(t_j)_{V_j},y)$; así $x=y$, y, $x\in U\cap V$ y $\germ_{x}s_{U}=\germ_{x}(t_j)_{V_j}$, luego $(U,s)\sim_{x}(V_j,t_j)$, es decir, existe $W\in\mathcal{O}(X)$ con $x\in W\subseteq U\cap V_j$ tal que $s|^{U}_{W}=t|^{V_j}_{W}$. Tomemos $a\in W$. Inmediatamente tenemos $(U,s)\sim_{a}(V_j,t_j)$, es decir, $\germ_{a}s_{U}=\germ_{a}(T_j)_{V_j}$ y $\dot{s}(a)=\dot{(t_j)}(a)$ con $a\in V_j$, luego $\dot{s}(a)\in\dot{(t_j)}(V_j)$ con $j\in I$, así que $\dot{s}(a)\in\bigcup_{i\in I}\dot{(t_i)}=\omega$, y $a\in\dot{s}^{-1}(\omega)$. Así $W\subseteq\dot{s}^{-1}(\omega)$, con $W\subseteqab X$ y $W\subseteq U$, es decir, $W\subseteqab U$, y $x\in W$. Esto prueba que $\dot{s}^{-1}(\omega)\subseteqab U$, y por tanto, que $\dot{s}:U\to\Lambda_P$ es continua. Además, dado $u\in U$ se tiene
   $$
      (\mathfrak{p}\circ\dot{s})(u)=\mathfrak{p}(\dot{s}(u))=\mathfrak{p}(\germ_{u}s_{U},u)=u=\iota_{U,X}(u).
   $$
   Por tanto $\mathfrak{p}\circ \dot{s}=\iota_{U,X}$.
\end{proof}
\begin{Def}[Manojo de gérmenes]
   Dado $P\in \PreSh(X)$, llamamos a $(\Lambda_P,\mathfrak{p})$ el manojo de gérmenes de $P$ sobre $X$.
\end{Def}
Ahora probamos que $\mathfrak{p}$, aun más que una función continua, es un homeomorfismo local sobre $X$, y que por tanto $\Lambda_P$ es un espacio étalé sobre $X$:
\begin{Lema}
   Sean $P\in\PreSh(X)$, $U\in \mathcal{O}(X)$ y $s\in PU$. La función $\dot{s}:U\to \dot{s}(U)\subseteq\Lambda_P$ es abierta, inyectiva y tiene a $\mathfrak{p}|^{\Lambda_P}_{\dot{s}(U)}$ como inversa bilátera. 
\end{Lema}
\begin{proof}
   \begin{itemize}
      \item La propiedad de ser abierta de $\dot{s}$ se sigue directamente de la definición de la topología de $\Lambda_P$.
      \item Dados $x,y\in U$, si $\dot{s}(x)=\dot{s}(y)$ entonces $(\germ_{x}s_{U},x)=(\germ_{y}s_{U},y)$ y $x=y$. Por tanto $\dot{s}$ es inyectiva.
      \item Dado $y\in \dot{s}(U)$, se tiene $y=\dot{s}(x)=((\germ_{x}s_{U}),x)$ para algún $x\in U$. Entonces $\mathfrak{p}|^{\Lambda_P}_{\dot{s}(U)}(y)=\mathfrak{p}(\germ_{x}s_{U},x)=x\in U$. Por tanto $\mathfrak{p}|^{\Lambda_P}_{\dot{s}(U)}$ es una función sobreyectiva de $\dot{s}(U)$ en $U$.
         % https://tikzcd.yichuanshen.de/#N4Igdg9gJgpgziAXAbVABwnAlgFyxMJZABgBoBGAXVJADcBDAGwFcYkQBVEAX1PU1z5CKcqWLU6TVuwA6MqBBzA43ABQcAlDwkwoAc3hFQAMwBOEALZIyIHBCSiQAIxhgoSAMw2GLNohByCkoqPHwgZpYONHbWNC5unt5SfgEyFvQ4ABZm9ADWwGjcAD4AesByADL0Fk5Q9AD6hfXl8orKaprc2txAA
\begin{center}
\begin{tikzcd}
                                   & \dot{s}(U) \arrow[ld, "\mathfrak{p}|^{\Lambda_p}_{\dot{s}(U)}", bend left] \\
U \arrow[ru, "\dot{s}", bend left] &                                                                           
\end{tikzcd}
\end{center}

         Dado $x\in U$ se tiene
         $$
         \begin{aligned}
            \mathfrak{p}|^{\Lambda_P}_{\dot{s}(U)}(\dot{s}(x))&=\mathfrak{p}|^{\Lambda_P}_{\dot{s}(U)}(\germ_{x}s_{U},x)\\
                                                              &=x\\
                                                              &=1_{U}(x),
         \end{aligned}
         $$
         así que $\mathfrak{p}|^{\Lambda_P}_{\dot{s}(U)}\circ \dot{s}=1_{U}$. Dado $z\in\dot{s}(U)$, existe $w\in U$ tal que $z=\dot{s}(w)=(\germ_{w}s_{U},w)$; entonces
         $$
         \begin{aligned}
            \dot{s}(\mathfrak{p}|^{\Lambda_P}_{\dot{s}(U)}(\germ_{w}s_{U},w))&=\dot{s}(w)\\
                                                                             &=w\\
                                                                             &=1_{\dot{s}(U)}(y),
         \end{aligned}
         $$
         luego $\dot{s}\circ\mathfrak{p}|^{\Lambda_P}_{\dot{s}(U)}=1_{\dot{s}(U)}$. Obtenemos que $\dot{s}:U\to\dot{s}(U)$ tiene a $\mathfrak{p}|^{\Lambda_P}_{\dot{s}(U)}$ como inversa bilátera.
   \end{itemize}
\end{proof}
\begin{Cor}
   Sea $P\in\PreSh(X)$. El manojo $\mathfrak{p}:\Lambda_P\to X$ es un homeomorfismo local.
\end{Cor}
\begin{proof}
   Sea $z\in\Lambda_P$. Existen $x\in X$ y $(U,s)\in\Su(x) (U\in\mathcal{O}(X),x\in U,s\in PU)$ tales que $z=(\germ_{x}s_{U},x)$.
   \begin{itemize}
      \item Tenemos $\dot{s}(U)\subseteqab \Lambda_P$, y como $x\in U$ y $\dot{s}(x)=(\germ_{x}s_{U},x)=z$; entonces $z\in\dot{s}(U)$.
      \item $\mathfrak{p}(\dot{s}(U))=\mathfrak{p}|^{\Lambda_P}_{\dot{s}(U)}(\dot{s}(U))=U\subseteqab X$.
      \item Como $\mathfrak{p}$ es continua, su restricción $\mathfrak{p}|^{\Lambda_P}_{\dot{s}(U)}$ es continua, y ésta tiene por inversa a la función continua $\dot{s}(U)$. Por tanto $\mathfrak{p}|^{\Lambda_P}_{\dot{s}(U)}:\dot{s}(U)\to U$ es un homeomorfismo.
      Obtenemos que $\mathfrak{p}:\Lambda_P\to X$ es un homeomorfismo local.
   \end{itemize}
\end{proof}

\subsection{El funtor $\Lambda$}
Como hemos resaltado, obtenemos el manojo $\Lambda_P$ sobre $X$, por cada prehaz $P$ sobre $X$. Nuestro deseo es que esta asignación en objetos nos produzca un funtor $\Lambda:\PreSh(X)\to\Top/X$; surge la pregunta de qué flecha de $\Top/X$ asignar a una flecha $h:P\dot{\to} Q$ en $\PreSh(X)$.

Sea $h:P\dot{\to}Q$ una transformación natural entre prehaces sobre $X$; queremos definir $\Lambda_h:\Lambda_{P}\to\Lambda_{Q}$, de modo que $\Lambda_h$ sea una función continua, y que $\mathfrak{q}\circ\Lambda_h=\mathfrak{p}$, es decir, que el siguiente diagrama en $\Top$ conmute:
% https://tikzcd.yichuanshen.de/#N4Igdg9gJgpgziAXAbVABwnAlgFyxMJZABgBpiBdUkANwEMAbAVxiRAB12AZOgWwCModAPrAACgF8QE0uky58hFAEZyVWoxZtOPAUNEBFKTLnY8BIquXr6zVohAANaephQA5vCKgAZgCcIXiQyEBwIJFUNO21uPkERYAALY1kQf0CI6jCkACZqWy0HTl46HET-OgBrYABHFN8AoMQQ7MQ8qMKOdhKyiuq0KWoGOn4YBjF5cyUQPyx3RJwXCSA
\begin{center}
\begin{tikzcd}
\Lambda_{P} \arrow[r, "\Lambda_{h}"] \arrow[rd, "\mathfrak{p}"'] & \Lambda_{Q} \arrow[d, "\mathfrak{q}"] \\
                                                                 & X                                    
\end{tikzcd}
\end{center}

Recordemos que 
$$
   \Lambda_P=\left\lbrace (\Pgerm_{x}s_{U},x)\mid x\in U\subseteqab X, s\in PU\right\rbrace
$$
y
$$
   \Lambda_Q=\left\lbrace (\Qgerm_{x}s_{U},x)\mid x\in U\subseteqab X, s\in QU\right\rbrace.
$$
Podría pensarse $\Lambda_h(\Pgerm_{x}s_{U},x)=(\Qgerm_{x}s_{U},x)$ (para cada $(\Pgerm_{x}s_{U},x)\in\Lambda_P$) como una asignación natural; sin embargo, el no uso de la transformación natural $h$ genera dudas. Otro camino que puede pensarse es el siguiente: dado $(\Pgerm_{x}s_{U},x)\in \Lambda_P$, como $U\in\mathcal{O}(X)$ y $h:\Lambda_P\dot{\to}\Lambda_Q$ es una transformación natural entre los funtores $\Lambda_P,\Lambda_Q:\mathcal{O}(X)^{\text{op}}\to\Set$, tenemos la función de conjuntos $h_{U}:PU\to QU$; como $s\in PU$ entonces $h_{U}s\in QU$, y $\Lambda_h(\Pgerm_{x}s_{U},x)=(\Qgerm_{x}(h_{U}s)_{U},x)$ (para cada $(\Pgerm_{x}s_{U},x)\in\Lambda_P$) nos sugiere otra definición de $\Lambda_h$. Como la anterior función trabaja con clases de equivalencia (los gérmenes sobre cada punto), debemos garantizar que está bien definida. 
\begin{itemize}
   \item Sean $x\in X; (U,s),(V,t)\in\PSu(X)$. Supongamos $\Pgerm_{x}s_{U}=\Pgerm_{x}t_{V}$ y garanticemos $\Qgerm_{x}(h_{U}s)_{U}=\Qgerm_{x}(h_{V}t)_{V}$. Tenemos $(U,s)\sim_{P,x}(V,t)$, luego existe $W\subseteqab X$ tal que $W\subseteq U$ y $W\subseteq V$, con $x\in W$, y además $s|^{U}_{W}=t|^{V}_{W}$. Tenemos en $\mathcal{O}(X)$ el diagrama
      % https://tikzcd.yichuanshen.de/#N4Igdg9gJgpgziAXAbVABwnAlgFyxMJZABgBpiBdUkANwEMAbAVxiRAFUQBfU9TXfIRQBGUsKq1GLNgHVuvEBmx4CRMgCYJ9Zq0QgAatwkwoAc3hFQAMwBOEALZJRIHBCTEe1u48TPXSdS4KLiA
\begin{center}
\begin{tikzcd}
U &                         \\
  & W \arrow[lu] \arrow[ld] \\
V &                        
\end{tikzcd}
\end{center}

      Como $h:P\dot{\to} Q$ es una transformación natural, tenemos en $\Set$ el siguiente diagrama conmutativo:
      % https://tikzcd.yichuanshen.de/#N4Igdg9gJgpgziAXAbVABwnAlgFyxMJZABgBpiBdUkANwEMAbAVxiRAAUBVEAX1PUy58hFAEZSoqrUYs27AOq9+IDNjwEiZAExT6zVog4A1JQLXCiW8rpkGQARW58zQjSgDMEm-rb3FzlUF1EWQrHWo9WUN7Ex4pGCgAc3giUAAzACcIAFskMhAcCCRPaR9DAAsAfW5qOHKsNJw8gMycpHECosQAFgjbNir-ZVbcxCtOpABWPrKQKtjhrNH8wvbqBjoAIxgGdiCLQwysRPKmmaiQAAoAHWu0crowQuzgCB4ASgAfAD1gTh5KsB5DxTCARkhxqtEB1InYbncHk8cq8Pj9gEYAUCQS0lsVqFDeqULvD7o9niivr9-oDgaDwYhphMeuc4bdSUiXm9KejMbT1lsdntzG4QEcTk04jwgA
\begin{center}
\begin{tikzcd}
PU \arrow[rr, "h_U"] \arrow[rd, "(\phantom{o})|^{U}_{W}"'] &                      & QU \arrow[rd, "(\phantom{o})|^{U}_{W}"]  &    \\
                                                           & PW \arrow[rr, "h_W"] &                                          & QW \\
PV \arrow[rr, "h_V"] \arrow[ru, "(\phantom{o})|^{V}_{W}"]  &                      & QV \arrow[ru, "(\phantom{o})|^{V}_{W}"'] &   
\end{tikzcd}
\end{center}

      Como $s\in PU, t\in PV$ y $s|^{U}_{W}=t|^{V}_{W}$, siguiendo en anterior diagrama vemos que:
      $$
         (h_{U}s)|^{U}_{W}=h_{W}(s|^{U}_{W})=h_{W}(t|^{V}_{W})=(h_{V}t)|^{V}_{W},
      $$
      de modo que $(U,h_{U}s)\sim_{Q,x}(V,h_{V}t)$ y $\Qgerm_{x}(h_{U}s)_{U}=\Qgerm_{x}(h_{V}t)_{V}$. Con esto, $\Lambda_h$ está bien definida, pues dados $(\Pgerm_{x}s_{U},x),(\Pgerm_{x}t_{V},x)\in\Lambda_P$, si $(\Pgerm_{x}s_{U},x)=(\Pgerm_{x}t_{V},x)$ entonces
      $$
      \begin{aligned}
         \Lambda_h(\Pgerm_{x}s_{U},x)&=(\Qgerm_{x}(h_{U}s)_{U},x)\\
                                     &=(\Qgerm_{x}(H_{V}t)_{V},x)\\
                                     &=\Lambda_h(\Pgerm_{x}t_{V},x).
      \end{aligned}
      $$
\end{itemize}
Ahora veamos que $\Lambda_h$ es continua:
\begin{itemize}
   \item Probamos que $\Lambda_h$ es continua puntualmente. Sea $(\Pgerm_{x}s_{u},x)\in \Lambda_P$ (tenemos $x\in U\subseteqab(X), s\in PU$). Tenemos $\Lambda_h(\Pgerm_{x}s_{U},x)=(\Qgerm_{x}(h_{U}s),x)$. Sea $\dot{t}(V)\in\mathcal{B}_{\Lambda_Q}$ una vecindad abierta básica de $(\Qgerm_{x}(h_{U}s)_{U},x)$, es decir, $V\in\mathcal{O}(X)$, $t\in QV$ y $(\Qgerm_{x}(h_{U}s)_{U},x)\in\dot{t}_{V}(V)$, con lo cual existe $y\in V$ tal que:
      $$
      \begin{aligned}
         (\Qgerm_{x}(h_{U}s)_{U},x)&=\dot{t}_{V}(y)\\
                                   &=(\Qgerm_{y}t_{V},y).
      \end{aligned}
      $$
      De este modo, $x=y$, $x\in V$ y $\Qgerm_{x}(h_{U}s)_{U}=\Qgerm_{x}t_{V}$, es decir, $(U,h_{U}s)\sim_{Q,x}(V,t)$; así, existe $W\in\mathcal{O}(X)$ tal que $x\in W\subseteq U\cap V$ y $(h_{U}s)|^{U}_{W}=t|^{V}_{W}$. Como $h:P\dot{\to} Q$ es una transformación natural y $W\subseteq U$, tenemos en $\Set$ el siguiente diagrama conmutativo:
      % https://tikzcd.yichuanshen.de/#N4Igdg9gJgpgziAXAbVABwnAlgFyxMJZABgBpiBdUkANwEMAbAVxiRAAUBVEAX1PUy58hFAEZyVWoxZsAitz4DseAkTKjJ9Zq0QcA6r34gMy4UXEbqWmbtkGekmFADm8IqABmAJwgBbJGQgOBBIAExW0jogABQAOrFoABZ0YMG+wADyPACUAD4AesCcPAD6wHo8INQMdABGMAzsgioiIF5Yzok4hp4+-ojiQSGIAMwR2mxxCcmpfpk5BUWl5ZWKIN5+AdTBSIPWUYklCkYb-eFDSGNSE7qHBtV1DU2mqrrtnd0OPEA
\begin{center}
\begin{tikzcd}
PU \arrow[d, "(\phantom{O})|^{U}_{W}"'] \arrow[r, "h_U"] & QU \arrow[d, "(\phantom{O})|^{U}_{W}"] \\
PW \arrow[r, "h_W"']                                     & QW                                    
\end{tikzcd}
\end{center}

      Como $s\in PU$,
      $$
      \begin{aligned}
         h_{W}(s|^{U}_{W})&=(h_{U}s)|^{U}_{W}=t|^{V}_{W}.
      \end{aligned}
      $$
      Notemos que $x\in W$ y 
      $$
         \dot{(s|^{U}_{W})}_{W}(x)=(\Pgerm_{x}(s|^{U}_{W})_W,x)=(\Pgerm_{x}s_{U},x),
      $$
      y por tanto $(\Pgerm_{x}s_{U},x)\in\dot{(s|^{U}_{W})(W)}$. Además $\Lambda_h(\Pgerm_{x}s_{U},x)=(\Qgerm_{x}(h_{U}s)_{U},x)$, con lo cual $(\Qgerm_{x}(h_{U}s)_{U},x)\in \Lambda_h(\dot{(s|^{U}_{W}})(W))$. Ahora, sea $z\in \Lambda_h(\dot{(s|^{U}_{W})}(W))$. Existe $\tilde{w}\in\dot{(s|^{U}_{W})}(W)$ tal que $z=\Lambda_h(\tilde{w})$. Existe $w\in W\subseteq V$ tal que $\tilde{w}=\dot{(s|^{U}_{W})}(w)=(\Pgerm_{w}(s|^{U}_{W}),w)$. Por ende
      $$
      \begin{aligned}
         z&=\Lambda_h(\tilde{w})\\
          &=\Lambda_h(\Pgerm_{w}(s|^{U}_{W}),w)\\
          &=(\Qgerm_{w}(h_{W}(s|^{U}_{W}))_{W},w)\\
          &=(\Qgerm_{w}(t|^{V}_{W})_{W},w)\\
          &=(\Qgerm_{w}t_{V},w)\\
          &=\dot{t}_{V}(w)
      \end{aligned}
      $$
      y $w\in V$, con lo cual $z\in\dot{t}_{V}(V)$ y $\Lambda_h(\dot{(s|^{U}_{W})}(W))\subseteq \dot{t}_{V}(V)$. Como $\dot{(s|^{U}_{W})}(W)\in\mathcal{B}_{\Lambda_P}$, esto completa la prueba de que $\Lambda_h:\Lambda_P\to \Lambda_Q$ es una función continua.
   \item Sumado a lo anterior, el siguiente diagrama conmuta:
      % https://tikzcd.yichuanshen.de/#N4Igdg9gJgpgziAXAbVABwnAlgFyxMJZABgBpiBdUkANwEMAbAVxiRAB12AZOgWwCModAPrAACgF8QE0uky58hFAEZyVWoxZtOPAUNEBFKTLnY8BIquXr6zVohAANaephQA5vCKgAZgCcIXiQyEBwIJFUNO21uPkERYAALY1kQf0CI6jCkACZqWy0HTl46HET-OgBrYABHFN8AoMQQ7MQ8qMKOdhKyiuq0KWoGOn4YBjF5cyUQPyx3RJwXCSA
\begin{center}
\begin{tikzcd}
\Lambda_{P} \arrow[r, "\Lambda_{h}"] \arrow[rd, "\mathfrak{p}"'] & \Lambda_{Q} \arrow[d, "\mathfrak{q}"] \\
                                                                 & X                                    
\end{tikzcd}
\end{center}

      pues, dado $(\Pgerm_{x}s_{U},x)\in \Lambda_P$, tenemos
      $$
      \begin{aligned}
         \mathfrak{q}(\Lambda_h(\Pgerm_{x}s_{U},x))&=\mathfrak{q}(\Qgerm_{x}(h_{U}s)_{U},x)\\
                                                   &=x\\
                                                   &=\mathfrak{p}(\Pgerm_{x}s_{U},x),
      \end{aligned}
      $$
      luego $\mathfrak{p}=\mathfrak{q}\circ\Lambda_h$.
\end{itemize}
Hemos probado la siguiente proposición:
\begin{Prop}
   Para cada $h:P\to Q$ en $\PreSh(X)$, la función 
   $$
      \Lambda_h:\Lambda_P\to \Lambda_Q:(\Pgerm_{x}s_{U},x)\mapsto (\Qgerm_{x}(h_{U}s)_{U},x)
   $$
   es una flecha de $(\Lambda_P,\mathfrak{p})\to(\Lambda_Q,\mathfrak{q})$ en $\Top/X$.
\end{Prop}

Con la siguiente proposición alcanzamos el objetivo de la presente sección:
\begin{Prop}
   La función $\Lambda$ que a cada $P\in \PreSh(X)$ le asigna el manojo $\Lambda_P\in\Top/X$, y a cada $h:P\dot{\to}Q$ en $\PreSh(X)$ le asigna la flecha $\Lambda_h:\Lambda_P\to \Lambda_Q$ en $\Top/X$, es un funtor covariante de $\PreSh(X)$ en $\Top/X$.   
\end{Prop}
\begin{proof}
   \begin{itemize}
      \item Veamos que $\Lambda$ respeta identidades. Dado $P\in\PreSh(X)$, para cada $(\Pgerm_{x}s_{U},x)\in\Lambda_P$ tenemos
         $$
         \begin{aligned}
            \Lambda_{1_{P}}(\Pgerm_{x}s_{U})&=(\Pgerm_{x}({(1_{P})_{U}s})_{U},x)\\
                                            &=(\Pgerm_{x}(1_{PU}s)_{U},x)\\
                                            &=(\Pgerm_{x}s_{U},x)\\
                                            &=1_{\Lambda_P}(\Pgerm_{x}s_{U},x),
         \end{aligned}
         $$
         y por tanto $\Lambda_{1_{P}}=1_{\Lambda_P}$.
      \item Veamos que $\Lambda$ respeta composiciones. Supongamos que tenemos flechas y objetos de $\PreSh(X)$ en la disposición
         % https://tikzcd.yichuanshen.de/#N4Igdg9gJgpgziAXAbVABwnAlgFyxMJZABgBpiBdUkANwEMAbAVxiRAAUQBfU9TXfIRQBGUsKq1GLNgEVuvEBmx4CRMgCYJ9Zq0QgAStwkwoAc3hFQAMwBOEALZIyIHBCSjJOtgAt51u46IHq5I6tTa0noA1n4gtg5O1CGIYZ6RIFEAOpkARkwMDDA4AAS+1Ax0OTAM7PwqQiA2WKbeOEZcQA
\begin{center}
\begin{tikzcd}
P \arrow[rd, "h", "\bullet" '] \arrow[dd, "\bullet", "k\bullet h"'] &                   \\
                                            & Q \arrow[ld, "k", "\bullet" '] \\
R                                           &                  
\end{tikzcd}
\end{center}

         y probemos que el siguiente diagrama en $\Top/X$ conmuta:
         % https://tikzcd.yichuanshen.de/#N4Igdg9gJgpgziAXAbVABwnAlgFyxMJZABgBpiBdUkANwEMAbAVxiRAB12AZOgWwCModAPrAACgF8QE0uky58hFAEZSyqrUYs2nHgKGiAilJlzseAkTIAmDfWatEHbn0EjgAJRMaYUAObwRKAAZgBOELxIZCA4EEiqmg46LvruABYmsiBhEfHUsUjW1PbaTrquBsAA1pkh4ZGI0QWIRYmlznpuolWc-EwMDDA4AAQZINQMdPwwDGLyFkogoVh+aTjSFBJAA
\begin{center}
\begin{tikzcd}
\Lambda_{P} \arrow[rd, "\Lambda_{h}"] \arrow[dd, "\Lambda_{k\bullet h}"'] &                                       \\
                                                                          & \Lambda_{Q} \arrow[ld, "\Lambda_{k}"] \\
\Lambda_{R}                                                               &                                      
\end{tikzcd}
\end{center}

         Dado $(\Pgerm_{x}s_{U},x)\in\Lambda_P$, tenemos
         $$
         \begin{aligned}
            (\Lambda_{k}\circ \Lambda_{h})(\Pgerm_{x}s_{U},x)&=\Lambda_{k}(\Lambda_{h}(\Pgerm_{x}s_{U},x))\\
                                                             &=\Lambda_{k}(\Qgerm_{x}(h_{U}s)_{U},x)\\
                                                             &=(\Rgerm_{x}(k_{U}(h_{U}s))_{U},x)\\
                                                             &=(\Rgerm_{x}((k_{U}\circ h_{U})(s))_{U},x)\\
                                                             &=(\Rgerm_{x}((k\bullet h)_{U}(s))_{U},x)\\
                                                             &=\Lambda_{k\bullet h}(\Pgerm_{x}s_{U},x),
         \end{aligned}
         $$
         y con esto $\Lambda_{k}\circ\Lambda_{h}=\Lambda_{k\bullet h}$.
   \end{itemize}
   Concluimos que $\Lambda:\PreSh(X)\to\Top/X$ es un funtor.
\end{proof}

   \section{Diálogo entre funtores}
   \newpage
   \section{Apéndice}\label{section:Apendice}
      \begin{theorem}[Lema de pegado]\label{tma:lemaPegado}
   Sean $X$ y $Y$ espacios topológicos. Sean $U\stackrel{ab}\subseteq X$, $\{U_i\}_{i\in I}$ un cubrimiento abierto de $U$ y $\{f_i\}_{i\in I}$ una familia de funciones, de modo que para cada $i\in I$, $f_i:U_i\to Y$ es una función continua. Además suponemos la siguiente ``condición de pegado": para cualesquiera $i,j\in I$ se tiene $f_i(x)=f_j(x)$ para todo $x\in U_i \cap U_j$. Entonces, $f:=\bigcup_{i\in I}f_i$ es una función continua de $U$ en $Y$.
\end{theorem}
\begin{proof}
   \begin{itemize}
      \item Veamos que $f$ es en efecto una función de $U$ en $Y$. Sea $x\in U=\bigcup_{i\in I} U_i$. Existe $j\in I$ tal que $x\in U_j$, luego $\langle x,f_j(x)\rangle\in f_j \subseteq \bigcup_{i\in I} f_i =f$. Como $f_j(x)\in Y$, obtenemos que $f$ relaciona a $x$ con un elemento de $Y$. Supongamos que para $y,y'\in Y$ se tiene $\langle x,y\rangle, \langle x,y'\rangle\in f=\bigcup_{i\in I}f_i$. Existen $j,k\in I$ tales que $\langle x,y\rangle\in f_j$ y $\langle x,y'\rangle\in f_k$, es decir $x\in I_j$ y $y=f_j(x)$, y, $x\in U_k$ y $y'\in f_k(x)$; entonces $x\in U_j\cap U_k$ y por la condición de pegado se tiene $y=f_j(x)=f_k(x)=y'$, con lo cual $\langle x,y\rangle=\langle x,y'\rangle$. Lo anterior nos muestra que $f$ relaciona cada elemento de $U$ con un único elemento de $Y$, es decir, $f$ es una función de $U$ en Y.
      \item Probemos que $f:U\to Y$ es continua mostrando que devuelve abiertos de $Y$ en abiertos de $U$ por la imagen recíproca . Sea $V\stackrel{ab}\subseteq Y$. Notemos que $f^{-1}(V)=\bigcup_{i\in I} f_i^{-1}(V)$:
         \begin{itemize}
            \item[$\subseteq$:] Sea $x\in f^{-1}(V)\subseteq U$, es decir, $f(x)\in V$. Existe $j\in I$ tal que $x\in U_j$, luego $f_j(x)=f(x)\in V$, y $x\in f_j^{-1}(V)\subseteq\bigcup_{i\in I}f_i^{-1}(V)$. 
            \item[$\supseteq$:] Sea $x\in\bigcup_{i\in I} f_i^{-1}(V)$, es decir $x\in f_j^{-1}(V)$ para algún $j\in I$. Entonces $f(x)=f_j(x)\in V$ y $x\in f^{-1}(V)$.
         \end{itemize}
         Ahora bien, para cada $i\in I$ tenemos $f_i^{-1}(U)\stackrel{ab}\subseteq U_i$, luego $f_i^{-1}(V)=W_i\cap U_i$ con $W_i\stackrel{ab}\subseteq U$. Como $U_i \stackrel{ab}\subseteq U$ entonces $f_i^{-1}(V) \stackrel{ab}\subseteq U$, de modo que
         $$
         f^{-1}(V)=\bigcup_{i\in I}f_i^{-1}(V) \stackrel{ab}\subseteq U. 
         $$
         Con esto, concluimos que $f=\bigcup_{i\in I}f_i:U\to  Y$ es continua.
   \end{itemize}
\end{proof}

\begin{Prop}
   Si $p:Y\to X$ es un homeomorfismo local, entonces $p$ es una función continua y abierta. Además, la colección de todos los conjuntos abiertos de $Y$ que satisfacen \textit{(i)} y \textit{(ii)} de la Definición \Ref{Def:HomeomorfismoLocal} forman una base para la topología de $Y$.
\end{Prop}
\begin{proof}
   \begin{itemize}
      \item Probamos la continuidad de $p$ puntualmente. Sean $y\in Y$ y $U\subseteqab X$ con $p(y)\in U$. Tenemos que $y\in V_y\subseteqab Y$ y $p(y)\in p(V_y)\subseteqab X$. Tomando $W=p(V_y)\cap U$ se tiene $p(y)\in W\subseteqab X$. Además $W\subseteq p(V_y)$ y $W\subseteq U$. Como $p|^{Y}_{V_y}:V_y\to p(V_y)$ es un homeomorfismo, entonces $p^{-1}(W)=(p|^{Y}_{V_y})^{-1}(W)\subseteqab V_y \subseteqab Y$, así que $y\in p^{-1}(W)\subseteqab Y$; igualmente, $p(p^{-1}(W))\subseteq W \subseteq U$, lo cual prueba que $p$ es continua en $y$. Como $y$ es arbitraria en $Y$, obtenemos que $p:Y\to X$ es continua.
      \item Sea $V\subseteqab Y$; probemos que $p(V)\subseteqab X$. Para cada $y\in V$ definimos $W_y=V_y\cap V\subseteqab V_y$. Como $p|^{V}_{V_y}:V_y\to p(V_y)$ es un homeomorfismo, en particular es una función abierta, luego $p(W_y)=(p|^{V}_{V_y})(W_y)\subseteqab p(V_y)$; como $p(V_y)\subseteqab X$ entonces $p(W_y)\subseteqab X$ para cada $y\in V$, luego $\bigcup_{y\in V} p(W_y)\subseteqab X$. Ya que
         $$
         \begin{aligned}
            \bigcup_{y\in V}p(W_y)&=\bigcup_{y\in V}p(V\cap V_y)\\
                                  &=p\left( \bigcup_{y\in V}(V\cap V_y)\right)\\
                                  &=p\left( V\cap\bigcup_{y\in V}V_y\right)\\
                                  &=p(V),
         \end{aligned}
         $$
         pues $V\subseteq \bigcup_{y\in V}V_y$, entonces $p(V)\subseteqab X$. Obtenemos así que $p:Y\to X$ es una función abierta.
   \end{itemize}
\end{proof}

\begin{Prop}
   Sean $U\subseteqab X$ y $s$ una sección transversal, de un espacio étalé $\langle p, Y\rangle$, sobre $U$. Entonces:
   \begin{itemize}
      \item[(i)] $p|^{Y}_{s(U)}=s^{-1}$.
      \item[(ii)] $s:U\to s(U)$ es un homeomorfismo y por tanto $s$ queda completamente determinada por $s(U)$.
      \item[(iii)] $s(U)$ es un subconjunto abierto de $Y$.
   \end{itemize}
\end{Prop}
\begin{proof}
   \begin{itemize}
      \item[(i)] Dado $x\in U$ tenemos
         $$
         \begin{aligned}
            (p|^Y_{s(U)}\circ s)(x)&=p|^Y_{s(U)}(s(x))\\
                                   &=p(s(x))\\
                                   &=in_{U,X}(x)\\
                                   &=x\\
                                   &=1_U(x),
         \end{aligned}
         $$
         luego $p|^Y_{s(U)}\circ s=1_U(x)$. Dado $y\in s(U)$, existe $z\in U$ tal que $y=s(z)$, y:
         $$
         \begin{aligned}
            (s\circ p|^Y_{s(U)})(y)&=s(p|^Y_{s(U)}(y))\\
                                  &=s(p(y))\\
                                  &=s(p(s(z)))\\
                                  &=s(z)\\
                                  &=y,
         \end{aligned}
         $$
         de modo que $s\circ p|^{Y}_{s(U)}=1_{s(U)}$. Esto prueba \textit{(i)}.
      \item[(ii)] Sabemos que $s$ es una función continua, y su inversa $p|^Y_{s(U)}$ es continua por ser la restricción de una función continua; por tanto, $s:U\to s(U)$ es un homeomorfismo.
   \item[(iii)] Ahora veamos que $s(U)\subseteqab Y$. Sea $y\in s(U)$. Dado $x\in s^{-1}(V_y)$, tenemos $s(x)\in V_y$ y $x=p(s(x))\in p(V_y)$; por tanto $s^{-1}(V_y)\subseteq p(V_y)$. Como $V_y\subseteqab Y$ y $s:U\to Y$ es continua, entonces $s^{-1}(V_y)\subseteqab U$; como $U\subseteqab X$, entonces $s^{-1}(V_y)\subseteqab X$ y $s^{-1}(V_y)\subseteqab p(V_y)$. Como $p|^Y_{V_y}$ es en particular continua, se tiene $(p|^Y_{V_y})^{-1}(s^{-1}(V_y))\subseteqab V_y$; como $V_y\subseteqab Y$ entonces $(p|^Y_{V_y})^{-1}(s^{-1}(V_y))\subseteqab Y$. Además, como $y\in s(U)$, se sigue que $s(p|^Y_{V_y}(y))=s(p|^Y_{s(U)}(y))=s(s^{-1}(y))=y\in V_y$ y por tanto $y\in (p|^Y_{V_y})^{-1}(s^{-1}(V_y))$. Así, nos falta probar que $(p|^Y_{V_y})^{-1}(s^{-1}(V_y))\subseteq s(U)$; para esto basta ver que $(p|^Y_{V_y})^{-1}(s^{-1}(V_y))=V_y\cap s(U)$:
      \begin{itemize}
         \item[$\subseteq$:] Por definición sabemos que $(p|^Y_{V_y})^{-1}(s^{-1}(V_y))\subseteq V_y$. Sea $t\in (p|^Y_{V_y})^{-1}(s^{-1}(V_y))$. Tenemos que $s(p|^Y_{V_y}(t))\in V_y$ y $p|^Y_{V_y}(t)\in s^{-1}(V_y)\subseteq U$. Además,
            $$
               p|^Y_{V_y}(s(p|^Y_{V_y}(t)))= p(s(p|^Y_{V_y}(t)))=p|^Y_{V_y}(t).
            $$
            Como $p|^Y_{V_y}$ es en particular inyectiva, obtenemos $t=s(p|^Y_{V_y}(t))$; ya que $p|^Y_{V_y}(t)\in U$, se sigue $t\in s(U)$. Con lo anterior se tiene $(p|^Y_{V_y})^{-1}(s^{-1}(V_y))\subseteq s(U)$ y por lo tanto $(p|^Y_{V_y})^{-1}(s^{-1}(V_y))\subseteq V_y\cap s(U)$.
         \item[$\supseteq$:] Sea $t\in V_y\cap s(U)$. Tenemos $p|^Y_{V_y}(t)=p(t)=p|^Y_{s(U)}(t)$, luego
            $$
               s(p|^Y_{V_y}(t))=s(p|^Y_{s(U)}(t))=s(s^{-1}(t))=t\in V_y,
            $$
            de modo que $t\in (p|^Y_{V_y})^{-1}(s^{-1}(V_y))$. Así, $V_y\cap s(U)\subseteq (p|^Y_{V_y})^{-1}(s^{-1}(V_y))$. 
      \end{itemize}
   \end{itemize}
\end{proof}


   %Comentario de prueba
   \newpage
   \nocite{*}
   \bibliographystyle{apalike}
   \bibliography{refs.bib}
\end{document}
