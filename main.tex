\documentclass[letterpaper]{article} %único necesario para: crear documento, hacer título, espacios, interlineado

\usepackage{vmargin} %necesario para ajustar las márgenes con \setmargins
%\usepackage{euler} %usa la fuente "euler" para las ecuaciones
\usepackage{amsfonts} %para Z estilizada
\usepackage{pifont} %para más estilos de viñetas
\usepackage{amsmath} %necesario equation*
\usepackage{amsthm} %necesario para proof enviroment
\usepackage{dsfont} %alguna letra caligráfica
\usepackage{tipa} %Epsilon bonito
\usepackage{graphicx} %para manejar imágenes
\usepackage{wrapfig} %para imágenes en modo "wrap"
\usepackage{lipsum} %para usar el texto de relleno
\usepackage{bold-extra} %permite usar textsc

\usepackage{tikz-cd} %para los diagramas conmutativos

\newtheorem{Tma}{Teorema}
\newtheorem{Prop}{Proposición}
\numberwithin{Tma}{section}
\numberwithin{Prop}{section}
\newtheorem{Def}{Definición}
\numberwithin{Def}{section}
\newtheorem{Lema}{Lema}
\numberwithin{Lema}{section}
\newtheorem{Ejc}{Ejercicio}
\numberwithin{Ejc}{section}
\newtheorem{Ejm}{Ejemplo}
\numberwithin{Ejm}{section}

\renewcommand*{\proofname}{\textbf{Prueba}} %Muestra "prueba" en lugar de "proof"
\renewcommand\refname{Referencias}

%letterpaper: 21.59cm x 27.94cm
\setmargins{2.8 cm} %margen izquierdo
{1.5 cm} %margen Superior
{15.5cm} %anchura del texto
{23cm} %altura del texto
{0pt} %altura de los encabezados
{1.5cm} %espacio entre el texto y los encabezados
{5 cm} %altura del pie de página
{1 cm} %espacio entre el texto y el pie de página

\addtolength{\skip\footins}{2pc}

%\pagenumbering{gobble} %Quita la numeración de las páginas
\renewcommand{\baselinestretch}{1.5} %Aumenta el interlineado a n veces el automático
\relpenalty=9999 %Evita que se rompan las ecuaciones en el cambio de renglón
\binoppenalty=9999
\renewenvironment{abstract}
{\small
   \list{}{%
      \setlength{\leftmargin}{1.4cm}% <---------- CHANGE HERE
      \setlength{\rightmargin}{\leftmargin}%
   }%
   \item\relax}
{\endlist}

\renewcommand{\abstractname}{\textsc{Resumen}}

\date{}

\begin{document}

   \input{title.tex}
   \vspace{1cm}

   \begin{abstract}
      \input{abstract}
        
      \vspace{0.3cm}
      \textsc{Palabras Clave.} Haz; espacio étalé; homeomorfismo local; manojo; hacificación; equivalencia de categorías; local vs global.
   \end{abstract}

   \vspace{1cm}

   \section{Haz como funtor}
      \subsection{Un ejemplo como motivación}\label{subsection:EjemploMotivacion}
   Una constante en el quehacer matemático es el tránsito entre aspectos locales y aspectos globales. Consideremos un ejemplo enmarcado en el área de la topología. Sean $X$ un espacio topológico y $U$ un subconjunto abierto de $X$, al cual dotamos con un cubrimiento $\left\lbrace U_i\right\rbrace_{i\in I}$ de subconjuntos abiertos de $U$. Una función continua $f:U\to \mathbb{R}$ se presenta como una herramienta para entender globalmente el conjunto $U$, y fácilmente nos permite pasar al conocimiento local de $U$ en el siguiente sentido:
\begin{itemize}
   \item[\textbf{(P1)}] Si $V\stackrel{ab}\subseteq U$ entonces $f|_V:V\to\mathbb{R}$ es también una función continua. 
\end{itemize}
De forma recíproca, gracias al lema de pegado (Teorema \ref{tma:lemaPegado}), $f$ nos permite pasar de un apropiado conocimiento local de $U$ a un conocimiento global, en la siguiente forma:
\begin{itemize}
   \item[\textbf{(P2)}] Sea $\left\lbrace U_i\right\rbrace_{i\in I}$ un cubrimiento abierto de $U$. Si $f_i:f|_{U_i}:U_i\to\mathbb{R}$ es continua para todo $i\in I$, entonces $f:U\to\mathbb{R}$ es continua.
\end{itemize}
Las propiedades \textbf{(P1)} y \textbf{(P2)} pueden ser capturadas en lenguaje categórico. Para esto, consideremos la categoría $\mathcal{O}(X)$ que tiene como objetos los subconjuntos abiertos de $X$, y en la cual, dados $U,V\in \mathcal{O}(X)$, hay una flecha de $V$ en $U$ si y solo si $V\subseteq U$; dicha flecha en $\mathcal{O}(X)$ (que será la única de $V$ en $U$) la representamos igualmente mediante ``$V\subseteq U$". Ahora, para cada $U\in\mathcal{O}(X)$ definimos el conjunto $CU$ de todas las funciones reales continuas sobre U:
$$
CU:=\left\lbrace f:U\to\mathbb{R}\mid f \text{ es continua}\right\rbrace,
$$
y para cualquier flecha $V\subseteq U$ en $\mathcal{O}(X)$, definimos la función de conjuntos
\input{Diagramas/Diag1.tex}
que a cada función continua de $U$ en $\mathbb{R}$ le asigna su respectiva función restricción al subconjunto $V$, que a su vez es una función continua de $V$ en $\mathbb{R}$. Tendremos entonces la siguiente propiedad:
\begin{prop}
   La regla $C$ que a cada $U\in\mathcal{O}(X)$ le asigna el conjunto $CU$ y a cada flecha $V\subseteq U$ en $\mathcal{O}(X)$ le asigna la función restricción de V en U, $C(V\subseteq U)$, es un funtor contravariante de $\mathcal{O}(X)$ en \normalfont{\textbf{Set}}.
\end{prop}
\begin{proof}
   
\end{proof}

\subsection{Igualadores}
   \begin{Def}
   En una categoría arbitraria $\textbf{C}$, sean $f,g:A\to B$ flechas paralelas. Un igualador de $f$ y $g$ es una pareja $\langle E,e\rangle$, con $E\in\textbf{C}$ y $e:E\to A$ en $\textbf{C}$, tal que $f\circ e=g\circ e$, y que es universal con esta propiedad, en el sentido de que si hay otra pareja $\langle U,u\rangle$ con $U\in \textbf{C}$ y $u:U\to A$ en $\textbf{C}$, tal que $f\circ u=g\circ u$, entonces existe una única flecha $v:U\to E$ en $\textbf{C}$ tal que $e\circ v=u$. 
\end{Def}
El siguiente diagrama conmutativo, que denominamos como `` diagrama igualador", se resume la anterior definición:
\input{Diagramas/Diag3.tex}

Los ejemplos de igualadores que más estaremos trabajando son aquellos que aparecen en la categoría \normalfont{\textbf{Set}}:
\begin{Ejm}
   Sean $A$ y $B$ conjuntos y $f,g$ funciones de $A$ en $B$. Verifiquemos que un igualador de $f$ y $g$ está dado por $\langle E,e\rangle$, donde $E=\left\lbrace a\in A\mid f(a)=g(a)\right\rbrace$ y $e$ es la función inclusión de $E$ en $A$:
   \begin{itemize}
      \item Dado $x\in E$ se tiene $(f\circ e)(x)=f(e(x))=f(x)=g(x)=g(e(x))=(g\circ e)(x)$, es decir, $f\circ e=g\circ e$.
      \item Supongamos que existe $\langle U,u\rangle$ con $U\in \normalfont{\textbf{Set}}$ y $u:U\to A$ en $\normalfont{\textbf{Set}}$, tal que $f\circ u=g\circ u$. Podemos definir $v:U\to E$ vía $v(x)=u(x)$ para todo $x\in U$, e inmediatamente se tendrá $e\circ v=u$; igualmente, si $v'$ es una fleca de $U\to E$ en $\normalfont{\textbf{Set}}$ tal que $e\circ v'=u$ entonces para cada $x\in U$ se tiene $v'(x)=e(v'(x))=u(x)=e(v(x))=v(x)$, de modo que $v=v'$.
   \end{itemize}
\hspace{\fill}$\Diamond$
\end{Ejm}
Directamente de la definición de igualadores, podemos derivar algunas propiedades que serán útiles másadelante:
\begin{Prop}
  
\end{Prop}

\subsection{Un ejemplo como motivación (continuación)}
   Continuando con nuestro ``ejemplo como motivación" (Sección \ref{subsection:EjemploMotivacion}), resaltamos la importancia, para la validez de la propiedad \textbf{(P2)}, de la existencia de una buena ``condición de pegado", en el sentido de que las funciones $f_i$ ($i\in I$) se respetan dondequiera que se solapen: para cualesquiera $i,j\in I$ y cualquier $x\in U_i\cap U_j$, se tiene $f_i(x)=f_j(x)$; es este buen comportamiento local en subconjuntos de $U$ lo que nos permite el paso a un conocimiento global de $U$ mediante la función continua $f$ que se reconstruye al pegar los elementos de la familia $\left\lbrace f_i\right\rbrace_{i\in I}$. Notemos que $\left\lbrace f_i\right\rbrace_{i\in I}$ es un elemento del producto cartesiano $\prod_{i\in I} CU_i$. Como, para cualesquiera $i,j\in I$ se tiene $f_i\in CU_i$ y $f_j\in CU_j$, y como $U_i\cap U_j\subseteq U_i$ y $U_i\cap U_j\subseteq U_j$, obtenemos, fruto de restringir adecuadamente a intersecciones, las funciones ${f_i}|^{U_i}_{U_i\cap U_j}, {f_j}|^{U_j}_{U_i\cap U_j}\in C(U_i\cap U_j)$, con las cuales formamos las familia $\left\lbrace {f_i}|^{U_i}_{U_i\cap U_j}\right\rbrace_{(i,j)\in I\times I}$ y $\left\lbrace {f_j}|^{U_j}_{U_i\cap U_j}\right\rbrace_{(i,j)\in I\times I}$, que a su vez son elementos del producto cartesiano $\prod_{(i,j)\in I\times I}C(U_i\cap U_j)$. Estas construcciones nos sugieren la definición de las siguientes funciones:
\begin{itemize}
   \item $e:CU\to \prod_{i\in I}CU_i$ que a cada $f\in CU$ le asigna la familia $\left\lbrace f|^{U}_{U_i}\right\rbrace_{i\in I}$.
   \item $\pi_1:\prod_{i\in I}CU_i\to \prod_{(i,j)\in I\times I}C(U_i\cap U_j)$ que a cada $\left\lbrace f_i\right\rbrace_{i\in I}$ le asigna la familia $\left\lbrace {f_i}|^{U_i}_{U_i\cap U_j}\right\rbrace_{(i,j)\in I\times I}$.
   \item $\pi_2:\prod_{i\in I}CU_i\to \prod_{(i,j)\in I\times I}C(U_i\cap U_j)$ que a cada $\left\lbrace f_i\right\rbrace_{i\in I}$ le asigna la familia $\left\lbrace {f_j}|^{U_j}_{U_i\cap U_j}\right\rbrace_{(i,j)\in I\times I}$.
\end{itemize}


   \section{Anexos}
      \input{Anexos/LemaDePegado.tex}
\begin{Prop}
   Si $p:Y\to X$ es un homeomorfismo local, entonces $p$ es una función continua y abierta.
\end{Prop}
\begin{proof}
   \begin{itemize}
      \item Probamos la continuidad de $p$ puntualmente. Sean $y\in Y$ y $U\subseteqab X$ con $p(y)\in U$. Tenemos que $y\in V_y\subseteqab Y$ y $p(y)\in p(V_y)\subseteqab X$. Tomando $W=p(V_y)\cap U$ se tiene $p(y)\in W\subseteqab X$. Además $W\subseteq p(V_y)$ y $W\subseteq U$. Como $p|^{Y}_{V_y}:V_y\to p(V_y)$ es un homeomorfismo, entonces $p^{-1}(W)=(p|^{Y}_{V_y})^{-1}(W)\subseteqab V_y \subseteqab Y$, así que $y\in p^{-1}(W)\subseteqab Y$; igualmente, $p(p^{-1}(W))\subseteq W \subseteq U$, lo cual prueba que $p$ es continua en $y$. Como $y$ es arbitraria en $Y$, obtenemos que $p:Y\to X$ es continua.
      \item Sea $V\subseteqab Y$; probemos que $p(V)\subseteqab X$. Para cada $y\in V$ definimos $W_y=V_y\cap V\subseteqab V_y$. Como $p|^{V}_{V_y}:V_y\to p(V_y)$ es un homeomorfismo, en particular es una función abierta, luego $p(W_y)=(p|^{V}_{V_y})(W_y)\subseteqab p(V_y)$; como $p(V_y)\subseteqab X$ entonces $p(W_y)\subseteqab X$ para cada $y\in V$, luego $\bigcup_{y\in V} p(W_y)\subseteqab X$. Ya que
         $$
         \begin{aligned}
            \bigcup_{y\in V}p(W_y)&=\bigcup_{y\in V}p(V\cap V_y)\\
                                  &=p\left( \bigcup_{y\in V}(V\cap V_y)\right)\\
                                  &=p\left( V\cap\bigcup_{y\in V}V_y\right)\\
                                  &=p(V),
         \end{aligned}
         $$
         pues $V\subseteq \bigcup_{y\in V}V_y$, entonces $p(V)\subseteqab X$. Obtenemos así que $p:Y\to X$ es una función abierta.
   \end{itemize}
\end{proof}

\begin{Prop}
   Sean $U\subseteqab X$ y $s$ una sección transversal, de un espacio étalé $\langle p, Y\rangle$, sobre $U$. Entonces:
   \begin{itemize}
      \item $p|^{Y}_{s(U)}=s^{-1}$.
      \item $s(U)$ es un subconjunto abierto de $Y$.
      \item $s:U\to s(U)$ es un homeomorfismo.
   \end{itemize}
\end{Prop}
\begin{proof}
    
\end{proof}



\end{document}
