En esta sección construimos y estudiamos un funtor $\Lambda$ de la categoría $\PreSh(X)$ de prehaces sobre $X$ en la categoría $\Top /X$ de manojos sobre $X$.

\subsection{Suelos, gérmenes y tallos}

Iniciamos definiendo el ``suelo" de un elemento de $X$ respecto a un prehaz sobre $X$:
\begin{Def}[$P$-suelo de $x$]
   Dados $x\in X$ y $P\in \PreSh(X)$, definimos el ``P-suelo de $x$" (denotado como $\PSu (x)$) como el conjunto $\left\lbrace (U,s)\mid x\in U\in \mathcal{O}(X); s\in PU\right\rbrace$
\end{Def}
 Buscamos definir una relación de equivalencia sobre el suelo de cada elemento de $X$:
\begin{Def}
   Sean $P\in \PreSh(X)$ y $x\in X$. Definimos en $\PSu(x)$ la relación $\sim_{P,x}$ de la siguiente manera:
   \begin{center}
      Dados $(U,s),(V,t)\in \PSu(x)$, $(U,s)\sim_{P,x}(V,t)$, si y sólo si, existe $W\in\mathcal{O}(X)$ tal que $x\in W\subseteq U\cap V$ y $s|^{U}_{W}=t|^{V}_{W}$.
   \end{center}
   Si $(U,s)\sim_{P,x}(V,t)$, decimos que $(U,s)$ y $(V,t)$ tienen el mismo P-germen en $x$. 
\end{Def}
La igualdad $s|^{U}_{W}=t|^{V}_{W}$ con $W\subseteq U\cap V$ nos recuerda la idea de ``coincidir localmente" que se mostró en las motivaciones dadas en la primera sección.

Notemos que la condición $W\subseteq U\cap V$ se tiene si y sólo si $W\subseteq U$ y $W\subseteq V$, y que en este caso $s|^{U}_{W}$ y $t|^{V}_{W}$ son elementos de $P(W)$, siempre que $s\in PU$ y $t\in PV$. De esta forma, podemos representar $(U,s)\sim_{P,x}(V,t)$ con el cumplimiento simultaneo de los dos siguientes diagramas:
% https://tikzcd.yichuanshen.de/#N4Igdg9gJgpgziAXAbVABwnAlgFyxMJZABgBpiBdUkANwEMAbAVxiRAFUQBfU9TXfIRRkATFVqMWbAGrdeIDNjwEiARlKrx9Zq0QgA6nL5LBRERq2TdIAB5GF-ZUOQBmcpZ1sACpx7GBKihuYtTaUnpesn4OJoHIACwWoVbehtGKAc6JlMmeegjpjqYoAKxJEnkgcAA+AHrA7FwA+sD6XAC8OHXA0s2tXPYZTkSJIRXhIDjc4jBQAObwRKAAZgBOEAC2SOaTEEjE0Wub29Q4e4iqh+tbiG67JyAAFjB0UEhgTAwMp3RYDGyQMCsagMLBAthwCCgt65CYAHThYJAILoACMYAwvEVAiBVlg5o8pldjohEvdEAA2Yk3MrkqnyI43ACcp3OtLC1gRSOpSAAHKykBTYZzEYQeYgAOwC0nCthcsUULhAA
\begin{center}
\begin{tikzcd}
U &                         &                     & PU \arrow[rd] & s \arrow[l, "\in",blue] &                                        \\
  & W \arrow[lu] \arrow[ld] & x {\arrow[l, "\in"', blue]} &               & PW                 & s|^{U}_{W}=t|^{V}_{W} \arrow[l, "\in",blue] \\
V &                         &                     & PV \arrow[ru] & t \arrow[l, "\in",blue] &                                 
\end{tikzcd}
\end{center}

Donde el diagrama de la izquierda está en $\mathcal{O}(X)$, el de la derecha en $Set$ y las flechas azules denotan pertenencia conjuntista.

En las dos anteriores definiciones, si es claro el prehaz $P$ con el que se está trabajando, solemos denotar $\Su(x)$ y $\sim_{x}$ en lugar de \PSu(x) y $\sim_{P,x}$, respectivamente.
\begin{Prop}
   Para cualesquiera $P\in\PreSh(X)$ y $x\in X$, la relación $\sim_{x}$ es de equivalencia sobre $\Su(x)$.
\end{Prop}
\begin{proof}
   La reflexividad y simetría de $\sim_{x}$ se siguen directamente de la definición. Veamos que $\sim_{x}$ es transitiva. Sean $(T,r),(U,s),(V,t)\in\Su(x)$ tales que $(T,r)\sim_{x}(U,s)$ y $(U,s)\sim_{x}(V,t)$. Existen $W_1,W_2\in\mathcal{O}(X)$ tales que $x\in W_1\subseteq T\cap U$ y $x\in W_2\subseteq U\cap V$. Además $r|^{T}_{W_1}=s|^{U}_{W_1}$ y $s|^{U}_{W_1}=t|^{V}_{W_2}$. Tenemos $W_1\cap W_2\in\mathcal{O}(X)$ y $x\in W_1\cap W_2\subseteq T\cap V$; se forma en $\mathcal{O}(X)$ el siguiente diagrama conmutativo:
   
   % https://tikzcd.yichuanshen.de/#N4Igdg9gJgpgziAXAbVABwnAlgFyxMJZABgBpiBdUkANwEMAbAVxiRABUQBfU9TXfIRQBGUsKq1GLNgHUA+sO68QGbHgJEATKU0T6zVohDzhAHVMBjOmgAE8zUr5rBRUQGY9Uw8bkOeTgQ0UMl1qfWkjAFVHFX51IRJSABZPAzYANRjVQIS3HVSIkAAPbgkYKABzeCJQADMAJwgAWyRREBwIJGJ-EAbm1uoOpCSevpbEbXbOxGFRxvHJocQ3Of7lwemR5TGkPKmkAFZVhY3D6gAjGDAoXe7t+aRF6bIQS+ukAFo3O7qHxAA2U4TMJeNjmLCEagMOiXBgABTiLiM9SwFQAFjhSlwgA
\begin{center}
\begin{tikzcd}
T &                           &                                                                                    &                     \\
  & W_1 \arrow[lu] \arrow[ld] &                                                                                    &                     \\
U &                           & W_1\cap W_2 \arrow[lu] \arrow[ld] \arrow[lldd, bend left] \arrow[lluu, bend right] & x \arrow[l, "\in"',blue] \\
  & W_2 \arrow[lu] \arrow[ld] &                                                                                    &                     \\
V &                           &                                                                                    &                    
\end{tikzcd}
\end{center}

   Como $P$ es un funtor contravariante de $\mathcal{O}(X)$ en $\Set$, obtenemos en $\Set$ el siguiente diagrama conmutativo:
   % https://tikzcd.yichuanshen.de/#N4Igdg9gJgpgziAXAbVABwnAlgFyxMJZABgBpiBdUkANwEMAbAVxiRAAUAVEAX1PUy58hFAEZSoqrUYs27AOoB9Ub34gM2PASIAmUjqn1mrRBwAUS0QB0rAYzpoABEp0BKVQM3Ci4gMyGZEw4XD3VBLRESfQDjOQBVUI0hbRQyABYY2VN2ADVeKRgoAHN4IlAAMwAnCABbJDIQHAgkUT4K6rrENOomlraQKtqWnubEHX7Bzt8RpHG1SaRuxtHfCY6kAFYZxFX59cQt5dnqACMYMCgkAFpfYjWhxAbesdPzy52GuAALLHKcevunSWz3GFB4QA
\begin{center}
\begin{tikzcd}
PT \arrow[rd] \arrow[rrdd, bend left]  &                 &                \\
                                       & PW_1 \arrow[rd] &                \\
PU \arrow[ru] \arrow[rd] \arrow[rr]    &                 & P(W_1\cap W_2) \\
                                       & PW_2 \arrow[ru] &                \\
PV \arrow[ru] \arrow[rruu, bend right] &                 &               
\end{tikzcd}
\end{center}

   Siguiéndolo tenemos que
   $$
   \begin{aligned}
      r|^{T}_{W_1\cap W_2}&=(r|^{T}_{W_1})|^{W_1}_{W_1\cap W_2}\\
                          &=(s|^{U}_{W_1})|^{W_1}_{W_1\cap W_2}\\
                          &=s|^{U}_{W_1\cap W_2}\\
                          &=(s|^{U}_{W_2})|^{W_2}_{W_1\cap W_2}\\
                          &=(t|^{V}_{W_2})|^{W_2}_{W_1\cap W_2}\\
                          &=t|^{V}_{W_1\cap W_2}.\\
   \end{aligned}
   $$
   Por tanto, $(T,r)\sim_{x} (V,t)$ y $\sim_{x}$ es transitiva. Obtenemos que $\sim_{x}$ es una relación de equivalencia en $\Su(x)$.
\end{proof}
Ahora consideramos las clases de equivalencia generadas por la relación de tener el mismo germen en un punto:
\begin{Def}[Germen en un punto]
   Sean $P\in \PreSh(X)$ y $x\in X$. Para cada $(U,s)\in \Su(x)$, la clase de equivalencia de $(U,s)$ respecto a $\sim_{P,x}$ se denota por $\Pgerm_{x}s_{U}$, y la llamamos el germen de $(U,s)$ en $x$.
\end{Def}
Nuevamente, si por el contexto es claro con qué prehaz estamos trabajando, puede omitirse el ``$P-$" en la anterior definición.

El siguiente lema, que será utilizado más adelante, nos muestra que el germen de un elemento se conserva bajo restricciones; propiedad que en efecto concuerda con la intuición desarrollada hasta el momento.
\begin{Lema}\label{Lema:LemaAzul}
   Sean $P\in\PreSh(X)$ y $U,V\in\mathcal{O}(X)$ con $V\subseteq U$ y $s\in PU$. Si $x\in V$ entonces $\germ_{x}s_{U}=\germ_{x}(s|^{U}_{V})_{V}$.
\end{Lema}
\begin{proof}
   Supongamos que $x\in V$; en particular $x\in V\subseteq U\cap V$. Tenemos los diagramas (izquierda en $\mathcal{O}(X)$ y derecha en $\Set$):
   % https://tikzcd.yichuanshen.de/#N4Igdg9gJgpgziAXAbVABwnAlgFyxMJZABgBpiBdUkANwEMAbAVxiRAFUQBfU9TXfIRQBGUsKq1GLNgDVuvEBmx4CRMgCYJ9Zq0Qg5PPssFEAzOS1TdIAAqdDi-iqHIALGMs62NgwqUDVFHNNam1pPR9uCRgoAHN4IlAAMwAnCABbJFEQHAgkMkkvPXYAHRK4JgAjOBgcGABHAAI5agY6SpgGGycTPRSsWIALHHlktMzEbNykdVCrWTKK6tqG5tGQVIykcxy8xHdC8NsAClLyqpq6ppkASnXNiYBWamn9uaKTmUWLleu71vanW6xkCIH6QxGXAoXCAA
\begin{center}
\begin{tikzcd}
U &                                                          &  & PU \arrow[rd, "P(U\subseteq V)"]  &    \\
  & V \arrow[lu, "U\subseteq V"'] \arrow[ld, "V\subseteq V"] &  &                                   & PV \\
V &                                                          &  & PV \arrow[ru, "P(V\subseteq V)"'] &   
\end{tikzcd}
\end{center}

   Como $V\subseteq V$ es la flecha identidad de $V$, y $P$ es en particular un funtor, entonces $P(V\subseteq V)$ es la flecha identidad de $PV$; como $s\in PU$ entonces $P(U\subseteq V)(s)=s|^{U}_{V}\in PV$, luego 
   $$
   s|^{U}_{V}=P(V\subseteq V)(s|^{U}_{V})=(s|^{U}_{V})|^{V}_{V}.
   $$
   Luego $(s,U)\sim_{x}(s|^{U}_{V},V)$, de modo que las respectivas clases de equivalencia son iguales, es decir, $\germ_{x}s_{U}=\germ_{x}(s|^{U}_{V})_{V}$.
\end{proof}
Al pasar al cociente por la relación de equivalencia de ``tener el mismo germen", obtenemos el tallo (stalk en inglés) de un prehaz en un punto dado:
\begin{Def}[Tallo en un punto]
   Sean $P\in\PreSh(X)$ y $x\in X$. Al conjunto cociente
   $$
   P_{x}:=\PSu(x)/\sim_{x}=\left\lbrace \Pgerm_{x}s_{U}\mid (U,s)\in\PSu(x)\right\rbrace
   $$
   lo llamamos el tallo de $P$ en $x$.
\end{Def}
Los tallos de un prehaz no son necesariamente disyuntos; ello lo muestra el siguiente ejemplo:
\begin{Ejm}
   Tomemos $X=\left\lbrace a,b\right\rbrace$ (con $a\neq b$) dotado con la topología trivial $\tau=\left\lbrace \phi,X\right\rbrace$, $C$ el haz de funciones continuas sobre $\mathbb{R}$ y $f:X\to \mathbb{R}$ la función constante en $0$ ($f(a)=f(b)=0$). Las únicas funciones continuas de $X$ en $\mathbb{R}$ son las constantes, es decir $CX=\left\lbrace g:X\to\mathbb{R}\mid g(a)=f(a)\right\rbrace$. Notemos que
   $$
   \begin{aligned}
      \Su(a)&=\left\lbrace (U,g)\mid a\in U\subseteqab X;g\in CU\right\rbrace\\
            &=\left\lbrace (X,g)\mid g:X\to\mathbb{R} \text{ es continua}\right\rbrace\\
            &=\left\lbrace (X,g)\mid g(a)=g(b)\right\rbrace;\\
   \end{aligned}
   $$
   del mismo modo se llega a $\Su(b)=\left\lbrace (X,g)\mid g(a)=g(b)\right\rbrace$, y por tanto $\Su(a)=\Su(b)$. Sea $(X,g)\in\germ_{a}f_{X}$. Entonces $(X,g)\in\Su(a)=\Su(b)$ y $(X,g)\sim_{a}(X,f)$ y por tanto existe $W\subseteqab X$ tal que $a\in W\subseteqab X$ y $g|^{X}_{W}=f|^{X}_{W}$; pero como el único abierto no vacío de $X$ es $X$, tenemos $W=X$ y $f=f|^{X}_{X}=g|^{X}_{X}=g$. De este modo $\germ_{a}f_{X}=\left\lbrace (X,f)\right\rbrace$. Análogamente se llega a $\germ_{b}f_{X}=\left\lbrace (X,f)\right\rbrace$. Con lo anterior, $\germ_{a}f_{X}\in C_{a}\cap C_{b}$, esto es, $C_a$ y $C_b$ no son disyuntos para $a\neq b$. Por lo tanto los tallos del haz $C$ no son disyuntos.
\end{Ejm}
\subsection{El manojo $(\Lambda_P,\mathfrak{p})$}
Nos interesa forzar a los tallos de un prehaz a que tengan intersección vacía; para esto tomamos su unión disyunta (que resulta ser el coproducto de los tallos en la categoría $\Set$):
\begin{Def}
   Sea $P\in\PreSh(X)$. Denotamos por $\Lambda_P$ a la unión disyunta de los tallos de $P$ en los elementos de $X$:
   $$
   \begin{aligned}
      \Lambda_P&:=\coprod_{x\in X}P_{x}\\
               &=\bigcup_{x\in X}(P_{x}\times\left\lbrace x\right\rbrace)\\
               &=\left\lbrace (\Pgerm_{x}s_{U},x)\mid x\in X;(U,s)\in\PSu(x)\right\rbrace.
   \end{aligned}
   $$
\end{Def}
Queremos obtener por cada prehaz sobre $X$ un manojo sobre $X$; el conjunto $\Lambda_P$ es el primer paso para esto; ahora definimos una función de dicho conjunto en $X$:
\begin{Def}
   Sea $P\in\PreSh(X)$. Definimos la función $\mathfrak{p}:\Lambda_P\to X$ mediante $\mathfrak{p}(\Pgerm_{x}s_{U},x)=x$ para cada $(\Pgerm_{x}s_{U},x)\in\Lambda_P$, y la llamamos la función canónica de $\Lambda_P$ en $X$. 
\end{Def}
Para que $\mathfrak{p}$ represente un manojo sobre $X$, debemos dotar a $\Lambda_P$ de una topología; con este fin introducimos una nueva familia de funciones:
\begin{Def}
   Para cualesquiera $U\in\mathcal{O}(X)$ y $s\in PU$, definimos la función $\dot{s}:U\to\Lambda_P$ mediante $\dot{s}(x)=(\Pgerm_{x}s_{U},x)$ para cada $x\in U$.
\end{Def}
Para cada función $\dot{s}$, el siguiente diagrama en $\Set$ es conmutativo:
   % https://tikzcd.yichuanshen.de/#N4Igdg9gJgpgziAXAbVABwnAlgFyxMJZABgBoBGAXVJADcBDAGwFcYkQBVEAX1PU1z5CKchWp0mrdgA0efEBmx4CRUcXEMWbRCAA6ugDL0AtgCMo9APrAACtx7iYUAObwioAGYAnCMaRkQHAgkACYaTSkdfSgIHGA4e15PHz9EAKCkUUD6LEZ2AAsICABrEHDJbT1dfBwrYA5SaXsaRnpTGEYbAWVhEC8sZ3ycOWTfUJoMxCyIyv1jehx873pi4DREym4gA
\begin{center}
\begin{tikzcd}
                                                          & \Lambda_{P} \arrow[d, "\mathfrak{p}"] \\
U \arrow[ru, "\dot{s}"] \arrow[r, "{\iota_{U,X}}"', hook] & X                                    
\end{tikzcd}
\end{center}

Este diagrama nos recuerda las secciones transversales sobre un manojo. Deseamos que la topología que asignemos a $\Lambda_P$ haga de cada función $\dot{s}$ una sección transversal sobre $\mathfrak{p}$.
\begin{Prop}
   Sea $P\in\PreSh(X)$. El conjunto
   $$
      \mathcal{B}_{\Lambda_P}:=\left\lbrace \dot{s}(U)\mid U\in\mathcal{O}(X);s\in PU\right\rbrace
   $$
   es base para una topología sobre $\Lambda_P$.
\end{Prop}
\begin{proof}
   Hacemos uso de la caracterización dada en la Proposición \Ref{Prop:ConjuntoEsBase}.
   \begin{itemize}
      \item Veamos que $\bigcup \mathcal{B}_{\Lambda_P}=\Lambda_P$, es decir,
      $$
         \bigcup_{\substack{U\in\mathcal{O}(X) \\ s\in PU}}\dot{s}(U)=\Lambda_P.
      $$
      La contenencia $\subseteq$ es inmediata, pues $\dot{s}(U)\subseteq\Lambda_P$ para cada $U\in\mathcal{O}(X)$ y $s\in PU$. Ahora, sea $z\in\Lambda_{P}$. Existen $y\in X$ y $(V,t)\in\Su(y)$ tales que $z=(\germ_{y}t_{V},y)$. Tenemos $y\in V\in\mathcal{O}(X)$ y $t\in PV$; como $\dot{t}_{V}(y)=(\germ_{y}t_{V},y)=z$, tenemos $z\in\dot{t}(V)\subseteq\bigcup{\dot{s}(U)}$; esto nos da la contenencia $\supseteq$. Obtenemos que $\Lambda_P$ es unión de elementos de $\mathcal{B}_{\Lambda_P}$.
   \item Ahora, sean $A,B\in\mathcal{B}_{\Lambda_P}$ y probemos que $A\cap B$ es unión de elementos de $\mathcal{B}_{\Lambda_P}$. Existen $T,V\in\mathcal{O}(X), t\in PT$ y $r\in PV$ tales que $A=\dot{t}(T)$ y $B=\dot{r}(V)$. Si $A\cap B=\phi$, entonces $A\cap B$ es la unión vacía de elementos de $\mathcal{B}_{\Lambda_P}$. Supongamos que $A\cap B\neq\phi$. Dado
      $$ 
         z\in A\cap B =\dot{t}(T)\cap\dot{r}(V)=\left\lbrace (\germ_{x}t_{T},x)\mid x\in T\right\rbrace\cap\left\lbrace (\germ_{x}r_{V},x)\mid x\in V\right\rbrace,
      $$
      existe $x\in T\cap V\in \mathcal{O}(X)$ tal que $z=(germ_{x}t_{T},x)=(germ_{x}r_{V},x)$; luego $germ_{x}t_{T}=germ_{x}r_{V}$ y $(T,t)\sim_{x}(V,r)$. Así, existe $W_{z}\in\mathcal{O}(X)$ tal que $x\in W_z\subseteq T\cap V$ y $t|^{T}_{W_z}=r|^{V}_{W_z}\in PW_{z}$. Probemos que
      $$
         A\cap B=\bigcup_{z\in A\cap B}\dot{(t|^{T}_{W_z})}(W_z).
      $$
      \begin{itemize}
         \item[($\subseteq$)] Sea $\omega\in A\cap B$. Existe $x\in W_{\omega}$ tal que $\omega=(\germ_{x}t_{T},x)$. Como, por el Lema \Ref{Lema:LemaAzul} se tiene $\germ_{x}t_{T}=\germ_{x}(t|^{T}_{W_{\omega}})_{W_{\omega}}$, entonces
            $$
            \begin{aligned}
               \omega&=(\germ_{x}t_{T},x)\\
                     &=(\germ_{x}(t|^{T}_{W_{\omega}})_{W_{\omega}},x)\\
                     &=\dot{(t|^{T}_{W_{\omega}})}_{W_{\omega}}(x)
            \end{aligned}
            $$
            con $x\in W_{\omega}$; así, 
            $$
               \omega\in\dot{(t|^{T}_{W_\omega})}(W_\omega)
            $$
            con $\omega\in A\cap B$, luego
            $$
               \omega\in\bigcup_{z\in A\cap B}\dot{(t|^{T}_{W_z})}(W_z),
            $$
            y con esto,
            $$
            A\cap B\subseteq\bigcup_{z\in A\cap B}\dot{(t|^{T}_{W_z})}(W_z).
            $$
         \item[($\supseteq$)] Sea $y\in \bigcup_{z\in A\cap B}\dot{(t|^{T}_{W_z})}(W_z)$. Existe $\omega\in A\cap B$ tal que $y\in\dot{(t|^{T}_{W_\omega})}(W_\omega)$, de modo que existe $x\in W_{\omega}$ tal que $y=\dot{(t|^{T}_{W_\omega})}(x)=(\germ_{x}(t|^{T}_{W_\omega}),x)$ con $W_\omega\subseteq T\cap V$ y $t|^{T}_{W_\omega}=r|^{V}_{W_\omega}$, con lo cual $(T,t)\sim_{x}(V,r)$ y $\germ_{x}t_{T}=\germ_{x}r_{V}$. Como $\germ_{x}t_{T}=germ_{x}(t|^{T}_{W_\omega})_{W_\omega}$, tenemos
            $$
            \begin{aligned}
               y&=(\germ_{x}(t|^{T}_{W_\omega})_{W_\omega},x)\\
                &=(\germ_{x}t_{T},x)\\
                &=(\germ_{x}r_{V},x),
            \end{aligned}
            $$
            es decir $y=\dot{t}(x)=\dot{r}(x)$ con $x\in W_{\omega}\subseteq T\cap V$, de modo que $y\in\dot{t}(T)\cap\dot{r}(V)=A\cap B$. Obtenemos $\bigcup_{z\in A\cap B}\dot{(t|^{T}_{W_z})}(W_z)\subseteq A\cap B$.
      \end{itemize}
      Así, $A\cap B=\bigcup_{z\in A\cap B}\dot{(t|^{T}_{W_z})}(W_z)$, y para cada $z\in A\cap B$, $\dot{(t|^{T}_{W_z})}(W_z)\in\mathcal{B}_{\Lambda_P}$; es decir, la intersección de dos elementos de $\Lambda_P$ es unión de elementos de $\mathcal{B}_{\Lambda_P}$. Concluimos que $\mathcal{B}_{\Lambda_P}$ es base para una topología sobre $\Lambda_P$.
   \end{itemize}
\end{proof}
Teniendo en cuenta la anterior proposición, a partir de ahora consideramos, para cada prehaz $P$ sobre $X$, a $\Lambda_P$ como espacio topológico, con la topología generada por $\mathcal{B}_{\Lambda_{P}}$ (es decir, aquella que tiene por conjuntos abiertos todas las uniones arbitrarias de elementos de $\mathcal{B}_{\Lambda_P}$). Igualmente, cada $U\in\mathcal{O}(X)$ se considera con la topología de subespacio heredada de $X$.

Las siguientes proposiciones nos muestran, respectivamente, que hemos logrado obtener, con $(\Lambda_{P},\mathfrak{p})$, un manojo sobre $X$ para cada $P\in\PreSh(X)$, y que hemos cumplido nuestro propósito de que cada función del tipo $\dot{s}$ sea una sección transversal sobre $\mathfrak{p}$.
\begin{Prop}
   Sea $P\in\PreSh(X)$. La función $\mathfrak{p}:\Lambda_P\to X$ es continua. 
\end{Prop}
\begin{proof}
   Probemos que $\mathfrak{p}$ devuelve abiertos de $X$ en abiertos de $\Lambda_P$ por la imagen inversa. Sean $U\subseteqab X$ y $z\in\mathfrak{p}^{-1}(U)\subseteq \Lambda$. Por la definición de $\Lambda_P$, existen $x\in X$ y $(V,t)\in\Su(x)$ (i.e. $x\in V\subseteqab X$ y $t\in PV$), tales que $z=(\germ_{x}t_{V},x)$; así, $x=\mathfrak{p}(z)\in U$ y $x\in U\cap V$.
   \begin{itemize}
      \item Probemos $z\in\dot{(t|^{V}_{U\cap V})}(U\cap V)\subseteq\mathfrak{p}^{-1}(U)$. Como $x\in U\cap V$ y
         $$
         \begin{aligned}
            \dot{(t|^{V}_{U\cap V})}(x)&=(\germ_{x}(t|^{V}_{U\cap V})_{U\cap V},x)\\
                                       &=(\germ_{x}t_{V},x)\\
                                       &=z,
         \end{aligned}
         $$
         luego $z\in\dot{(t|^{V}_{U\cap V})}(U\cap V)$. Ahora, sea $w\in\dot{(t|^{V}_{U\cap V})}(U\cap V)$. Existe $y\in U\cap V$ tal que $w=\dot{(t|^{V}_{U\cap V})}(y)=(\germ_{y}(t|^{V}_{U\cap V})_{U\cap V},y)=(\germ_{y}t_{V},y)$, luego $\mathfrak{p}(w)=\mathfrak{p}(\germ_{y}t_{V},y)=y$, con $y\in U$, de modo que $w\in\mathfrak{p}^{-1}(U)$. Así, $\dot{(t|^{V}_{U\cap V})}(U\cap V)\subseteq \mathfrak{p}^{-1}(U)$.
   \end{itemize}
   Notemos que $\dot{(t|^{V}_{U\cap V})}(U\cap V)\in\mathcal{B}_{\Lambda_P}$, luego $\dot{(t|^{V}_{U\cap V})}(U\cap V)\subseteqab \Lambda_P$. Como $z\in\dot{(t|^{V}_{U\cap V})}(U\cap V)\subseteq \mathfrak{p}^{-1}(U)$, hemos probado que $\mathfrak{p}^{-1}(U)\subseteqab \Lambda_P$. Concluimos que $\mathfrak{p}:\Lambda_P\to X$ es una función continua. 
\end{proof}
\begin{Prop}
   Sea $P\in \PreSh(X)$. Para cada $U\in\mathcal{O}(X)$ y cada $s\in PU$ se tiene que la función $\dot{s}:U\to \Lambda_P$ es continua. Además $\mathfrak{p}\circ\dot{s}=\iota_{U,X}$.
\end{Prop}
\begin{proof}
   Sean $U\in\mathcal{O}(X)$ y $s\in PU$ cualesquiera. Veamos que $\dot{s}:U\to\Lambda_P$ devuelve abiertos de $\Lambda_P$ en abiertos de $U$ por la imagen recíproca. Sea $\omega\subseteqab \Lambda_P$. Existen $\left\lbrace V_i\right\rbrace_{i\in I}$ familia de abiertos de $X$ y $\left\lbrace t_i\right\rbrace_{i\in I}$ con $t_i\in PV_i$ para cada $i\in I$, tales que $\omega=\bigcup_{i\in I}\dot{t_i}(V_i)$. Ahora, sea $x\in\dot{s}^{-1}(\omega)$, y probemos que $x$ tiene una vecindad abierta (en $X$) contenida en $\dot{s}^{-1}(\omega)$. Tenemos $\dot{s}(x)\in\omega=\bigcup_{i\in I}\dot{(t_i)}(V_i)$, de modo que existe $j\in J$ tal que $\dot{s}(x)\in\dot{(t_{j})}(V_j)$, luego, existe $y\in V_j$ tal que $\dot{s}(x)=\dot{(t_j)}(y)$, es decir, $(\germ_{x}s_U,x)=(\germ_{y}(t_j)_{V_j},y)$; así $x=y$, y, $x\in U\cap V$ y $\germ_{x}s_{U}=\germ_{x}(t_j)_{V_j}$, luego $(U,s)\sim_{x}(V_j,t_j)$, es decir, existe $W\in\mathcal{O}(X)$ con $x\in W\subseteq U\cap V_j$ tal que $s|^{U}_{W}=t|^{V_j}_{W}$. Tomemos $a\in W$. Inmediatamente tenemos $(U,s)\sim_{a}(V_j,t_j)$, es decir, $\germ_{a}s_{U}=\germ_{a}(T_j)_{V_j}$ y $\dot{s}(a)=\dot{(t_j)}(a)$ con $a\in V_j$, luego $\dot{s}(a)\in\dot{(t_j)}(V_j)$ con $j\in I$, así que $\dot{s}(a)\in\bigcup_{i\in I}\dot{(t_i)}=\omega$, y $a\in\dot{s}^{-1}(\omega)$. Así $W\subseteq\dot{s}^{-1}(\omega)$, con $W\subseteqab X$ y $W\subseteq U$, es decir, $W\subseteqab U$, y $x\in W$. Esto prueba que $\dot{s}^{-1}(\omega)\subseteqab U$, y por tanto, que $\dot{s}:U\to\Lambda_P$ es continua. Además, dado $u\in U$ se tiene
   $$
      (\mathfrak{p}\circ\dot{s})(u)=\mathfrak{p}(\dot{s}(u))=\mathfrak{p}(\germ_{u}s_{U},u)=u=\iota_{U,X}(u).
   $$
   Por tanto $\mathfrak{p}\circ \dot{s}=\iota_{U,X}$.
\end{proof}
\begin{Def}[Manojo de gérmenes]
   Dado $P\in \PreSh(X)$, llamamos a $(\Lambda_P,\mathfrak{p})$ el manojo de gérmenes de $P$ sobre $X$.
\end{Def}
Ahora probamos que $\mathfrak{p}$, aun más que una función continua, es un homeomorfismo local sobre $X$, y que por tanto $\Lambda_P$ es un espacio étalé sobre $X$:
\begin{Lema}
   Sean $P\in\PreSh(X)$, $U\in \mathcal{O}(X)$ y $s\in PU$. La función $\dot{s}:U\to \dot{s}(U)\subseteq\Lambda_P$ es abierta, inyectiva y tiene a $\mathfrak{p}|^{\Lambda_P}_{\dot{s}(U)}$ como inversa bilátera. 
\end{Lema}
\begin{proof}
   \begin{itemize}
      \item La propiedad de ser abierta de $\dot{s}$ se sigue directamente de la definición de la topología de $\Lambda_P$.
      \item Dados $x,y\in U$, si $\dot{s}(x)=\dot{s}(y)$ entonces $(\germ_{x}s_{U},x)=(\germ_{y}s_{U},y)$ y $x=y$. Por tanto $\dot{s}$ es inyectiva.
      \item Dado $y\in \dot{s}(U)$, se tiene $y=\dot{s}(x)=((\germ_{x}s_{U}),x)$ para algún $x\in U$. Entonces $\mathfrak{p}|^{\Lambda_P}_{\dot{s}(U)}(y)=\mathfrak{p}(\germ_{x}s_{U},x)=x\in U$. Por tanto $\mathfrak{p}|^{\Lambda_P}_{\dot{s}(U)}$ es una función sobreyectiva de $\dot{s}(U)$ en $U$.
         % https://tikzcd.yichuanshen.de/#N4Igdg9gJgpgziAXAbVABwnAlgFyxMJZABgBoBGAXVJADcBDAGwFcYkQBVEAX1PU1z5CKcqWLU6TVuwA6MqBBzA43ABQcAlDwkwoAc3hFQAMwBOEALZIyIHBCSiQAIxhgoSAMw2GLNohByCkoqPHwgZpYONHbWNC5unt5SfgEyFvQ4ABZm9ADWwGjcAD4AesByADL0Fk5Q9AD6hfXl8orKaprc2txAA
\begin{center}
\begin{tikzcd}
                                   & \dot{s}(U) \arrow[ld, "\mathfrak{p}|^{\Lambda_p}_{\dot{s}(U)}", bend left] \\
U \arrow[ru, "\dot{s}", bend left] &                                                                           
\end{tikzcd}
\end{center}

         Dado $x\in U$ se tiene
         $$
         \begin{aligned}
            \mathfrak{p}|^{\Lambda_P}_{\dot{s}(U)}(\dot{s}(x))&=\mathfrak{p}|^{\Lambda_P}_{\dot{s}(U)}(\germ_{x}s_{U},x)\\
                                                              &=x\\
                                                              &=1_{U}(x),
         \end{aligned}
         $$
         así que $\mathfrak{p}|^{\Lambda_P}_{\dot{s}(U)}\circ \dot{s}=1_{U}$. Dado $z\in\dot{s}(U)$, existe $w\in U$ tal que $z=\dot{s}(w)=(\germ_{w}s_{U},w)$; entonces
         $$
         \begin{aligned}
            \dot{s}(\mathfrak{p}|^{\Lambda_P}_{\dot{s}(U)}(\germ_{w}s_{U},w))&=\dot{s}(w)\\
                                                                             &=w\\
                                                                             &=1_{\dot{s}(U)}(y),
         \end{aligned}
         $$
         luego $\dot{s}\circ\mathfrak{p}|^{\Lambda_P}_{\dot{s}(U)}=1_{\dot{s}(U)}$. Obtenemos que $\dot{s}:U\to\dot{s}(U)$ tiene a $\mathfrak{p}|^{\Lambda_P}_{\dot{s}(U)}$ como inversa bilátera.
   \end{itemize}
\end{proof}
\begin{Cor}
   Sea $P\in\PreSh(X)$. El manojo $\mathfrak{p}:\Lambda_P\to X$ es un homeomorfismo local.
\end{Cor}
\begin{proof}
   Sea $z\in\Lambda_P$. Existen $x\in X$ y $(U,s)\in\Su(x) (U\in\mathcal{O}(X),x\in U,s\in PU)$ tales que $z=(\germ_{x}s_{U},x)$.
   \begin{itemize}
      \item Tenemos $\dot{s}(U)\subseteqab \Lambda_P$, y como $x\in U$ y $\dot{s}(x)=(\germ_{x}s_{U},x)=z$; entonces $z\in\dot{s}(U)$.
      \item $\mathfrak{p}(\dot{s}(U))=\mathfrak{p}|^{\Lambda_P}_{\dot{s}(U)}(\dot{s}(U))=U\subseteqab X$.
      \item Como $\mathfrak{p}$ es continua, su restricción $\mathfrak{p}|^{\Lambda_P}_{\dot{s}(U)}$ es continua, y ésta tiene por inversa a la función continua $\dot{s}(U)$. Por tanto $\mathfrak{p}|^{\Lambda_P}_{\dot{s}(U)}:\dot{s}(U)\to U$ es un homeomorfismo.
      Obtenemos que $\mathfrak{p}:\Lambda_P\to X$ es un homeomorfismo local.
   \end{itemize}
\end{proof}

\subsection{El funtor $\Lambda$}
Como hemos resaltado, obtenemos el manojo $\Lambda_P$ sobre $X$, por cada prehaz $P$ sobre $X$. Nuestro deseo es que esta asignación en objetos nos produzca un funtor $\Lambda:\PreSh(X)\to\Top/X$; surge la pregunta de qué flecha de $\Top/X$ asignar a una flecha $h:P\dot{\to} Q$ en $\PreSh(X)$.

Sea $h:P\dot{\to}Q$ una transformación natural entre prehaces sobre $X$; queremos definir $\Lambda_h:\Lambda_{P}\to\Lambda_{Q}$, de modo que $\Lambda_h$ sea una función continua, y que $\mathfrak{q}\circ\Lambda_h=\mathfrak{p}$, es decir, que el siguiente diagrama en $\Top$ conmute:
% https://tikzcd.yichuanshen.de/#N4Igdg9gJgpgziAXAbVABwnAlgFyxMJZABgBpiBdUkANwEMAbAVxiRAB12AZOgWwCModAPrAACgF8QE0uky58hFAEZyVWoxZtOPAUNEBFKTLnY8BIquXr6zVohAANaephQA5vCKgAZgCcIXiQyEBwIJFUNO21uPkERYAALY1kQf0CI6jCkACZqWy0HTl46HET-OgBrYABHFN8AoMQQ7MQ8qMKOdhKyiuq0KWoGOn4YBjF5cyUQPyx3RJwXCSA
\begin{center}
\begin{tikzcd}
\Lambda_{P} \arrow[r, "\Lambda_{h}"] \arrow[rd, "\mathfrak{p}"'] & \Lambda_{Q} \arrow[d, "\mathfrak{q}"] \\
                                                                 & X                                    
\end{tikzcd}
\end{center}

Recordemos que 
$$
   \Lambda_P=\left\lbrace (\Pgerm_{x}s_{U},x)\mid x\in U\subseteqab X, s\in PU\right\rbrace
$$
y
$$
   \Lambda_Q=\left\lbrace (\Qgerm_{x}s_{U},x)\mid x\in U\subseteqab X, s\in QU\right\rbrace.
$$
Podría pensarse $\Lambda_h(\Pgerm_{x}s_{U},x)=(\Qgerm_{x}s_{U},x)$ (para cada $(\Pgerm_{x}s_{U},x)\in\Lambda_P$) como una asignación natural; sin embargo, el no uso de la transformación natural $h$ genera dudas. Otro camino que puede pensarse es el siguiente: dado $(\Pgerm_{x}s_{U},x)\in \Lambda_P$, como $U\in\mathcal{O}(X)$ y $h:\Lambda_P\dot{\to}\Lambda_Q$ es una transformación natural entre los funtores $\Lambda_P,\Lambda_Q:\mathcal{O}(X)^{\text{op}}\to\Set$, tenemos la función de conjuntos $h_{U}:PU\to QU$; como $s\in PU$ entonces $h_{U}s\in QU$, y $\Lambda_h(\Pgerm_{x}s_{U},x)=(\Qgerm_{x}(h_{U}s)_{U},x)$ (para cada $(\Pgerm_{x}s_{U},x)\in\Lambda_P$) nos sugiere otra definición de $\Lambda_h$. Como la anterior función trabaja con clases de equivalencia (los gérmenes sobre cada punto), debemos garantizar que está bien definida. 
\begin{itemize}
   \item Sean $x\in X; (U,s),(V,t)\in\PSu(X)$. Supongamos $\Pgerm_{x}s_{U}=\Pgerm_{x}t_{V}$ y garanticemos $\Qgerm_{x}(h_{U}s)_{U}=\Qgerm_{x}(h_{V}t)_{V}$. Tenemos $(U,s)\sim_{P,x}(V,t)$, luego existe $W\subseteqab X$ tal que $W\subseteq U$ y $W\subseteq V$, con $x\in W$, y además $s|^{U}_{W}=t|^{V}_{W}$. Tenemos en $\mathcal{O}(X)$ el diagrama
      % https://tikzcd.yichuanshen.de/#N4Igdg9gJgpgziAXAbVABwnAlgFyxMJZABgBpiBdUkANwEMAbAVxiRAFUQBfU9TXfIRQBGUsKq1GLNgHVuvEBmx4CRMgCYJ9Zq0QgAatwkwoAc3hFQAMwBOEALZJRIHBCTEe1u48TPXSdS4KLiA
\begin{center}
\begin{tikzcd}
U &                         \\
  & W \arrow[lu] \arrow[ld] \\
V &                        
\end{tikzcd}
\end{center}

      Como $h:P\dot{\to} Q$ es una transformación natural, tenemos en $\Set$ el siguiente diagrama conmutativo:
      % https://tikzcd.yichuanshen.de/#N4Igdg9gJgpgziAXAbVABwnAlgFyxMJZABgBpiBdUkANwEMAbAVxiRAAUBVEAX1PUy58hFAEZSoqrUYs27AOq9+IDNjwEiZAExT6zVog4A1JQLXCiW8rpkGQARW58zQjSgDMEm-rb3FzlUF1EWQrHWo9WUN7Ex4pGCgAc3giUAAzACcIAFskMhAcCCRPaR9DAAsAfW5qOHKsNJw8gMycpHECosQAFgjbNir-ZVbcxCtOpABWPrKQKtjhrNH8wvbqBjoAIxgGdiCLQwysRPKmmaiQAAoAHWu0crowQuzgCB4ASgAfAD1gTh5KsB5DxTCARkhxqtEB1InYbncHk8cq8Pj9gEYAUCQS0lsVqFDeqULvD7o9niivr9-oDgaDwYhphMeuc4bdSUiXm9KejMbT1lsdntzG4QEcTk04jwgA
\begin{center}
\begin{tikzcd}
PU \arrow[rr, "h_U"] \arrow[rd, "(\phantom{o})|^{U}_{W}"'] &                      & QU \arrow[rd, "(\phantom{o})|^{U}_{W}"]  &    \\
                                                           & PW \arrow[rr, "h_W"] &                                          & QW \\
PV \arrow[rr, "h_V"] \arrow[ru, "(\phantom{o})|^{V}_{W}"]  &                      & QV \arrow[ru, "(\phantom{o})|^{V}_{W}"'] &   
\end{tikzcd}
\end{center}

      Como $s\in PU, t\in PV$ y $s|^{U}_{W}=t|^{V}_{W}$, siguiendo en anterior diagrama vemos que:
      $$
         (h_{U}s)|^{U}_{W}=h_{W}(s|^{U}_{W})=h_{W}(t|^{V}_{W})=(h_{V}t)|^{V}_{W},
      $$
      de modo que $(U,h_{U}s)\sim_{Q,x}(V,h_{V}t)$ y $\Qgerm_{x}(h_{U}s)_{U}=\Qgerm_{x}(h_{V}t)_{V}$. Con esto, $\Lambda_h$ está bien definida, pues dados $(\Pgerm_{x}s_{U},x),(\Pgerm_{x}t_{V},x)\in\Lambda_P$, si $(\Pgerm_{x}s_{U},x)=(\Pgerm_{x}t_{V},x)$ entonces
      $$
      \begin{aligned}
         \Lambda_h(\Pgerm_{x}s_{U},x)&=(\Qgerm_{x}(h_{U}s)_{U},x)\\
                                     &=(\Qgerm_{x}(H_{V}t)_{V},x)\\
                                     &=\Lambda_h(\Pgerm_{x}t_{V},x).
      \end{aligned}
      $$
\end{itemize}
Ahora veamos que $\Lambda_h$ es continua:
\begin{itemize}
   \item Probamos que $\Lambda_h$ es continua puntualmente. Sea $(\Pgerm_{x}s_{u},x)\in \Lambda_P$ (tenemos $x\in U\subseteqab(X), s\in PU$). Tenemos $\Lambda_h(\Pgerm_{x}s_{U},x)=(\Qgerm_{x}(h_{U}s),x)$. Sea $\dot{t}(V)\in\mathcal{B}_{\Lambda_Q}$ una vecindad abierta básica de $(\Qgerm_{x}(h_{U}s)_{U},x)$, es decir, $V\in\mathcal{O}(X)$, $t\in QV$ y $(\Qgerm_{x}(h_{U}s)_{U},x)\in\dot{t}_{V}(V)$, con lo cual existe $y\in V$ tal que:
      $$
      \begin{aligned}
         (\Qgerm_{x}(h_{U}s)_{U},x)&=\dot{t}_{V}(y)\\
                                   &=(\Qgerm_{y}t_{V},y).
      \end{aligned}
      $$
      De este modo, $x=y$, $x\in V$ y $\Qgerm_{x}(h_{U}s)_{U}=\Qgerm_{x}t_{V}$, es decir, $(U,h_{U}s)\sim_{Q,x}(V,t)$; así, existe $W\in\mathcal{O}(X)$ tal que $x\in W\subseteq U\cap V$ y $(h_{U}s)|^{U}_{W}=t|^{V}_{W}$. Como $h:P\dot{\to} Q$ es una transformación natural y $W\subseteq U$, tenemos en $\Set$ el siguiente diagrama conmutativo:
      % https://tikzcd.yichuanshen.de/#N4Igdg9gJgpgziAXAbVABwnAlgFyxMJZABgBpiBdUkANwEMAbAVxiRAAUBVEAX1PUy58hFAEZyVWoxZsAitz4DseAkTKjJ9Zq0QcA6r34gMy4UXEbqWmbtkGekmFADm8IqABmAJwgBbJGQgOBBIAExW0jogABQAOrFoABZ0YMG+wADyPACUAD4AesCcPAD6wHo8INQMdABGMAzsgioiIF5Yzok4hp4+-ojiQSGIAMwR2mxxCcmpfpk5BUWl5ZWKIN5+AdTBSIPWUYklCkYb-eFDSGNSE7qHBtV1DU2mqrrtnd0OPEA
\begin{center}
\begin{tikzcd}
PU \arrow[d, "(\phantom{O})|^{U}_{W}"'] \arrow[r, "h_U"] & QU \arrow[d, "(\phantom{O})|^{U}_{W}"] \\
PW \arrow[r, "h_W"']                                     & QW                                    
\end{tikzcd}
\end{center}

      Como $s\in PU$,
      $$
      \begin{aligned}
         h_{W}(s|^{U}_{W})&=(h_{U}s)|^{U}_{W}=t|^{V}_{W}.
      \end{aligned}
      $$
      Notemos que $x\in W$ y 
      $$
         \dot{(s|^{U}_{W})}_{W}(x)=(\Pgerm_{x}(s|^{U}_{W})_W,x)=(\Pgerm_{x}s_{U},x),
      $$
      y por tanto $(\Pgerm_{x}s_{U},x)\in\dot{(s|^{U}_{W})(W)}$. Además $\Lambda_h(\Pgerm_{x}s_{U},x)=(\Qgerm_{x}(h_{U}s)_{U},x)$, con lo cual $(\Qgerm_{x}(h_{U}s)_{U},x)\in \Lambda_h(\dot{(s|^{U}_{W}})(W))$. Ahora, sea $z\in \Lambda_h(\dot{(s|^{U}_{W})}(W))$. Existe $\tilde{w}\in\dot{(s|^{U}_{W})}(W)$ tal que $z=\Lambda_h(\tilde{w})$. Existe $w\in W\subseteq V$ tal que $\tilde{w}=\dot{(s|^{U}_{W})}(w)=(\Pgerm_{w}(s|^{U}_{W}),w)$. Por ende
      $$
      \begin{aligned}
         z&=\Lambda_h(\tilde{w})\\
          &=\Lambda_h(\Pgerm_{w}(s|^{U}_{W}),w)\\
          &=(\Qgerm_{w}(h_{W}(s|^{U}_{W}))_{W},w)\\
          &=(\Qgerm_{w}(t|^{V}_{W})_{W},w)\\
          &=(\Qgerm_{w}t_{V},w)\\
          &=\dot{t}_{V}(w)
      \end{aligned}
      $$
      y $w\in V$, con lo cual $z\in\dot{t}_{V}(V)$ y $\Lambda_h(\dot{(s|^{U}_{W})}(W))\subseteq \dot{t}_{V}(V)$. Como $\dot{(s|^{U}_{W})}(W)\in\mathcal{B}_{\Lambda_P}$, esto completa la prueba de que $\Lambda_h:\Lambda_P\to \Lambda_Q$ es una función continua.
   \item Sumado a lo anterior, el siguiente diagrama conmuta:
      % https://tikzcd.yichuanshen.de/#N4Igdg9gJgpgziAXAbVABwnAlgFyxMJZABgBpiBdUkANwEMAbAVxiRAB12AZOgWwCModAPrAACgF8QE0uky58hFAEZyVWoxZtOPAUNEBFKTLnY8BIquXr6zVohAANaephQA5vCKgAZgCcIXiQyEBwIJFUNO21uPkERYAALY1kQf0CI6jCkACZqWy0HTl46HET-OgBrYABHFN8AoMQQ7MQ8qMKOdhKyiuq0KWoGOn4YBjF5cyUQPyx3RJwXCSA
\begin{center}
\begin{tikzcd}
\Lambda_{P} \arrow[r, "\Lambda_{h}"] \arrow[rd, "\mathfrak{p}"'] & \Lambda_{Q} \arrow[d, "\mathfrak{q}"] \\
                                                                 & X                                    
\end{tikzcd}
\end{center}

      pues, dado $(\Pgerm_{x}s_{U},x)\in \Lambda_P$, tenemos
      $$
      \begin{aligned}
         \mathfrak{q}(\Lambda_h(\Pgerm_{x}s_{U},x))&=\mathfrak{q}(\Qgerm_{x}(h_{U}s)_{U},x)\\
                                                   &=x\\
                                                   &=\mathfrak{p}(\Pgerm_{x}s_{U},x),
      \end{aligned}
      $$
      luego $\mathfrak{p}=\mathfrak{q}\circ\Lambda_h$.
\end{itemize}
Hemos probado la siguiente proposición:
\begin{Prop}
   Para cada $h:P\to Q$ en $\PreSh(X)$, la función 
   $$
      \Lambda_h:\Lambda_P\to \Lambda_Q:(\Pgerm_{x}s_{U},x)\mapsto (\Qgerm_{x}(h_{U}s)_{U},x)
   $$
   es una flecha de $(\Lambda_P,\mathfrak{p})\to(\Lambda_Q,\mathfrak{q})$ en $\Top/X$.
\end{Prop}

% SIGUIENTE. PROBAR QUE $\Lambda:\PreSh(X)\to\Top/X$ ES UN FUNTOR

