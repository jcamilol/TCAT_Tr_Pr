En esta sección construimos y estudiamos un funtor $\Lambda$ de la categoría $\PreSh(X)$ de prehaces sobre $X$ en la categoría $\Top /X$ de manojos sobre $X$.

Iniciamos definiendo el ``suelo" de un elemento de $X$ respecto a un prehaz sobre $X$:
\begin{Def}[$P$-suelo de $x$]
   Dados $x\in X$ y $P\in \PreSh(X)$, definimos el ``P-suelo de $x$" (denotado como $\PSu (x)$) como el conjunto $\left\lbrace (U,s)\mid x\in U\in \mathcal{O}(X); s\in PU\right\rbrace$
\end{Def}
 Buscamos definir una relación de equivalencia sobre el suelo de cada elemento de $X$:
\begin{Def}
   Sean $P\in \PreSh(X)$ y $x\in X$. Definimos en $\PSu(x)$ la relación $\sim_{P,x}$ de la siguiente manera:
   \begin{center}
      Dados $(U,s),(V,t)\in \PSu(x)$, $(U,s)\sim_{P,x}(V,t)$, si y sólo si, existe $W\in\mathcal{O}(X)$ tal que $x\in W\subseteq U\cap V$ y $s|^{U}_{W}=t|^{V}_{W}$.
   \end{center}
   Si $(U,s)\sim_{P,x}(V,t)$, decimos que $(U,s)$ y $(V,t)$ tienen el mismo P-germen en $x$. 
\end{Def}
La igualdad $s|^{U}_{W}=t|^{V}_{W}$ con $W\subseteq U\cap V$ nos recuerda la idea de ``coincidir localmente" que se mostró en las motivaciones dadas en la primera sección.

Notemos que la condición $W\subseteq U\cap V$ se tiene si y sólo si $W\subseteq U$ y $W\subseteq V$, y que en este caso $s|^{U}_{W}$ y $t|^{V}_{W}$ son elementos de $P(W)$, siempre que $s\in PU$ y $t\in PV$. De esta forma, podemos representar $(U,s)\sim_{P,x}(V,t)$ con el cumplimiento simultaneo de los dos siguientes diagramas:
\input{Diagramas/Diag18.tex}
Donde el diagrama de la izquierda está en $\mathcal{O}(X)$, el de la derecha en $Set$ y las flechas azules denotan pertenencia conjuntista.

En las dos anteriores definiciones, si es claro el prehaz $P$ con el que se está trabajando, solemos denotar $\Su(x)$ y $\sim_{x}$ en lugar de \PSu(x) y $\sim_{P,x}$, respectivamente.
\begin{Prop}
   Para cualesquiera $P\in\PreSh(X)$ y $x\in X$, la relación $\sim_{x}$ es de equivalencia sobre $\Su(x)$.
\end{Prop}
\begin{proof}
   La reflexividad y simetría de $\sim_{x}$ se siguen directamente de la definición. Veamos que $\sim_{x}$ es transitiva. Sean $(T,r),(U,s),(V,t)\in\Su(x)$ tales que $(T,r)\sim_{x}(U,s)$ y $(U,s)\sim_{x}(V,t)$. Existen $W_1,W_2\in\mathcal{O}(X)$ tales que $x\in W_1\subseteq T\cap U$ y $x\in W_2\subseteq U\cap V$. Además $r|^{T}_{W_1}=s|^{U}_{W_1}$ y $s|^{U}_{W_1}=t|^{V}_{W_2}$. Tenemos $W_1\cap W_2\in\mathcal{O}(X)$ y $x\in W_1\cap W_2\subseteq T\cap V$; se forma en $\mathcal{O}(X)$ el siguiente diagrama conmutativo:
   
   \input{Diagramas/Diag19.tex}
   Como $P$ es un funtor contravariante de $\mathcal{O}(X)$ en $\Set$, obtenemos en $\Set$ el siguiente diagrama conmutativo:
   \input{Diagramas/Diag20.tex}
   Siguiéndolo tenemos que
   $$
   \begin{aligned}
      r|^{T}_{W_1\cap W_2}&=(r|^{T}_{W_1})|^{W_1}_{W_1\cap W_2}\\
                          &=(s|^{U}_{W_1})|^{W_1}_{W_1\cap W_2}\\
                          &=s|^{U}_{W_1\cap W_2}\\
                          &=(s|^{U}_{W_2})|^{W_2}_{W_1\cap W_2}\\
                          &=(t|^{V}_{W_2})|^{W_2}_{W_1\cap W_2}\\
                          &=t|^{V}_{W_1\cap W_2}.\\
   \end{aligned}
   $$
   Por tanto, $(T,r)\sim_{x} (V,t)$ y $\sim_{x}$ es transitiva. Obtenemos que $\sim_{x}$ es una relación de equivalencia en $\Su(x)$.
\end{proof}
Ahora consideramos las clases de equivalencia generadas por la relación de tener el mismo germen en un punto:
\begin{Def}
   Sean $P\in \PreSh(X)$ y $x\in X$. Para cada $(U,s)\in \Su(x)$, la clase de equivalencia de $(U,s)$ respecto a $\sim_{P,x}$ se denota por $\Pgerm_{x}s$, y la llamamos el germen de $(U,s)$ en $x$.
\end{Def}
El siguiente lema, que será utilizado más adelante, nos muestra que el germen de un elemento se conserva bajo restricciones; propiedad que en efecto concuerda con la intuición desarrollada hasta el momento.
\begin{Lema}
   Sean $P\in\PreSh(X)$ y $U,V\in\mathcal{O}(X)$ con $V\subseteq U$ y $s\in PU$. Si $x\in V$ entonces $\germ_{x}s=\germ_{x}s|^{U}_{V}$.
\end{Lema}
\begin{proof}
   Supongamos que $x\in V$; en particular $x\in V\subseteq U\cap V$. Tenemos los diagramas (izquierda en $\mathcal{O}(X)$ y derecha en $\Set$):
   \input{Diagramas/Diag21.tex}
   Como $V\subseteq V$ es la flecha identidad de $V$, y $P$ es en particular un funtor, entonces $P(V\subseteq V)$ es la flecha identidad de $PV$; como $s\in PU$ entonces $P(U\subseteq V)(s)=s|^{U}_{V}\in PV$, luego 
   $$
   s|^{U}_{V}=P(V\subseteq V)(s|^{U}_{V})=(s|^{U}_{V})|^{V}_{V}.
   $$
   Luego $s\sim_{x}s|^{U}_{V}$, de modo que las respectivas clases de equivalencia son iguales, es decir, $\germ_{x}s=\germ_{x}s|^{U}_{V}$.
\end{proof}
