En esta sección construimos y estudiamos un funtor $\Lambda$ de la categoría $\PreSh(X)$ de prehaces sobre $X$ en la categoría $\Top /X$ de manojos sobre $X$.

Iniciamos definiendo el ``suelo" de un elemento de $X$ respecto a un prehaz sobre $X$:
\begin{Def}[$P$-suelo de $x$]
   Dados $x\in X$ y $P\in \PreSh(X)$, definimos el ``P-suelo de $x$" (denotado como $\PSu (x)$) como el conjunto $\left\lbrace (U,s)\mid x\in U\in \mathcal{O}(X); s\in PU\right\rbrace$
\end{Def}
 Buscamos definir una relación de equivalencia sobre el suelo de cada elemento de $X$:
\begin{Def}
   Sean $P\in \PreSh(X)$ y $x\in X$. Definimos en $\PSu(x)$ la relación $\sim_{P,x}$ de la siguiente manera:
   \begin{center}
      Dados $(U,s),(V,t)\in \PSu(x)$, $(U,s)\sim_{P,x}(V,t)$, si y sólo si, existe $W\in\mathcal{O}(X)$ tal que $x\in W\subseteq U\cap V$ y $s|^{U}_{W}=t|^{V}_{W}$.
   \end{center}
   Si $(U,s)\sim_{P,x}(V,t)$, decimos que $(U,s)$ y $(V,t)$ tienen el mismo P-germen en $x$. 
\end{Def}
La igualdad $s|^{U}_{W}=t|^{V}_{W}$ con $W\subseteq U\cap V$ nos recuerda la idea de ``coincidir localmente" que se mostró en las motivaciones dadas en la primera sección.

Notemos que la condición $W\subseteq U\cap V$ se tiene si y sólo si $W\subseteq U$ y $W\subseteq V$, y que en este caso $s|^{U}_{W}$ y $t|^{V}_{W}$ son elementos de $P(W)$, siempre que $s\in PU$ y $t\in PV$. De esta forma, podemos representar $(U,s)\sim_{P,x}(V,t)$ con el cumplimiento simultaneo de los dos siguientes diagramas:
\input{Diagramas/Diag18.tex}
Donde el diagrama de la izquierda está en $\mathcal{O}(X)$, el de la derecha en $Set$ y las flechas azules denotan pertenencia conjuntista.

En las dos anteriores definiciones, si es claro el prehaz $P$ con el que se está trabajando, solemos denotar $\Su(x)$ y $\sim_{x}$ en lugar de \PSu(x) y $\sim_{P,x}$, respectivamente.
\begin{Prop}
   Para cualesquiera $P\in\PreSh(X)$ y $x\in X$, la relación $\sim_{x}$ es de equivalencia sobre $\Su(x)$.
\end{Prop}
\begin{proof}
   La reflexividad y simetría de $\sim_{x}$ se siguen directamente de la definición. Veamos que $\sim_{x}$ es transitiva. Sean $(T,r),(U,s),(V,t)\in\Su(x)$ tales que $(T,r)\sim_{x}(U,s)$ y $(U,s)\sim_{x}(V,t)$. Existen $W_1,W_2\in\mathcal{O}(X)$ tales que $x\in W_1\subseteq T\cap U$ y $x\in W_2\subseteq U\cap V$. Además $r|^{T}_{W_1}=s|^{U}_{W_1}$ y $s|^{U}_{W_1}=t|^{V}_{W_2}$. Tenemos $W_1\cap W_2\in\mathcal{O}(X)$ y $x\in W_1\cap W_2\subseteq T\cap V$; se forma en $\mathcal{O}(X)$ el siguiente diagrama conmutativo:
   
   \input{Diagramas/Diag19.tex}
   Como $P$ es un funtor contravariante de $\mathcal{O}(X)$ en $\Set$, obtenemos en $\Set$ el siguiente diagrama conmutativo:
   \input{Diagramas/Diag20.tex}
   Siguiéndolo tenemos que
   $$
   \begin{aligned}
      r|^{T}_{W_1\cap W_2}&=(r|^{T}_{W_1})|^{W_1}_{W_1\cap W_2}\\
                          &=(s|^{U}_{W_1})|^{W_1}_{W_1\cap W_2}\\
                          &=s|^{U}_{W_1\cap W_2}\\
                          &=(s|^{U}_{W_2})|^{W_2}_{W_1\cap W_2}\\
                          &=(t|^{V}_{W_2})|^{W_2}_{W_1\cap W_2}\\
                          &=t|^{V}_{W_1\cap W_2}.\\
   \end{aligned}
   $$
   Por tanto, $(T,r)\sim_{x} (V,t)$ y $\sim_{x}$ es transitiva. Obtenemos que $\sim_{x}$ es una relación de equivalencia en $\Su(x)$.
\end{proof}
Ahora consideramos las clases de equivalencia generadas por la relación de tener el mismo germen en un punto:
\begin{Def}[Germen en un punto]
   Sean $P\in \PreSh(X)$ y $x\in X$. Para cada $(U,s)\in \Su(x)$, la clase de equivalencia de $(U,s)$ respecto a $\sim_{P,x}$ se denota por $\Pgerm_{x}s_{U}$, y la llamamos el germen de $(U,s)$ en $x$.
\end{Def}
Nuevamente, si por el contexto es claro con qué prehaz estamos trabajando, puede omitirse el ``$P-$" en la anterior definición.

El siguiente lema, que será utilizado más adelante, nos muestra que el germen de un elemento se conserva bajo restricciones; propiedad que en efecto concuerda con la intuición desarrollada hasta el momento.
\begin{Lema}\label{Lema:LemaAzul}
   Sean $P\in\PreSh(X)$ y $U,V\in\mathcal{O}(X)$ con $V\subseteq U$ y $s\in PU$. Si $x\in V$ entonces $\germ_{x}s_{U}=\germ_{x}(s|^{U}_{V})_{V}$.
\end{Lema}
\begin{proof}
   Supongamos que $x\in V$; en particular $x\in V\subseteq U\cap V$. Tenemos los diagramas (izquierda en $\mathcal{O}(X)$ y derecha en $\Set$):
   \input{Diagramas/Diag21.tex}
   Como $V\subseteq V$ es la flecha identidad de $V$, y $P$ es en particular un funtor, entonces $P(V\subseteq V)$ es la flecha identidad de $PV$; como $s\in PU$ entonces $P(U\subseteq V)(s)=s|^{U}_{V}\in PV$, luego 
   $$
   s|^{U}_{V}=P(V\subseteq V)(s|^{U}_{V})=(s|^{U}_{V})|^{V}_{V}.
   $$
   Luego $(s,U)\sim_{x}(s|^{U}_{V},V)$, de modo que las respectivas clases de equivalencia son iguales, es decir, $\germ_{x}s_{U}=\germ_{x}(s|^{U}_{V})_{V}$.
\end{proof}
Al pasar al cociente por la relación de equivalencia de ``tener el mismo germen", obtenemos el tallo (stalk en inglés) de un prehaz en un punto dado:
\begin{Def}[Tallo en un punto]
   Sean $P\in\PreSh(X)$ y $x\in X$. Al conjunto cociente
   $$
   P_{x}:=\PSu(x)/\sim_{x}=\left\lbrace \Pgerm_{x}s_{U}\mid (U,s)\in\PSu(x)\right\rbrace
   $$
   lo llamamos el tallo de $P$ en $x$.
\end{Def}
Los tallos de un prehaz no son necesariamente disyuntos; ello lo muestra el siguiente ejemplo:
\begin{Ejm}
   Tomemos $X=\left\lbrace a,b\right\rbrace$ (con $a\neq b$) dotado con la topología trivial $\tau=\left\lbrace \phi,X\right\rbrace$, $C$ el haz de funciones continuas sobre $\mathbb{R}$ y $f:X\to \mathbb{R}$ la función constante en $0$ ($f(a)=f(b)=0$). Las únicas funciones continuas de $X$ en $\mathbb{R}$ son las constantes, es decir $CX=\left\lbrace g:X\to\mathbb{R}\mid g(a)=f(a)\right\rbrace$. Notemos que
   $$
   \begin{aligned}
      \Su(a)&=\left\lbrace (U,g)\mid a\in U\subseteqab X;g\in CU\right\rbrace\\
            &=\left\lbrace (X,g)\mid g:X\to\mathbb{R} \text{ es continua}\right\rbrace\\
            &=\left\lbrace (X,g)\mid g(a)=g(b)\right\rbrace;\\
   \end{aligned}
   $$
   del mismo modo se llega a $\Su(b)=\left\lbrace (X,g)\mid g(a)=g(b)\right\rbrace$, y por tanto $\Su(a)=\Su(b)$. Sea $(X,g)\in\germ_{a}f_{X}$. Entonces $(X,g)\in\Su(a)=\Su(b)$ y $(X,g)\sim_{a}(X,f)$ y por tanto existe $W\subseteqab X$ tal que $a\in W\subseteqab X$ y $g|^{X}_{W}=f|^{X}_{W}$; pero como el único abierto no vacío de $X$ es $X$, tenemos $W=X$ y $f=f|^{X}_{X}=g|^{X}_{X}=g$. De este modo $\germ_{a}f_{X}=\left\lbrace (X,f)\right\rbrace$. Análogamente se llega a $\germ_{b}f_{X}=\left\lbrace (X,f)\right\rbrace$. Con lo anterior, $\germ_{a}f_{X}\in C_{a}\cap C_{b}$, esto es, $C_a$ y $C_b$ no son disyuntos para $a\neq b$. Por lo tanto los tallos del haz $C$ no son disyuntos.
\end{Ejm}
Nos interesa forzar a los tallos de un prehaz a que tengan intersección vacía; para esto tomamos su unión disyunta (que resulta ser el coproducto de los tallos en la categoría $\Set$):
\begin{Def}
   Sea $P\in\PreSh(X)$. Denotamos por $\Lambda_P$ a la unión disyunta de los tallos de $P$ en los elementos de $X$:
   $$
   \begin{aligned}
      \Lambda_P&:=\coprod_{x\in X}P_{x}\\
               &=\bigcup_{x\in X}(P_{x}\times\left\lbrace x\right\rbrace)\\
               &=\left\lbrace (\Pgerm_{x}s_{U},x)\mid x\in X;(U,s)\in\PSu(x)\right\rbrace.
   \end{aligned}
   $$
\end{Def}
Queremos obtener por cada prehaz sobre $X$ un manojo sobre $X$; el conjunto $\Lambda_P$ es el primer paso para esto; ahora definimos una función de dicho conjunto en $X$:
\begin{Def}
   Sea $P\in\PreSh(X)$. Definimos la función $\mathfrak{p}:\Lambda_P\to X$ mediante $\mathfrak{p}(\Pgerm_{x}s_{U},x)=x$ para cada $(\Pgerm_{x}s_{U},x)\in\Lambda_P$, y la llamamos la función canónica de $\Lambda_P$ en $X$. 
\end{Def}
Para que $\mathfrak{p}$ represente un manojo sobre $X$, debemos dotar a $\Lambda_P$ de una topología; con este fin introducimos una nueva familia de funciones:
\begin{Def}
   Para cualesquiera $U\in\mathcal{O}(X)$ y $s\in PU$, definimos la función $\dot{s}:U\to\Lambda_P$ mediante $\dot{s}(x)=(\Pgerm_{x}s_{U},x)$ para cada $x\in U$.
\end{Def}
Para cada función $\dot{s}$, el siguiente diagrama en $\Set$ es conmutativo:
   \input{Diagramas/Diag22.tex}
Este diagrama nos recuerda las secciones transversales sobre un manojo. Deseamos que la topología que asignemos a $\Lambda_P$ haga de cada función $\dot{s}$ una sección transversal sobre $\mathfrak{p}$.
\begin{Prop}
   Sea $P\in\PreSh(X)$. El conjunto
   $$
      \mathcal{B}_{\Lambda_P}:=\left\lbrace \dot{s}(U)\mid U\in\mathcal{O}(X);s\in PU\right\rbrace
   $$
   es base para una topología sobre $\Lambda_P$.
\end{Prop}
\begin{proof}
   Hacemos uso de la caracterización dada en la Proposición \Ref{Prop:ConjuntoEsBase}.
   \begin{itemize}
      \item Veamos que $\bigcup \mathcal{B}_{\Lambda_P}=\Lambda_P$, es decir,
      $$
         \bigcup_{\substack{U\in\mathcal{O}(X) \\ s\in PU}}\dot{s}(U)=\Lambda_P.
      $$
      La contenencia $\subseteq$ es inmediata, pues $\dot{s}(U)\subseteq\Lambda_P$ para cada $U\in\mathcal{O}(X)$ y $s\in PU$. Ahora, sea $z\in\Lambda_{P}$. Existen $y\in X$ y $(V,t)\in\Su(y)$ tales que $z=(\germ_{y}t_{V},y)$. Tenemos $y\in V\in\mathcal{O}(X)$ y $t\in PV$; como $\dot{t}_{V}(y)=(\germ_{y}t_{V},y)=z$, tenemos $z\in\dot{t}(V)\subseteq\bigcup{\dot{s}(U)}$; esto nos da la contenencia $\supseteq$. Obtenemos que $\Lambda_P$ es unión de elementos de $\mathcal{B}_{\Lambda_P}$.
   \item Ahora, sean $A,B\in\mathcal{B}_{\Lambda_P}$ y probemos que $A\cap B$ es unión de elementos de $\mathcal{B}_{\Lambda_P}$. Existen $T,V\in\mathcal{O}(X), t\in PT$ y $r\in PV$ tales que $A=\dot{t}(T)$ y $B=\dot{r}(V)$. Si $A\cap B=\phi$, entonces $A\cap B$ es la unión vacía de elementos de $\mathcal{B}_{\Lambda_P}$. Supongamos que $A\cap B\neq\phi$. Dado
      $$ 
         z\in A\cap B =\dot{t}(T)\cap\dot{r}(V)=\left\lbrace (\germ_{x}t_{T},x)\mid x\in T\right\rbrace\cap\left\lbrace (\germ_{x}r_{V},x)\mid x\in V\right\rbrace,
      $$
      existe $x\in T\cap V\in \mathcal{O}(X)$ tal que $z=(germ_{x}t_{T},x)=(germ_{x}r_{V},x)$; luego $germ_{x}t_{T}=germ_{x}r_{V}$ y $(T,t)\sim_{x}(V,r)$. Así, existe $W_{z}\in\mathcal{O}(X)$ tal que $x\in W_z\subseteq T\cap V$ y $t|^{T}_{W_z}=r|^{V}_{W_z}\in PW_{z}$. Probemos que
      $$
         A\cap B=\bigcup_{z\in A\cap B}\dot{(t|^{T}_{W_z})}(W_z).
      $$
      \begin{itemize}
         \item[($\subseteq$)] Sea $\omega\in A\cap B$. Existe $x\in W_{\omega}$ tal que $\omega=(\germ_{x}t_{T},x)$. Como, por el Lema \Ref{Lema:LemaAzul} se tiene $\germ_{x}t_{T}=\germ_{x}(t|^{T}_{W_{\omega}})_{W_{\omega}}$, entonces
            $$
            \begin{aligned}
               \omega&=(\germ_{x}t_{T},x)\\
                     &=(\germ_{x}(t|^{T}_{W_{\omega}})_{W_{\omega}},x)\\
                     &=\dot{(t|^{T}_{W_{\omega}})}_{W_{\omega}}(x)
            \end{aligned}
            $$
            con $x\in W_{\omega}$; así, 
            $$
               \omega\in\dot{(t|^{T}_{W_\omega})}(W_\omega)
            $$
            con $\omega\in A\cap B$, luego
            $$
               \omega\in\bigcup_{z\in A\cap B}\dot{(t|^{T}_{W_z})}(W_z),
            $$
            y con esto,
            $$
            A\cap B\subseteq\bigcup_{z\in A\cap B}\dot{(t|^{T}_{W_z})}(W_z).
            $$
         \item[($\supseteq$)] Sea $y\in \bigcup_{z\in A\cap B}\dot{(t|^{T}_{W_z})}(W_z)$. Existe $\omega\in A\cap B$ tal que $y\in\dot{(t|^{T}_{W_\omega})}(W_\omega)$, de modo que existe $x\in W_{\omega}$ tal que $y=\dot{(t|^{T}_{W_\omega})}(x)=(\germ_{x}(t|^{T}_{W_\omega}),x)$ con $W_\omega\subseteq T\cap V$ y $t|^{T}_{W_\omega}=r|^{V}_{W_\omega}$, con lo cual $(T,t)\sim_{x}(V,r)$ y $\germ_{x}t_{T}=\germ_{x}r_{V}$. Como $\germ_{x}t_{T}=germ_{x}(t|^{T}_{W_\omega})_{W_\omega}$, tenemos
            $$
            \begin{aligned}
               y&=(\germ_{x}(t|^{T}_{W_\omega})_{W_\omega},x)\\
                &=(\germ_{x}t_{T},x)\\
                &=(\germ_{x}r_{V},x),
            \end{aligned}
            $$
            es decir $y=\dot{t}(x)=\dot{r}(x)$ con $x\in W_{\omega}\subseteq T\cap V$, de modo que $y\in\dot{t}(T)\cap\dot{r}(V)=A\cap B$. Obtenemos $\bigcup_{z\in A\cap B}\dot{(t|^{T}_{W_z})}(W_z)\subseteq A\cap B$.
      \end{itemize}
      Así, $A\cap B=\bigcup_{z\in A\cap B}\dot{(t|^{T}_{W_z})}(W_z)$, y para cada $z\in A\cap B$, $\dot{(t|^{T}_{W_z})}(W_z)\in\mathcal{B}_{\Lambda_P}$; es decir, la intersección de dos elementos de $\Lambda_P$ es unión de elementos de $\mathcal{B}_{\Lambda_P}$. Concluimos que $\mathcal{B}_{\Lambda_P}$ es base para una topología sobre $\Lambda_P$.
   \end{itemize}
\end{proof}

