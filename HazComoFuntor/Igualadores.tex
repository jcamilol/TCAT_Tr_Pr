\begin{Def}
   En una categoría arbitraria $\textbf{C}$, sean $f,g:A\to B$ flechas paralelas. Un igualador de $f$ y $g$ es una pareja $\langle E,e\rangle$, con $E\in\textbf{C}$ y $e:E\to A$ en $\textbf{C}$, tal que $f\circ e=g\circ e$, y que es universal con esta propiedad, en el sentido de que si hay otra pareja $\langle U,u\rangle$ con $U\in \textbf{C}$ y $u:U\to A$ en $\textbf{C}$, tal que $f\circ u=g\circ u$, entonces existe una única flecha $v:U\to E$ en $\textbf{C}$ tal que $e\circ v=u$. 
\end{Def}
El siguiente diagrama conmutativo, que denominamos como `` diagrama igualador", se resume la anterior definición:
% https://tikzcd.yichuanshen.de/#N4Igdg9gJgpgziAXAbVABwnAlgFyxMJZARgBoAGAXVJADcBDAGwFcYkQBBEAX1PU1z5CKAEwVqdJq3YAhHnxAZseAkXLiaDFm0QgAovP7Kha0sQlbpugKo8JMKAHN4RUADMAThAC2SMSBwIJHVJbXY2XncvX0QQwKQyAPosRnZIMDYaACMYMChgmjgACyw3HCQAWkTLHRA3QzrogoCgxEScvObi0vK2mkZ6HMYABQEVYRAPLEci8s0pWscGzx8kAGYaeMR-RiwM9ih6YocQebDdWmWmxA2W5pr2ZlOQAaHR41VdKZny7kpuIA
\begin{center}
\begin{tikzcd}
E \arrow[r, "e"]                          & A \arrow[r, "f", shift left] \arrow[r, "g"', shift right] & B \\
U \arrow[u, "v", dashed] \arrow[ru, "u"'] &                                                           &  
\end{tikzcd}
\end{center}


Los ejemplos de igualadores que más estaremos trabajando son aquellos que aparecen en la categoría \normalfont{\textbf{Set}}:
\begin{Ejm}
   Sean $A$ y $B$ conjuntos y $f,g$ funciones de $A$ en $B$. Verifiquemos que un igualador de $f$ y $g$ está dado por $\langle E,e\rangle$, donde $E=\left\lbrace a\in A\mid f(a)=g(a)\right\rbrace$ y $e$ es la función inclusión de $E$ en $A$:
   \begin{itemize}
      \item Dado $x\in E$ se tiene $(f\circ e)(x)=f(e(x))=f(x)=g(x)=g(e(x))=(g\circ e)(x)$, es decir, $f\circ e=g\circ e$.
      \item Supongamos que existe $\langle U,u\rangle$ con $U\in \normalfont{\textbf{Set}}$ y $u:U\to A$ en $\normalfont{\textbf{Set}}$, tal que $f\circ u=g\circ u$. Podemos definir $v:U\to E$ vía $v(x)=u(x)$ para todo $x\in U$, e inmediatamente se tendrá $e\circ v=u$; igualmente, si $v'$ es una fleca de $U\to E$ en $\normalfont{\textbf{Set}}$ tal que $e\circ v'=u$ entonces para cada $x\in U$ se tiene $v'(x)=e(v'(x))=u(x)=e(v(x))=v(x)$, de modo que $v=v'$.
   \end{itemize}
\hspace{\fill}$\Diamond$
\end{Ejm}
Directamente de la definición de igualadores, podemos derivar algunas propiedades que serán útiles másadelante:
\begin{Prop}
  
\end{Prop}
