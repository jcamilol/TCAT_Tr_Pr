\begin{Def}
   En una categoría arbitraria $\textbf{C}$, sean $f,g:A\to B$ flechas paralelas. Un igualador de $f$ y $g$ es una pareja $\langle E,e\rangle$, con $E\in\textbf{C}$ y $e:E\to A$ en $\textbf{C}$, tal que $f\circ e=g\circ e$, y que es universal con esta propiedad, en el sentido de que si hay otra pareja $\langle U,u\rangle$ con $U\in \textbf{C}$ y $u:U\to A$ en $\textbf{C}$, tal que $f\circ u=g\circ u$, entonces existe una única flecha $v:U\to E$ en $\textbf{C}$ tal que $e\circ v=u$. 
\end{Def}
El siguiente diagrama conmutativo, que denominamos como `` diagrama igualador", se resume la anterior definición:
% https://tikzcd.yichuanshen.de/#N4Igdg9gJgpgziAXAbVABwnAlgFyxMJZARgBoAGAXVJADcBDAGwFcYkQBBEAX1PU1z5CKAEwVqdJq3YAhHnxAZseAkXLiaDFm0QgAovP7Kha0sQlbpugKo8JMKAHN4RUADMAThAC2SMSBwIJHVJbXY2XncvX0QQwKQyAPosRnZIMDYaACMYMChgmjgACyw3HCQAWkTLHRA3QzrogoCgxEScvObi0vK2mkZ6HMYABQEVYRAPLEci8s0pWscGzx8kAGYaeMR-RiwM9ih6YocQebDdWmWmxA2W5pr2ZlOQAaHR41VdKZny7kpuIA
\begin{center}
\begin{tikzcd}
E \arrow[r, "e"]                          & A \arrow[r, "f", shift left] \arrow[r, "g"', shift right] & B \\
U \arrow[u, "v", dashed] \arrow[ru, "u"'] &                                                           &  
\end{tikzcd}
\end{center}


Los ejemplos de igualadores que más estaremos trabajando son aquellos que aparecen en la categoría \normalfont{\textbf{Set}}:
\begin{Ejm}
   Sean $A$ y $B$ conjuntos y $f,g$ funciones de $A$ en $B$. Verifiquemos que un igualador de $f$ y $g$ está dado por $\langle E,e\rangle$, donde $E=\left\lbrace a\in A\mid f(a)=g(a)\right\rbrace$ y $e$ es la función inclusión de $E$ en $A$:
   \begin{itemize}
      \item Dado $x\in E$ se tiene $(f\circ e)(x)=f(e(x))=f(x)=g(x)=g(e(x))=(g\circ e)(x)$, es decir, $f\circ e=g\circ e$.
      \item Supongamos que existe $\langle U,u\rangle$ con $U\in \normalfont{\textbf{Set}}$ y $u:U\to A$ en $\normalfont{\textbf{Set}}$, tal que $f\circ u=g\circ u$. Podemos definir $v:U\to E$ vía $v(x)=u(x)$ para todo $x\in U$, e inmediatamente se tendrá $e\circ v=u$; igualmente, si $v'$ es una fleca de $U\to E$ en $\normalfont{\textbf{Set}}$ tal que $e\circ v'=u$ entonces para cada $x\in U$ se tiene $v'(x)=e(v'(x))=u(x)=e(v(x))=v(x)$, de modo que $v=v'$.
   \end{itemize}
\hspace{\fill}$\Diamond$
\end{Ejm}
En la práctica, si no hay lugar a confusiones, nos referimos indistintamente por `` igualador" tanto al par $\langle E,e\rangle$ como simplemente a la flecha $e$. Directamente de la definición de igualadores, podemos derivar algunas propiedades que serán útiles más adelante:
\begin{Prop}\label{Prop:Igualadores1}
   En cualquier categoría, todo igualador es un monomorfismo.  
\end{Prop}
\begin{proof}
   Sean $\textbf{C}$ una categoría, $f,g:A\to B$ flechas paralelas en $\textbf{C}$ y $\langle E,e\rangle$ un igualador de $f$ y $g$. Supongamos que existen flechas $i,j:F\to E$ en $\textbf{C}$ tales que $e\circ i=e\circ j$. Tenemos $(f\circ e)\circ j=(g\circ e)\circ j$, es decir, $f\circ (e\circ j)=g\circ (e\circ j)$. Como $e$ es un igualador de $f$ y $g$, existe una única flecha $k:F\to E$ en $\textbf{C}$ tal que $e\circ k=e\circ j$; trivialmente $j$ cumple esta propiedad, pero también lo hace $i$, pues por hipótesis $e\circ i=e\circ j$. Se sigue que $i=j$ y por tanto $e$ es un monomorfismo en $\textbf{C}$.

\end{proof}
Como en \normalfont{\textbf{Set}}, para una flecha es lo mismo ser monomorfismo que ser una función inyectiva, como corolario de lo anterior obtenemos que cualquier igualador en \normalfont{\textbf{Set}} es una función inyectiva.
\begin{Prop}\label{Prop:Igualadores2}
   Supongamos que en \normalfont{\textbf{Set}} el siguiente es un diagrama igualador:
   % https://tikzcd.yichuanshen.de/#N4Igdg9gJgpgziAXAbVABwnAlgFyxMJZARgBoAGAXVJADcBDAGwFcYkQBBEAX1PU1z5CKAEwVqdJq3YAhHnxAZseAkXLiaDFm0QgAojwkwoAc3hFQAMwBOEALZIxIHBCTrJ29m15XbDxO4uSGTO9FiM7JBgbDQARjBgUG40cAAWWJY4SAC0IVrSupbyvvbJzq6IIfGJZWkZWZU0jPTxjAAKAirCINZYJqlZmlI6ICaG3EA
\begin{center}
\begin{tikzcd}
E \arrow[r, "e"] & A \arrow[r, "f", shift left] \arrow[r, "g"', shift right] & B
\end{tikzcd}
\end{center}

   Entonces, para todo $a\in A$ tal que $f(a)=g(a)$, existe $\alpha\in E$ tal que $e(\alpha)=a$.
\end{Prop}
\begin{proof}
  Definimos $F:=\left\lbrace x\in A\mid f(x)=g(x)\right\rbrace (\subseteq A)$. Por la Proposición \ref{Prop:Igualadores1}, sabemos que $\langle F,\normalfont{in}_{F,A}\rangle$ (donde $\normalfont{in}_{F,A}$ es la función inclusión de $F$ en $A$), es un igualador de $f$ y $g$, con lo cual, existe una única flecha $v:F\to E$ tal que $e\circ v=\normalfont{in}_{F,A}$. Como $a\in F$, tenemos $\alpha:=v(a)\in E$ y $e(\alpha)=e(v(a))=\normalfont{in}_{F,A}(a)=a$. 

\end{proof}
