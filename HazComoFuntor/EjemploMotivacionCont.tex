Continuando con nuestro ``ejemplo como motivación" (Sección \ref{subsection:EjemploMotivacion}), resaltamos la importancia, para la validez de la propiedad \textbf{(P2)}, de la existencia de una buena ``condición de pegado", en el sentido de que las funciones $f_i$ ($i\in I$) se respetan dondequiera que se solapen: para cualesquiera $i,j\in I$ y cualquier $x\in U_i\cap U_j$, se tiene $f_i(x)=f_j(x)$; es este buen comportamiento local en subconjuntos de $U$ lo que nos permite el paso a un conocimiento global de $U$ mediante la función continua $f$ que se reconstruye al pegar los elementos de la familia $\left\lbrace f_i\right\rbrace_{i\in I}$. Notemos que $\left\lbrace f_i\right\rbrace_{i\in I}$ es un elemento del producto cartesiano $\prod_{i\in I} CU_i$. Como, para cualesquiera $i,j\in I$ se tiene $f_i\in CU_i$ y $f_j\in CU_j$, y como $U_i\cap U_j\subseteq U_i$ y $U_i\cap U_j\subseteq U_j$, obtenemos, fruto de restringir adecuadamente a intersecciones, las funciones ${f_i}|^{U_i}_{U_i\cap U_j}, {f_j}|^{U_j}_{U_i\cap U_j}\in C(U_i\cap U_j)$, con las cuales formamos las familia $\left\lbrace {f_i}|^{U_i}_{U_i\cap U_j}\right\rbrace_{(i,j)\in I\times I}$ y $\left\lbrace {f_j}|^{U_j}_{U_i\cap U_j}\right\rbrace_{(i,j)\in I\times I}$, que a su vez son elementos del producto cartesiano $\prod_{(i,j)\in I\times I}C(U_i\cap U_j)$. Estas construcciones nos sugieren la definición de las siguientes funciones:
\begin{itemize}
   \item $e:CU\to \prod_{i\in I}CU_i$ que a cada $f\in CU$ le asigna la familia $\left\lbrace f|^{U}_{U_i}\right\rbrace_{i\in I}$.
   \item $\pi_1:\prod_{i\in I}CU_i\to \prod_{(i,j)\in I\times I}C(U_i\cap U_j)$ que a cada $\left\lbrace f_i\right\rbrace_{i\in I}$ le asigna la familia $\left\lbrace {f_i}|^{U_i}_{U_i\cap U_j}\right\rbrace_{(i,j)\in I\times I}$.
   \item $\pi_2:\prod_{i\in I}CU_i\to \prod_{(i,j)\in I\times I}C(U_i\cap U_j)$ que a cada $\left\lbrace f_i\right\rbrace_{i\in I}$ le asigna la familia $\left\lbrace {f_j}|^{U_j}_{U_i\cap U_j}\right\rbrace_{(i,j)\in I\times I}$.
\end{itemize}
