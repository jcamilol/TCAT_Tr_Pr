Una constante en el quehacer matemático es el tránsito entre aspectos locales y aspectos globales. Consideremos un ejemplo enmarcado en el área de la topología. Sean $X$ un espacio topológico y $U$ un subconjunto abierto de $X$, al cual dotamos con un cubrimiento $\left\lbrace U_i\right\rbrace_{i\in I}$ de subconjuntos abiertos de $U$. Una función continua $f:U\to \mathbb{R}$ se presenta como una herramienta para entender globalmente el conjunto $U$, y fácilmente nos permite pasar al conocimiento local de $U$ en el siguiente sentido:
\begin{itemize}
   \item[\textbf{(P1)}] Si $V\stackrel{ab}\subseteq U$ entonces $f|_V:V\to\mathbb{R}$ es también una función continua. 
\end{itemize}
De forma recíproca, gracias al lema de pegado (Teorema \ref{tma:lemaPegado}), $f$ nos permite pasar de un apropiado conocimiento local de $U$ a un conocimiento global, en la siguiente forma:
\begin{itemize}
   \item[\textbf{(P2)}] Sea $\left\lbrace U_i\right\rbrace_{i\in I}$ un cubrimiento abierto de $U$. Si $f_i:f|_{U_i}:U_i\to\mathbb{R}$ es continua para todo $i\in I$, entonces $f:U\to\mathbb{R}$ es continua.
\end{itemize}
Las propiedades \textbf{(P1)} y \textbf{(P2)} pueden ser capturadas en lenguaje categórico. Para esto, consideremos la categoría $\mathcal{O}(X)$ que tiene como objetos los subconjuntos abiertos de $X$, y en la cual, dados $U,V\in \mathcal{O}(X)$, hay una flecha de $V$ en $U$ si y solo si $V\subseteq U$; dicha flecha en $\mathcal{O}(X)$ (que será la única de $V$ en $U$) la representamos igualmente mediante ``$V\subseteq U$". Ahora, para cada $U\in\mathcal{O}(X)$ definimos el conjunto $CU$ de todas las funciones reales continuas sobre U:
$$
CU:=\left\lbrace f:U\to\mathbb{R}\mid f \text{ es continua}\right\rbrace,
$$
y para cualquier flecha $V\subseteq U$ en $\mathcal{O}(X)$, definimos la función de conjuntos
\begin{center}
% https://tikzcd.yichuanshen.de/#N4Igdg9gJgpgziAXAbVABwnAlgFyxMJZABgBpiBdUkANwEMAbAVxiRAGEAKANQB1e4TAEZwYOGAEcABAFUAlIhABfUuky58hFAEZyVWoxZt2M5apAZseAkQBMe6vWatF7bmbVXNRXdv1OjRQAzDwt1ay1kez9HQxcQIIAfAH13JX0YKABzeCJQIIAnCABbJF0QHAgkWxV8otLEAGZqSqQAFha6LAY2Yro0OFb0pSA
\begin{tikzcd}[row sep=-4pt, column sep=7pt]
C(V\subseteq U): & CU \arrow[r]         & CV   \\
                 & f \arrow[r, maps to] & f|^U_V
\end{tikzcd}
\end{center}

que a cada función continua de $U$ en $\mathbb{R}$ le asigna su respectiva función restricción al subconjunto $V$, que a su vez es una función continua de $V$ en $\mathbb{R}$. Tendremos entonces la siguiente propiedad:
\begin{prop}
   La regla $C$ que a cada $U\in\mathcal{O}(X)$ le asigna el conjunto $CU$ y a cada flecha $V\subseteq U$ en $\mathcal{O}(X)$ le asigna la función restricción de V en U, $C(V\subseteq U)$, es un funtor contravariante de $\mathcal{O}(X)$ en \normalfont{\textbf{Set}}.
\end{prop}
\begin{proof}
   
\end{proof}
