\begin{Prop}
   Si $p:Y\to X$ es un homeomorfismo local, entonces $p$ es una función continua y abierta. Además, la colección de todos los conjuntos abiertos de $Y$ que satisfacen \textit{(i)} y \textit{(ii)} de la Definición \Ref{Def:HomeomorfismoLocal} forman una base para la topología de $Y$.
\end{Prop}
\begin{proof}
   \begin{itemize}
      \item Probamos la continuidad de $p$ puntualmente. Sean $y\in Y$ y $U\subseteqab X$ con $p(y)\in U$. Tenemos que $y\in V_y\subseteqab Y$ y $p(y)\in p(V_y)\subseteqab X$. Tomando $W=p(V_y)\cap U$ se tiene $p(y)\in W\subseteqab X$. Además $W\subseteq p(V_y)$ y $W\subseteq U$. Como $p|^{Y}_{V_y}:V_y\to p(V_y)$ es un homeomorfismo, entonces $p^{-1}(W)=(p|^{Y}_{V_y})^{-1}(W)\subseteqab V_y \subseteqab Y$, así que $y\in p^{-1}(W)\subseteqab Y$; igualmente, $p(p^{-1}(W))\subseteq W \subseteq U$, lo cual prueba que $p$ es continua en $y$. Como $y$ es arbitraria en $Y$, obtenemos que $p:Y\to X$ es continua.
      \item Sea $V\subseteqab Y$; probemos que $p(V)\subseteqab X$. Para cada $y\in V$ definimos $W_y=V_y\cap V\subseteqab V_y$. Como $p|^{V}_{V_y}:V_y\to p(V_y)$ es un homeomorfismo, en particular es una función abierta, luego $p(W_y)=(p|^{V}_{V_y})(W_y)\subseteqab p(V_y)$; como $p(V_y)\subseteqab X$ entonces $p(W_y)\subseteqab X$ para cada $y\in V$, luego $\bigcup_{y\in V} p(W_y)\subseteqab X$. Ya que
         $$
         \begin{aligned}
            \bigcup_{y\in V}p(W_y)&=\bigcup_{y\in V}p(V\cap V_y)\\
                                  &=p\left( \bigcup_{y\in V}(V\cap V_y)\right)\\
                                  &=p\left( V\cap\bigcup_{y\in V}V_y\right)\\
                                  &=p(V),
         \end{aligned}
         $$
         pues $V\subseteq \bigcup_{y\in V}V_y$, entonces $p(V)\subseteqab X$. Obtenemos así que $p:Y\to X$ es una función abierta.
   \end{itemize}
\end{proof}
