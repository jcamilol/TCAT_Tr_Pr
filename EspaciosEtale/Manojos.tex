\begin{Def}[Manojo]
    Un manojo sobre un espacio topológico $X$ es una pareja $\langle Y, p\rangle$ con $Y$ un espacio topológico y $p:Y\to X$ una función continua.
\end{Def}
Según el contexto, es frecuente denotar al manojo $\langle Y,p\rangle$ simplemente por $p$. Las inversas puntuales $p^{-1}\left\lbrace x\right\rbrace$ ($x\in X$) son llamadas las fibras de $p$ sobre $X$. La colección de todos los manojos sobre un espacio topológico tiene estructura de categoría:
\begin{Def}[Categoría $\textbf{Top}/X$]
   La categoría $\Top /X$ (léase ``categoría Top sobre X") o $\Bund (X)$ tiene por objetos todos los manojos sobre $X$. Dados $\langle Y,p\rangle,\langle Y',q\rangle\in \Top/X$, $f$ es una flecha $\langle Y,p\rangle\to\langle Y',q\rangle$ en $\Top/X$ si $f:Y\to Y'$ es una función continua y $q\circ f=p$.
   \input{Diagramas/Diag9.tex}
\end{Def}
La categoría $\Top /X$ es un ejemplo de ``categoría sobre" (ver por ejemplo \cite[p.~45]{CWM} donde se denota por $(\Top\downarrow X)$). La notación $\Bund (X)$ es debido a \textit{bundle}, la traducción al inglés de la palabra \textit{manojo}. 
