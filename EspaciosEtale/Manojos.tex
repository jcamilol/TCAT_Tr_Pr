\begin{Def}[Manojo]
    Un manojo sobre un espacio topológico $X$ es una pareja $\langle Y, p\rangle$ con $Y$ un espacio topológico y $p:Y\to X$ una función continua.
\end{Def}
Según el contexto, es frecuente denotar al manojo $\langle Y,p\rangle$ simplemente por $p$. Las inversas puntuales $p^{-1}\left\lbrace x\right\rbrace$ ($x\in X$) son llamadas las fibras de $p$ sobre $X$. La colección de todos los manojos sobre un espacio topológico tiene estructura de categoría:
\begin{Def}[Categoría $\textbf{Top}/X$]
   La categoría $\Top /X$ (léase ``categoría Top sobre X") o $\Bund (X)$ tiene por objetos todos los manojos sobre $X$. Dados $\langle Y,p\rangle,\langle Y',q\rangle\in \Top/X$, $f$ es una flecha $\langle Y,p\rangle\to\langle Y',q\rangle$ en $\Top/X$ si $f:Y\to Y'$ es una función continua y $q\circ f=p$.
   \begin{center}
    % https://tikzcd.yichuanshen.de/#N4Igdg9gJgpgziAXAbVABwnAlgFyxMJZABgBpiBdUkANwEMAbAVxiRAE0QBfU9TXfIRQBGclVqMWbdgHJuvEBmx4CRUcPH1mrRCAAa3cTCgBzeEVAAzAE4QAtkjIgcEJKIna2l+VdsPETi5IAEzUWlK6aCDUDHQARjAMAAr8KkIg1lgmABY4PiA29m7UQYihHhEgAI6GXEA
\begin{tikzcd}
Y \arrow[r, "f"] \arrow[rd, "p"'] & Y' \arrow[d, "q"] \\
                                  & X                
\end{tikzcd}
\end{center}

\end{Def}
La categoría $\Top /X$ es un ejemplo de ``categoría sobre" (ver por ejemplo \cite[p.~45]{CWM} donde se denota por $(\Top\downarrow X)$). La notación $\Bund (X)$ es debido a \textit{bundle}, la traducción al inglés de la palabra \textit{manojo}. 
\begin{Def}[Homeomorfismo local]
   Dados $X$ y $Y$ espacios topológicos, decimos que una función $p:Y\to X$ es un homeomorfismo local si para todo $x\in X$ existe $U_x\in \mathcal{O}(X)$ con $x\in U_x$ tal que $p(U_x)\in\mathcal{O} (Y)$ y $p|^{X}_{U_x}:U_x\to p(U_x)$ es un homeomorfismo. 
\end{Def}
La siguiente proposición, cuya prueba se presenta en la sección \Ref{section:Apendice} (Apéndice) nos permite considerar a un homeomorfismo local como un caso especial de manojo.
\begin{Prop}
   Todo homeomorfismo es una función continua y abierta; además la familia $\left\lbrace U_x\right\rbrace_{x\in X}$ forma una base para la topología de $X$.
\end{Prop}
