Para introducir los espacios étalé recordamos un concepto de topología:
\begin{Def}[Homeomorfismo local]\label{Def:HomeomorfismoLocal}
   Dados $X$ y $Y$ espacios topológicos, decimos que una función $p:Y\to X$ es un homeomorfismo local sobre $X$ si para todo $y\in Y$ existe $V_y\subseteqab Y$ con $y\in V_y$ tal que:
   \begin{itemize}
      \item[(i)] $p(V_y)$ es un subconjunto abierto de $X$.
      \item[(ii)] $p|^{Y}_{V_y}:V_y\to p(V_y)$ es un homeomorfismo. 
   \end{itemize}
\end{Def}
Una propiedad importante de los homeomorfismos locales (cuya prueba se presenta en la Sección \Ref{section:Apendice} (Apéndice)) es la siguiente:
\begin{Prop}
   Si $p:Y\to X$ es un homeomorfismo local, entonces $p$ es una función continua y abierta. Además, la colección de todos los conjuntos abiertos de $Y$ que satisfacen \textit{(i)} y \textit{(ii)} de la Definición \Ref{Def:HomeomorfismoLocal} forman una base para la topología de $Y$. 
\end{Prop}
Con esto, podemos introducir los espacios fibrados o étalé como un caso particular de manojo:
\begin{Def}[Espacio étalé]
   Un espacio fibrado o étalé sobre un espacio topológico $X$ es un manojo $\langle p, Y\rangle$, donde $p: Y\to X$ es además un homeomorfismo local.
\end{Def}
Con lo anterior, podemos considerar la subcategoría plena de $\Top/X$ que tiene por objetos todos los espacios étalé sobre $X$, y que denotamos por $\Etale(X)$.

Si $p:Y\to X$ es un homeomorfismo local, las inversas puntuales $p^{-1}\left\lbrace x\right\rbrace$ ($x\in X$) son llamadas las fibras de $p$ sobre $X$. De este modo, el espacio $Y$ se presenta como la unión disyunta de las fibras de $p$, justificando así el uso de las palabras fibrado y étalé en las definiciones dadas. Las secciones de un espacio étalé se comportan especialmente bien; muestra de ello lo da la siguiente proposición:
\begin{Prop}
   Sean $U\subseteqab X$ y $s$ una sección transversal, de un espacio étalé $\langle p, Y\rangle$, sobre $U$. Entonces:
   \begin{itemize}
      \item $p|^{Y}_{s(U)}=s^{-1}$.
      \item Los espacios $U$ y $s(U)$ son homeomorfos.
      \item $s(U)$ es un subconjunto abierto de $Y$.
   \end{itemize}
\end{Prop}
