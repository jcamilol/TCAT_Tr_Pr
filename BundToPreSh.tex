(A lo largo de esta sección y la siguiente, $X$ denota un espacio topológico fijo).

En esta sección construimos y estudiamos un funtor $\Gamma$ de la categoría $\Top /X$ de los manojos sobre $X$, en la categoría $\PreSh (X)$ de prehaces sobre $X$.

\begin{Def}[Acción de $\Gamma_p$ sobre objetos]
   Sea $p:Y\to X$ un manojo sobre $X$. Para cada $U\in\mathcal{O}(X)$ definimos $\Gamma_p U$ como el conjunto de todas las secciones transversales de $p$ sobre $U$.
   \begin{center}
% https://tikzcd.yichuanshen.de/#N4Igdg9gJgpgziAXAbVABwnAlgFyxMJZABgBoBGAXVJADcBDAGwFcYkQBVEAX1PU1z5CKcqWLU6TVuwCaPPiAzY8BIqKo0GLNohAANef2VCiZcZqk6QAHWsBxegFtH9APpoABF24SYUAObwRKAAZgBOEI5IZCA4EEgATDQ49FiM7AAWEBAA1iA0AEYwYFDRNHAZWCE4SOS8oRFRiKKx8YhJktrsaIYg4ZFlrbU0jPRFjAAKAirCIFhg2LD5ndK6AHQbvf1NMXHDIEUlSADMxPV9jYN7zYXFpYgAtKfn21dtLYf3ACwAnD7cQA
\begin{tikzcd}
\Gamma_p U                                                                                                                         & Y \arrow[d, "p"] \\
U \arrow[r, hook, shift right] \arrow[ru, "..." description] \arrow[ru, bend left] \arrow[ru, bend right] \arrow[ru, bend left=49] & X               
\end{tikzcd}
\end{center}

\end{Def}
Notemos que si se tiene la flecha $V\subseteq U$ en $\mathcal{O}(X)$, para cada $s\in\Gamma_p U$ surge una sección transversal de $p$ sobre $V$, a saber, $s\circ \iota_{V,U}=s|^{U}_{V}$, es decir $s|^{U}_{V}\in\Gamma_p V$:
\begin{center}
% https://tikzcd.yichuanshen.de/#N4Igdg9gJgpgziAXAbVABwnAlgFyxMJZABgBoAmAXVJADcBDAGwFcYkQA1EAX1PU1z5CKAIykR1Ok1bsAqjz4gM2PASLlSxSQxZtEIAJoL+KoeorbpekAA0ekmFADm8IqABmAJwgBbJGRAcCCQAZhoceixGdgALCAgAaxAaOBisdxwkMSlddgAdPPwIgH1gDlIbbmSQRnoAIxhGAAUBVWEQTywnGMzeD28-RACgrPDI6P04xOrU9MzEbJ0ZfQKi+lLy2SqaWobm1rN9Tu7exS9ff3DgxA0QBrAoUICl6zgAHwA9YC2Nqr6Qc6DbIjG4pNIZUY5ZYgBD-QGQkFhQLjWLxJI0F75QoQErfCrbGr1RotUxqfSMGAQuEDJC3REYqzsND2bhAA
\begin{tikzcd}
                                                                                                                                        &                                                                  & Y \arrow[dd, "p"] \\
                                                                                                                                        & U \arrow[ru, "s", shift right] \arrow[rd, "{\iota_{U,X}}", hook] &                   \\
V \arrow[rr, "{\iota_{V,X}}"', hook, shift right] \arrow[ru, "{\iota_{V,U}}"', hook, shift right] \arrow[rruu, "s|^{U}_{V}", bend left] &                                                                  & X                
\end{tikzcd}
\end{center}

\begin{Def}[Acción de $\Gamma_p$ sobre flechas]
   Sea $p:Y\to X$ un manojo sobre $X$. Para cada flecha $V\subseteq U$ en $\mathcal{O}(X)$, definimos la flecha (de $\Set$) $\Gamma_p(V\subseteq U):\Gamma_p U\to \Gamma_p V:s\mapsto s|^{U}_{V}$.
\end{Def}
Las anteriores asignaciones de flechas y objetos hacen de $\Gamma_p$ un funtor contravariante $\Gamma_p:\mathcal{O}(X)^{\text{op}}\to \Set$, es decir, un prehaz sobre $X$; más aún, $\Gamma_p$ resulta ser un haz sobre $X$.
\begin{Prop}
   Para cada manojo $p:Y\to X$, $\Gamma_p$ es un haz sobre $X$.
\end{Prop}
\begin{proof}
   \begin{itemize}
      \item La prueba de que $\Gamma_p$ es un prehaz sobre $X$ es similar a la que se da para la Proposición ($\Ref{Prop:P1}$).
      \item Sean $U\in \mathcal{O}(X)$ y $\left\lbrace U_i\right\rbrace_{i\in I}$ un cubrimiento abierto de $U$. El siguiente diagrama en $\Set$ es un igualador:
         \begin{center}
\begin{tikzcd}
\Gamma_p U \arrow[r, "e"] & \prod_{i\in I}\Gamma_p U_i \arrow[r, "\pi_2"', shift right] \arrow[r, "\pi_1", shift left] & {\prod_{\left(i,j\right)\in I\times I}\Gamma_p (U_i\cap U_j)}
\end{tikzcd}
\end{center}

         La prueba de esto es similar a la que se da para la Proposición (\Ref{Prop:P2}), sumada al siguiente hecho: dada una familia $\left\lbrace g_i\right\rbrace_{i\in I}\in \prod_{i\in I}\Gamma_p U$, se tiene $p\circ \bigcup_{i\in I} g_i=\iota_{U,X}$: dada $x\in U=\bigcup_{i\in I}U_i$, existe $j\in I$ tal que $x\in U_j$; como $g_j:U_j\to Y$ es una sección transversal de $p$ sobre $U_j$, se tiene $p\circ g_j = \iota_{U_j,X}$, y por tanto
         $$
         \begin{aligned}
            \left(p\circ \bigcup_{i\in I}g_i\right)(x)&=p\left( \bigcup_{i\in I}g_i(x)\right)\\
                                           &=p(g_j(x))\\
                                           &=\iota_{U_j,X}(x)\\
                                           &=x\\
                                           &=\iota_{U,X}(x);
         \end{aligned}
         $$
         con esto $p\circ \bigcup_{i\in I}g_i=\iota_{U,X}$. Lo anterior se hace para garantizar $\bigcup_{i\in I}g_i\in \Gamma_p U$.
   \end{itemize}
   Por tanto $\Gamma_p$ es un haz sobre $X$.
\end{proof}
A los haces del tipo $\Gamma_p$, con $p:Y\to X$ un manojo sobre $X$, los llamamos \textit{haces de secciones transversales} sobre $X$. En ocaciones denotamos $\Gamma Y$ en lugar de $\Gamma_p$. Obtenemos por cada $p\in\Top /X$ un objeto $\Gamma_p \in \Sh (X)$. Ahora deseamos obtener, por cada flecha $f:\langle Y,p \rangle \to \langle Y',p'\rangle$ en $\Top /X$, una flecha $\Gamma_f:\Gamma_p\to\Gamma_p'$ de $\Sh(X)$, es decir, una transformación natural entre los funtores $\Gamma_p,\Gamma_p':\mathcal{O}(X)^\text{op}\to\Set$; para esto necesitamos asignar, para cada $U\in\mathcal{O}(X)$, una función $\Gamma_f U$ entre los conjuntos $\Gamma_p U$ y $\Gamma_p' U$, que además nos garantice que si $V\subseteq U$ es una flecha en $\mathcal{O}(X)$, entonces el siguiente diagrama de $\Set$ conmute:
% https://tikzcd.yichuanshen.de/#N4Igdg9gJgpgziAXAbVABwnAlgFyxMJZABgBpiBdUkANwEMAbAVxiRAB12BxOgW17oB9NAAIAaiAC+pdJlz5CKMgEYqtRizace-IaICqUmSAzY8BIgCZya+s1aIO3PgMHA0AckkTpsswqtSVWo7TUdtFyF3LxFDSTUYKABzeCJQADMAJwheJGVqHAgkMnV7LWddYQAKMU44JgAjOBgcGABHWIBKIwzs3MQAZgKixGsQBjoGmAYABTlzRRBMrCSACxwQEI0HJx1XaMkausbm1o79bt8QLJzi4aQx0J2IyvTxTZA4Vax0jcQSqZgKDFK43fr5ECFJBDUphXaRQRvQzUCZTWbzAKOZZrDbxSRAA
\begin{center}
\begin{tikzcd}
\Gamma_p V \arrow[rr, "\Gamma_f V"]                                      &  & \Gamma_{p'}V                                          \\
\Gamma_p U \arrow[u, "\Gamma_p(V\subseteq U)"] \arrow[rr, "\Gamma_f U"'] &  & \Gamma_{p'} U \arrow[u, "\Gamma_{p'}(V\subseteq U)"']
\end{tikzcd}
\end{center}

Notemos que la función $\Gamma_f U$ nos exige asignar a cada sección transversal $s$ de $p$ sobre $U$ (i.e. $s:U\to Y$ es una función continua y $p\circ s=\iota_{U,X}$) una sección transversal $\Gamma_f U (s)$ de $p'$ sobre $U$ (i.e. una función continua $\Gamma_f U (s):U\to Y'$ tal que $p'\circ \Gamma_f U(s)=\iota_{U,X}$). El siguiente diagrama que se forma en $\Top /X$
% https://tikzcd.yichuanshen.de/#N4Igdg9gJgpgziAXAbVABwnAlgFyxMJZABgBoBGAXVJADcBDAGwFcYkQA1EAX1PU1z5CKchWp0mrdgA0efEBmx4CRUcXEMWbRCACac-kqFEATKXU1NUnboDkPcTCgBzeEVAAzAE4QAtkjIQHAgkMwktdgReTx9-RDDgpABmS0ltEA8DDNikUSCQxBTw6wV7GkZ6ACMYRgAFAWVhEC8sZwALHCzvPwCaRMQ8nHosRnY2iAgAaxByqpr6oxUdFvbO1IidAB1N-CGAfWAOUmluLpz4voK8q3S0B24gA
\begin{center}
\begin{tikzcd}
                                                    & Y \arrow[r, "f"] \arrow[d, "p"] & Y' \\
U \arrow[ru, "s"] \arrow[r, "{\iota_{U,X}}"', hook] & X \arrow[ru, "p'"']             &   
\end{tikzcd}
\end{center}

nos muestra que $f\circ s$ es una función continua de $U$ en $Y$, y
$$
\begin{aligned}
   p'\circ (f\circ s)&=(p'\circ f)\circ s\\
                     &=p\circ s\\
                     &=\iota_{U,X},
\end{aligned}
$$
sugiriéndonos tomar $\Gamma_f U (s)=f\circ s$. Con esto tenemos
$$
\begin{aligned}
   (\Gamma_f V\circ\Gamma_p(V\subseteq U))(s) &=\Gamma_f V(\Gamma_p (V\subseteq U)(s))\\
                                              &= \Gamma_f V(s|^{U}_{V})\\
                                              &=f\circ s|^{U}_{V}\\
                                              &=f\circ (s\circ \iota_{V,U})\\
                                              &=(f\circ s)\circ \iota_{V,U}\\
                                              &=(f\circ s)|^{U}_{V}\\
                                              &=\Gamma_{p'}(V\subseteq U)(f\circ s)\\
                                              &=\Gamma_{p'}(V\subseteq U)(\Gamma_f U(s))\\
                                              &=(\Gamma_{p'}(V\subseteq U)\circ\Gamma_f U)(s),\\
\end{aligned}
$$
Y por tanto $\Gamma_f V\circ \Gamma_p(V\subseteq U)=\Gamma_{p'}(V\subseteq U)\circ \Gamma_f U$. Lo anterior prueba que $\Gamma_f$ es una transformación natural, para cada flecha $f$ en $\Top /X$. Estas asignaciones hacen de $\Gamma$ un funtor de $\Top /X$ en $\PreSh(X)$ (de hecho, en $\Sh(X)$, pero las razones de tomar la categoría de prehaces sobre $X$ como codominio de $\Gamma$ se entenderán más adelante):
\begin{Prop}
   Las asignaciones $p\mapsto \Gamma_p$ para cada $p\in \Top /X$ y $f\mapsto \Gamma_f$ para cada flecha $f$ en $\Top /X$, determinan un funtor $\Gamma:\Top /X\to\PreSh(X)$.   
\end{Prop}
\begin{proof}
   \begin{itemize}
      \item Veamos que $\Gamma$ respeta identidades. Sea $\langle Y, p\rangle\in \Top /X$. Debemos ver $\Gamma_{1_{p}}=1_{\Gamma_{p}}$, donde $1_{\Gamma_{p}}$ es la transformación natural identidad del funtor $\Gamma_p$ en sí mismo. Dados $U\in \mathcal{O}(X)$ y $s\in \Gamma_p U$, tenemos que $\Gamma_{1_p} U$ es una función de $\Gamma_p U$ en $\Gamma_p U$ y $\Gamma_{1_p}U(s)=1_p\circ s=s$; por tanto $\Gamma_{1_p}U=1_{\Gamma_p U}=1_{\Gamma_p}U$. Como lo anterior se tiene para $U\in \mathcal{O}(X)$ arbitrario, se sigue que $\Gamma_{1_p}=1_{\Gamma_p}$.
      \item Veamos que $\Gamma$ respeta composiciones. Supongamos que tenemos en $\Top /X$ el siguiente diagrama conmutativo
         % https://tikzcd.yichuanshen.de/#N4Igdg9gJgpgziAXAbVABwnAlgFyxMJZABgBpiBdUkANwEMAbAVxiRDRAF9T1Nd9CKMgCYqtRizZoA5NK492fPASIBGUqrH1mrROzmcxMKAHN4RUADMAThAC2SMiBwQk68TrYmAOt4DGWNZ+AASWINQMdABGMAwACkoCbNZYJgAWOPJWtg6ITi5IwtTaknph3Nn2hdQFiO4luiAmXBScQA
\begin{center}
\begin{tikzcd}
p \arrow[dd, "g\circ f"'] \arrow[rd, "f"] &                    \\
                                          & p' \arrow[ld, "g"] \\
p''                                       &                   
\end{tikzcd}
\end{center}

         y veamos que en $\PreSh(X)$ el siguiente diagrama conmuta:
         % https://tikzcd.yichuanshen.de/#N4Igdg9gJgpgziAXAbVABwnAlgFyxMJZABgBpiBdUkANwEMAbAVxiRAB12BxOgW17oB9NCAC+pdJlz5CKAIyk5VWoxZtOPfkOBoA5KLESQGbHgJEyAJmX1mrRB258Bgnbv1jlMKAHN4RUAAzACcIXiQyEBwIJAUVO3UnLVdAg3Eg0PDEOOikS2pbNQcNZyEfQwywiOpcxHz4osdNF2AfTgBjLGD2gAJUkGoGOgAjGAYABSkzWRBgrB8ACxxPUSA
\begin{center}
\begin{tikzcd}
\Gamma_p \arrow[rd, "\Gamma_{f}"] \arrow[dd, "\Gamma_{g\circ f}"'] &                                    \\
                                                                   & \Gamma_{p'} \arrow[ld, "\Gamma_g"] \\
\Gamma_{p''}                                                       &                                   
\end{tikzcd}
\end{center}

         Sean $U\in\mathcal{O}(X)$ y $s\in\Gamma_p U$. Tenemos
         $$
         \begin{aligned}
            (\Gamma_{g\circ f}U)(s)&=(g\circ f)\circ s\\
                                   &=g\circ(f\circ s)\\
                                   &=(\Gamma_{g}U)(f\circ s)\\
                                   &=(\Gamma_{g}U)(\Gamma_{f}U(s))\\
                                   &=(\Gamma_{g}U\circ\Gamma_{f}U)(s).
         \end{aligned}
         $$
         Por tanto $\Gamma_{g\circ f}=\Gamma_{g}U\circ\Gamma_{f}U$; como esto vale para $U\in\mathcal{O}(X)$ arbitrario, se sigue $\Gamma_{g\circ f}=\Gamma_{g}\circ\Gamma_{f}$.
   \end{itemize}
   Lo anterior prueba que $\Gamma$ es un funtor de $\Top /X$ en $\PreSh(X)$.
\end{proof}
