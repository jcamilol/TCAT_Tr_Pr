(A lo largo de esta sección $X$ denota un espacio topológico fijo).

En esta sección construimos y estudiamos un funtor $\Gamma$ de la categoría $\Top /X$ de los manojos sobre $X$, en la categoría $\PreSh (X)$ de prehaces sobre $X$.

\begin{Def}[Acción de $\Gamma_p$ sobre objetos]
   Sea $p:Y\to X$ un manojo sobre $X$. Para cada $U\in\mathcal{O}(X)$ definimos $\Gamma_p U$ como el conjunto de todas las secciones transversales de $p$ sobre $U$.
   \begin{center}
% https://tikzcd.yichuanshen.de/#N4Igdg9gJgpgziAXAbVABwnAlgFyxMJZABgBoBGAXVJADcBDAGwFcYkQBVEAX1PU1z5CKcqWLU6TVuwCaPPiAzY8BIqKo0GLNohAANef2VCiZcZqk6QAHWsBxegFtH9APpoABF24SYUAObwRKAAZgBOEI5IZCA4EEgATDQ49FiM7AAWEBAA1iA0AEYwYFDRNHAZWCE4SOS8oRFRiKKx8YhJktrsaIYg4ZFlrbU0jPRFjAAKAirCIFhg2LD5ndK6AHQbvf1NMXHDIEUlSADMxPV9jYN7zYXFpYgAtKfn21dtLYf3ACwAnD7cQA
\begin{tikzcd}
\Gamma_p U                                                                                                                         & Y \arrow[d, "p"] \\
U \arrow[r, hook, shift right] \arrow[ru, "..." description] \arrow[ru, bend left] \arrow[ru, bend right] \arrow[ru, bend left=49] & X               
\end{tikzcd}
\end{center}

\end{Def}
Notemos que si se tiene la flecha $V\subseteq U$ en $\mathcal{O}(X)$, para cada $s\in\Gamma_p U$ surge una sección transversal de $p$ sobre $V$, a saber, $s\circ \iota_{V,U}=s|^{U}_{V}$, es decir $s|^{U}_{V}\in\Gamma_p V$:
\begin{center}
% https://tikzcd.yichuanshen.de/#N4Igdg9gJgpgziAXAbVABwnAlgFyxMJZABgBoAmAXVJADcBDAGwFcYkQA1EAX1PU1z5CKAIykR1Ok1bsAqjz4gM2PASLlSxSQxZtEIAJoL+KoeorbpekAA0ekmFADm8IqABmAJwgBbJGRAcCCQAZhoceixGdgALCAgAaxAaOBisdxwkMSlddgAdPPwIgH1gDlIbbmSQRnoAIxhGAAUBVWEQTywnGMzeD28-RACgrPDI6P04xOrU9MzEbJ0ZfQKi+lLy2SqaWobm1rN9Tu7exS9ff3DgxA0QBrAoUICl6zgAHwA9YC2Nqr6Qc6DbIjG4pNIZUY5ZYgBD-QGQkFhQLjWLxJI0F75QoQErfCrbGr1RotUxqfSMGAQuEDJC3REYqzsND2bhAA
\begin{tikzcd}
                                                                                                                                        &                                                                  & Y \arrow[dd, "p"] \\
                                                                                                                                        & U \arrow[ru, "s", shift right] \arrow[rd, "{\iota_{U,X}}", hook] &                   \\
V \arrow[rr, "{\iota_{V,X}}"', hook, shift right] \arrow[ru, "{\iota_{V,U}}"', hook, shift right] \arrow[rruu, "s|^{U}_{V}", bend left] &                                                                  & X                
\end{tikzcd}
\end{center}

\begin{Def}[Acción de $\Gamma_p$ sobre flechas]
   Sea $p:Y\to X$ un manojo sobre $X$. Para cada flecha $V\subseteq U$ en $\mathcal{O}(X)$, definimos la flecha (de $\Set$) $\Gamma_p(V\subseteq U):\Gamma_p U\to \Gamma_p V:s\mapsto s|^{U}_{V}$. 
\end{Def}
Las anteriores asignaciones de flechas y objetos hacen de $\Gamma_p$ un funtor contravariante $\Gamma_p:\mathcal{O}(X)^{\text{op}}\to \Set$, es decir, un prehaz sobre $X$; más aún, $\Gamma_p$ resulta ser un haz sobre $X$.
\begin{Prop}
   Para cada manojo $p:Y\to X$, $\Gamma_p$ es un haz sobre $X$.
\end{Prop}
\begin{proof}
   \begin{itemize}
      \item La prueba de que $\Gamma_p$ es un prehaz sobre $X$ es similar a la que se da para la Proposición ($\Ref{Prop:P1}$).
      \item Sean $U\in \mathcal{O}(X)$ y $\left\lbrace U_i\right\rbrace_{i\in I}$ un cubrimiento abierto de $U$. El siguiente diagrama en $\Set$ es un igualador:
         \begin{center}
\begin{tikzcd}
\Gamma_p U \arrow[r, "e"] & \prod_{i\in I}\Gamma_p U_i \arrow[r, "\pi_2"', shift right] \arrow[r, "\pi_1", shift left] & {\prod_{\left(i,j\right)\in I\times I}\Gamma_p (U_i\cap U_j)}
\end{tikzcd}
\end{center}

         La prueba de esto es similar a la que se da para la Proposición (\Ref{Prop:P2}), sumada al siguiente hecho: dada una familia $\left\lbrace g_i\right\rbrace_{i\in I}\in \prod_{i\in I}\Gamma_p U$, se tiene $p\circ \bigcup_{i\in I} g_i=\iota_{U,X}$: dada $x\in U=\bigcup_{i\in I}U_i$, existe $j\in I$ tal que $x\in U_j$; como $g_j:U_j\to Y$ es una sección transversal de $p$ sobre $U_j$, se tiene $p\circ g_j = \iota_{U_j,X}$, y por tanto
         $$
         \begin{aligned}
            \left(p\circ \bigcup_{i\in I}g_i\right)(x)&=p\left( \bigcup_{i\in I}g_i(x)\right)\\
                                           &=p(g_j(x))\\
                                           &=\iota_{U_j,X}(x)\\
                                           &=x\\
                                           &=\iota_{U,X}(x);
         \end{aligned}
         $$
         con esto $p\circ \bigcup_{i\in I}g_i=\iota_{U,X}$. Lo anterior se hace para garantizar $\bigcup_{i\in I}g_i\in \Gamma_p U$.
   \end{itemize}
   Por tanto $\Gamma_p$ es un haz sobre $X$.
\end{proof}

